\documentclass[12pt]{article}
\usepackage{graphicx}
\usepackage{latexsym, amsmath,amsfonts,amssymb}
\usepackage{xcolor}
\usepackage{wrapfig}
\usepackage[margin=.75in]{geometry}
\usepackage{kpfonts}  %Changing the default fonts
\usepackage[T1]{fontenc}
\setlength{\parskip}{1.2ex} %space between paragraphs
\setlength{\parindent}{1em} %Paragraph indentation
\clubpenalty = 10000
\widowpenalty = 10000
\newcommand\T{\rule{0pt}{2ex}} % \T will create extra space above (used to fix tables)
\newcommand\B{\rule[-1.5ex]{0pt}{0pt}}% \B will create extra space below (used to fix tables)
\linespread{1.25} %spacing between lines
\pagestyle{empty}
\begin{document}
\begin{center}
\textbf{{\Large Inductive Versus Deductive Reasoning}}
\end{center}
A simple illustration of inductive versus deductive reasoning can
demonstrate the power of deductive reasoning. Your students will
need some background on conjectures, counterexamples, and
theorems. Knowing that inductive reasoning isn't foolproof (as
we saw in ``\ldots is ambiguous'') is essential; it allows you to 
compare it with deductive reasoning, which is perfect.

\begin{center}
\begin{tabular}{cccc}
$1+2+3=2(3)$& $2+3+4=3(3)$& $3+4+5=4(3)$ & $4+5+6=5(3)$\\
$5+6+7=6(3)$&$6+7+8=7(3)$&$7+8+9=8(3)$ & $8+9+10=9(3)$\\
$9+10+11=10(3)$&$10+11+12=11(3)$&$11+12+13=12(3)$ & $12+13+14=13(3)$\\
\end{tabular}
\end{center}
Stress that inductive reasoning is what we use to formulate a 
conjecture. It shouldn't be that difficult for them to see:\\\\
\textbf{Conjecture: }The sum of $3$ consecutive integers is equal to $3$ multiplied by the middle number.\\\\
Note that no computer or calculator can help us solve the problem
completely because there are an infinite number of equations 
to check. Deductive reasoning, however, allows us to prove 
the result for an infinite number of equations in a finite amount of 
time.\\\\
\textbf{Proof:} Any $3$ consecutive numbers can be written 
as $n$, $n+1$, and $n+2$ for some $n$. The sum is then 
$n+(n+1)+(n+2)$ which simplifies to $3n+3=3(n+1)$. That is, 
the sum is equal to $3$ multiplied by the middle number.\\\\
Depending on the quality of your students you'll need to provide
more details. Going from $3+4+5$ to $n+(n+1)+(n+2)$ is a jump
that I found many of my students weren't ready for.
\end{document}