\documentclass[12pt]{article}
\usepackage{graphicx}
\usepackage{latexsym, amsmath,amsfonts,amssymb}
\usepackage{xcolor}
\usepackage[margin=.75in]{geometry}
\usepackage{kpfonts}  %Changing the default fonts
\usepackage[T1]{fontenc}
\setlength{\parskip}{1.2ex} %space between paragraphs
\setlength{\parindent}{1em} %Paragraph indentation
\clubpenalty = 10000
\widowpenalty = 10000
\newcommand\T{\rule{0pt}{2ex}} % \T will create extra space above (used to fix tables)
\newcommand\B{\rule[-1.5ex]{0pt}{0pt}}% \B will create extra space below
\linespread{1.25} %spacing between lines
\pagestyle{empty} %remove page numbers
\begin{document}
\begin{center}
\Large{\textbf{Law of Sines Proof}}
\end{center}

\noindent {\bf Law of Sines:} For $\triangle ABC$ with sides $a, b, c$: $\frac{a}{\sin(A)}=\frac{b}{\sin(B)}=\frac{c}{\sin(C)}$.\\\\
{\bf Proof:} There are two cases to consider depending on whether the triangle
is acute or obtuse.\\\\
Case $1$: $\triangle ABC$ is acute.\\\\
Arrange the triangle so that the base is $\overline{AB}$ and drop a perpendicular from point $C$ to $\overline{AB}$ to create point $D$ as shown in the following diagram:
\begin{figure}[h]
\begin{center}
\includegraphics {AcuteTri.pdf}
\caption{Proving the Law of Sines with an acute triangle.} \label{fg:lofs}
\end{center}
\end{figure}

\noindent From $\triangle ADC$ we have $\sin(A)=\frac{h_1}{b}$ and from $\triangle BDC$,  $\sin(B)=\frac{h_1}{a}$. Solving for $h_1$ yields $h_1=b\sin(A)$ and $h_1=a\sin(B)$. Combining the two equations 
gives $b\sin(A)=a\sin(B)$ which simplifies to $\frac{a}{\sin(A)}=\frac{b}{\sin(B)}$. Now rotate the triangle so that $\overline{AC}$ is the base and
drop a perpendicular from $B$ to $\overline{AC}$ to create $D$ as shown:
\begin{figure}[h]
\begin{center}
\includegraphics {AcuteTriB.pdf}
\end{center}
\end{figure}

\noindent Follow the same argument as above to get $\sin(C)=\frac{h_2}{a}$ and $\sin(A)=\frac{h_2}{c}$. Solving for $h_2$ gives $h_2=c\sin(A)$ and $h_2=a\sin(C)$. These two equations yield $a\sin(C)=c\sin(A)$ which
simplifies to $\frac{a}{\sin(A)}=\frac{c}{\sin(C)}$. Combine with
$\frac{a}{\sin(A)}=\frac{b}{\sin(B)}$ to get 
$\frac{a}{\sin(A)}=\frac{b}{\sin(B)}=\frac{c}{\sin(C)}$.\\\\
Case $2$: $\triangle ABC$ is obtuse.\\\\
Arrange the triangle so that the base is the side opposite the obtuse angle, $C$. Drop a perpendicular to the other side as in the following diagram:
\begin{figure}[h]
\begin{center}
\includegraphics {ObtuseTri.pdf}
\caption{Proving the Law of Sines with an obtuse triangle.} \label{fg:lofs2}
\end{center}
\end{figure}
\noindent From $\triangle ADC$ we have $\sin(A)=\frac{h_1}{b}$ and from $\triangle BDC$,  $\sin(B)=\frac{h_1}{a}$. Solving for $h_1$ in both equations yields $h_1=b\sin(A)$ and $h_1=a\sin(B)$ which means $b\sin(A)=a\sin(B)$. This is equivalent to $\frac{a}{\sin(A)}=\frac{b}{\sin(B)}$. Now rotate the triangle so that $\overline{AC}$ is the base and
drop a perpendicular from $B$ to $\overline{AC}$ to create point $D$ as shown in the diagram below:
\begin{figure}[h]
\begin{center}
\includegraphics {ObtuseTriB.pdf}
\end{center}
\end{figure}

\noindent Since $C$ is an obtuse angle $\sin(C)=\sin(180^{\circ}-C)=\frac{h_2}{a}$; that is, $\sin(C)=\frac{h_2}{a}$. Combine this with $\sin(A)=\frac{h_2}{c}$ to get $c\sin(A)=a\sin(C)$. This is 
equivalent to  $\frac{a}{\sin(A)}=\frac{c}{\sin(C)}$ which, when
combined with $\frac{a}{\sin(A)}=\frac{b}{\sin(B)}$ gives $\frac{a}{\sin(A)}=\frac{b}{\sin(B)}=\frac{c}{\sin(C)}$.

\noindent In both cases, the Law of Sines is established.

\end{document}