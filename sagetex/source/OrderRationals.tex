\documentclass[12pt]{article}
\usepackage{latexsym, amsmath,amsfonts,amssymb}
\usepackage{xcolor}
\usepackage[margin=.75in]{geometry}
\usepackage{kpfonts}  %Changing the default fonts
\usepackage[T1]{fontenc}
\setlength{\parskip}{1.2ex} %space between paragraphs
\setlength{\parindent}{1em} %Paragraph indentation
\linespread{1.4} %spacing between lines
\pagestyle{empty}
\begin{document}
\begin{center}
\textbf{{\Large Ordering the Rationals in $[0,1]$}}
\end{center}
The ordering of the rationals in $[0,1]$ depends on a series result
posted earlier:
 \[\sum_{i=1}^{n}i=\frac{n(n+1)}{2}\]
Next, list all the rational numbers in $(0,1]$:\\\\
$\frac{1}{1}$\\
$\frac{1}{2}\hspace{2ex} \frac{2}{2}$\\
$\frac{1}{3}\hspace{2ex} \frac{2}{3} \hspace{2ex} 
\frac{3}{3}$\\
$\frac{1}{4}\hspace{2ex} \frac{2}{4} \hspace{2ex} 
\frac{3}{4} \hspace{2ex} \frac{4}{4}$\\
$\frac{1}{5}\hspace{2ex} \frac{2}{5} \hspace{2ex} 
\frac{3}{5}\hspace{2ex}\frac{4}{5}\hspace{2ex}\frac{5}{5}$\\
$\vdots  \hspace{2ex}\cdots$\\\\
Explain how every rational number in the interval $(0,1]$
has been represented because every possible denominator is listed 
and, for each denominator, every possible numerator is listed that 
results in a number bigger than $0$ and no bigger than $1$. Note
that, by examining the first column, the number of rationals in 
$(0,1]$ is infinite since $\frac{1}{1}>\frac{1}{2}> \frac{1}{3}>
 \frac{1}{4}>\frac{1}{5}> \cdots$ so the only question is whether
 every rational number can be included in some list. There are 
 some rational numbers that get listed more than once 
 $\left(\mbox{e.g. }\frac{1}{1}=\frac{2}{2}=\frac{3}{3}\right)$
 but clearly if
every rational number above can be listed then all the rationals in
 $(0,1]$ will be listed as well. Finally, notice that there is 1 term in 
the first row, 2 terms in the second row, 3 terms in the third row and, 
in general, $n$ terms are in the $\mbox{n}^{\tiny\mbox{th}}$ row. 
Now write the terms row by row to create the 
sequence: $\frac{1}{1}, \frac{1}{2},
\frac{2}{2}, \frac{1}{3}, \frac{2}{3}, \frac{3}{3}, \frac{1}{4},
\frac{2}{4}, \frac{3}{4}, \frac{4}{4}, \frac{1}{5}, \frac{2}{5},
\frac{3}{5}, \frac{4}{5}, \frac{5}{5}, \cdots$\\\\
WARNING: At this point you've gotten the basic idea across to 
your class that all the rational in $[0,1]$ can be listed.
The details that follow have been included because the lesson is
built from pieces put together over a long period of time
before they finally get put together so they aren't essential to the
explanation. When you're teaching or reviewing functions you could, 
for example, work in an example using the functions below. That 
way, when it's finally time to demonstrate the rationals have 
measure $0$ you can simply remind the class of math done 
previously. Unless you've got an excellent class, this information
may be too complicated.

Define the function from the different fractional representations
of rationals in $(0,1]$ to the positive integers by 
$f\left(\frac{m}{n}\right)=m+\sum_{i=1}^{n-1} i$.
It's easy to see why this formula works; the summation 
$\sum_{i=1}^{n-1} i$ counts up all the terms that 
appeared in earlier rows and adds it to $m$, the number of
terms in the current row. Use the series formula to rewrite the 
function as $f\left(\frac{m}{n}\right)=m+\frac{(n-1)n}{2}$. 
Show how this function gives:
$f\left(\frac{1}{1}\right)=1+\frac{(0)1}{2}=1, 
f\left(\frac{1}{2}\right)=1+\frac{(1)2}{2}=2,
f\left(\frac{2}{2}\right)=2+\frac{(1)2}{2}=3,
f\left(\frac{1}{3}\right)=1+\frac{(2)3}{2}=4,
f\left(\frac{2}{3}\right)=2+\frac{(2)3}{2}=5$, and so on. All
the rationals in $(0,1]$ have been listed. To make sure we get the
number of $0$ as well, just put it at the beginning of the sequence
to get $ 0, \frac{1}{1}, \frac{1}{2}, \frac{2}{2}, 
\frac{1}{3}, \frac{2}{3}, \frac{3}{3}, \frac{1}{4},
\frac{2}{4}, \frac{3}{4}, \frac{4}{4}, \frac{1}{5}, \frac{2}{5},
\frac{3}{5}, \frac{4}{5}, \frac{5}{5}, \cdots$

Now define $g(0)=1$ and 
$g\left(\frac{m}{n}\right)=m+\frac{(n-1)n}{2} +1$. 
Work through the following:
$g(0)=1$\\
$g\left(\frac{1}{1}\right)=1+\frac{(0)1}{2} +1=2, 
g\left(\frac{1}{2}\right)=1+\frac{(1)2}{2} +1=3,
g\left(\frac{2}{2}\right)=2+\frac{(1)2}{2}+1=4,
g\left(\frac{1}{3}\right)=1+\frac{(2)3}{2}+1=5,
g\left(\frac{2}{3}\right)=2+\frac{(2)3}{2}+1=6$

This function, $g$, shows (but doesn't prove) that the rational 
numbers in $[0,1]$ can be listed. 
\end{document}