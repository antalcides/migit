\documentclass{article}
\title{Examples of embedding Sage output in \LaTeX\ documents}
\author{}

\usepackage{sagetex}
\usepackage{fullpage}
\usepackage{verbatim}

\def\bs{\ensuremath{\backslash}}

\begin{document}
\maketitle

\section{Overview}

\begin{enumerate}
\item
Get ahold of the files \texttt{sagetex.sty} and \texttt{sagetex.py}.  
They can be found in the Sage directory
(\texttt{\$SAGEROOT/examples/latex\underline{ }embed/}), or on the CTAN website
(\texttt{http://www.ctan.org}). Put these files in the same directory as your
.tex file.

\item
In your .tex file, add the following line to the preamble:
\begin{quote}
\texttt{{$\backslash$}usepackage\{sagetex\}}
\end{quote}

\item
Add some Sage code to your article (see below for examples).

\item
Compile:
\begin{itemize}
\item \texttt{latex filename.tex} (or use pdflatex)
\item \texttt{sage filename.sage}
\item \texttt{pdflatex filename.tex}
\end{itemize}
Note that once you've run Sage, you don't need to run it again unless you've
changed the Sage code in your article.
\end{enumerate}

\section{\texttt{{\bs}sage}}

Use the command \texttt{$\backslash$sage} to include the output of Sage
commands inline. For example, the following line
\begin{quote}
\begin{verbatim}
The second derivative of $\sage{x*sin(x^2)}$
is $\sage{diff(x*sin(x^2),x,2)}$.
\end{verbatim}
\end{quote}
Will typeset the following:
\begin{quote}
The second derivative of $\sage{x*sin(x^2)}$
is $\sage{diff(x*sin(x^2),x,2)}$.
\end{quote}

\section{\texttt{{\bs}sageblock}}

The \texttt{sageblock} environment allows you to display \emph{and}
execute some Sage code.
\begin{sageblock}
 var('a,b,c')
 eqn = a*x^2 + b*x + c
 s = solve(eqn, x)
\end{sageblock}
The variables \texttt{eqn} and \texttt{s} are available throughout
the document now. For example:

\begin{quote}
\begin{verbatim}
Solutions of $\mbox{eqn}=\sage{eqn}$:
\begin{displaymath}
\sage{s[0]}, \sage{s[1]}
\end{displaymath}
\end{verbatim}
\end{quote}
outputs:
\begin{quote}
Solutions of $\mbox{eqn}=\sage{eqn}$:
\begin{displaymath}
\sage{s[0]}, \sage{s[1]}
\end{displaymath}
\end{quote}

Note that you can do anything in a code block that you can do in Sage and/or
Python. 

\section{\texttt{{\bs}sagesilent}}

The \texttt{sagesilent} environment executes some Sage code, but does not
display the code. For example,

\begin{quote}
\begin{verbatim}
The following code block doesn't appear in the typeset file\dots
\begin{sagesilent}
  var('x')
  f = log(sin(x)/x)
\end{sagesilent}
but we can refer to whatever we did in that code block:
The Taylor Series of $f(x)=\sage{f}$ is: $\sage{ f.taylor(x, 0, 10) }$.
\end{verbatim}
\end{quote}
typesets the following.

\begin{quote}
The following code block doesn't appear in the typeset file\dots
\begin{sagesilent}
  var('x')
  f = log(sin(x)/x)
\end{sagesilent}
but we can refer to whatever we did in that code block:
The Taylor Series of $f(x)=\sage{f}$ is: $\sage{ f.taylor(x, 0, 10) }$.
\end{quote}

\section{\texttt{{\bs}sageplot}}

To plot images, use the \texttt{sageplot} command.
The following \LaTeX\ code will include the image below.
\begin{quote}
\begin{verbatim}
\sageplot{graphs.FlowerSnark().plot()}
\end{verbatim}
\end{quote}

\sageplot{graphs.FlowerSnark().plot()}

\end{document}
