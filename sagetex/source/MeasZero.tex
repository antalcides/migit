\documentclass[12pt]{article}
\usepackage{latexsym, amsmath,amsfonts,amssymb}
\usepackage{xcolor}
\usepackage[margin=.75in]{geometry}
\usepackage{kpfonts}  %Changing the default fonts
\usepackage[T1]{fontenc}
\setlength{\parskip}{1.2ex} %space between paragraphs
\setlength{\parindent}{1em} %Paragraph indentation
\linespread{1.4} %spacing between lines
\pagestyle{empty}
\begin{document}
\begin{center}
\textbf{{\Large The Rational Numbers Have Measure Zero}}
\end{center}
The TED-Ed video ``How Big is Infinity'' illustrates how to list
the positive rational numbers: $\frac{1}{1}, \frac{1}{2},
\frac{2}{1}, \frac{3}{1}, \frac{2}{2}, \frac{1}{3}, \frac{1}{4},
\ldots$ and this idea can be extended to all rational numbers 
by including $0$ as well as the negative rational numbers:
$0,\frac{1}{1}, -\frac{1}{1}, \frac{1}{2}, -\frac{1}{2},
\frac{2}{1}, -\frac{2}{1}, \frac{3}{1}, -\frac{3}{1}, \frac{2}{2}, -\frac{2}{2}, \frac{1}{3}, -\frac{1}{3}, \frac{1}{4}, 
-\frac{1}{4}, \ldots$
On a number line where the integers are $1$ centimeter apart,
cover $0$ with a piece of paper of length $.1$ centimeters, 
$\frac{1}{1}$ with a paper of length $.01$ centimeters, 
$-\frac{1}{1}$ with a paper of length $.001$ centimeters, and
so on. In general, cover the $\mbox{i}^{\tiny\mbox{th}}$
rational number with a piece of paper of length $10^{-i}$. 
The result is $.1+.01+.001+\cdots$ which all of your students
should agree will have length $\overline{.1}$ centimeters. 
Stop the class to summarize what's happened: the number line
had an infinite length yet the rational numbers can be covered
with just $\overline{.1}$ centimeters of paper. What's not 
covered by the paper? Most (but not all) of the irrational numbers
because each piece of paper covering a rational number also 
covers lots of other rational and irrational numbers. Even though 
it seems obvious that every number is rational
the infinite length is due the irrational numbers it should now be 
clear: \textbf{proof is necessary, even when things seem obvious.}
Seeing how small the rational numbers are compared to the 
irrational numbers is an essential lesson every high school student 
should learn. 

For students who have knowledge of limits, start with the 
infinite geometric series formula:
\[\sum_{i=1}^{\infty}ar^{n-1}=\frac{a}{1-r} \hspace{4ex}\mbox{for }|r|<1\]
and note that $.1+.01+.001+\cdots$ is an infinite geometric
series with $a=.1$ and $r=.1$ hence the total length of paper used 
is $\frac{.1}{1-.1}=\frac{1}{9}$ centimeters. But there
was no need to use that much paper since a single rational 
number is a point on the number line and points have no 
length. Cover $0$ with a piece of paper of length $.01$ centimeters, 
$\frac{1}{1}$ with a paper of length $.001$ centimeters, 
$-\frac{1}{1}$ with a paper of length $.0001$ centimeters, and
so on. In general, cover the $\mbox{i}^{\tiny\mbox{th}}$
rational number with a piece of paper of length $10^{-i-1}$. The
rational numbers are now covered with paper having total length
$\frac{.01}{1-.1}=\frac{1}{90}$ centimeters. Of course, you can 
improve on this further, so that you cover $0$ with a piece of paper 
of length $.001$ centimeters, $\frac{1}{1}$ with a paper of length 
$.0001$ centimeters, $-\frac{1}{1}$ with a paper of length 
$.00001$ centimeters, and so on. In general, cover the 
$\mbox{i}^{\tiny\mbox{th}}$ rational number with a piece of 
paper of length $10^{-i-2}$ and the rational numbers are now 
covered with paper having total length
$\frac{.001}{1-.1}=\frac{1}{900}=.00\overline{1}$ centimeters.
This argument can continue forever, so the amount of paper 
needed to cover the rationals is 
$\lim_{n\to \infty}\frac{1}{9}\frac{1}{10^n}=0$. That's why 
it's called measure $0$; the length is $0$.
\end{document}