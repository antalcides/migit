\documentclass[12pt]{article}
\usepackage{latexsym, amsmath,amsfonts,amssymb}
\usepackage[margin=.75in]{geometry}
\usepackage{kpfonts}  %Changing the default fonts
\usepackage[T1]{fontenc}
\setlength{\parskip}{1.2ex} %space between paragraphs
\setlength{\parindent}{1em} %Paragraph indentation
\linespread{1.25} %spacing between lines
\pagestyle{empty} %no page numbers
\begin{document}
\begin{center}
\Large{\textbf{Coins}}
\end{center}
\textbf{Rules:} Start with 2 players and a pile of 20 coins.
The players take turns deciding whether to remove 1, 2, or 3 coins 
from the pile. The player who takes the last coin \emph{loses}. \\

You can implement this game, which is really Nim with one heap,
into your classroom in many ways. Here's one method. Start by
drawing 20 circles onto the whiteboard, explaining the rules, and  
playing several sample games. Afterwards, coins becomes a 
warmup activity for your class, but make sure you start with a 
different number of coins each and give your class the choice on 
whether they move first or second. This prevents students from 
copying your play from earlier games as well as making it more 
difficult to figure out a winning strategy. After playing the game 
many times until the class understands the rules, pose questions
like these:
\begin{enumerate}
\item If the pile has 15 coins is it better to go first? What if the 
pile had 37 coins? 1256 coins?
\item Suppose we change the rules so that the number of coins
we can take is either 1 or 2? Now what happens for 37 coins?
\item Suppose we change the rules so that the player who gets 
the last coin wins. Who will win if the game has 53 coins?
\end{enumerate}

Budget some time for them to think and respond to your questions.
Be ready to critique the answers, too. After giving adequate time,
explain that you can prove the answer to these and other questions
by thinking like a mathematician. All we need to do is think 
backwards and assume our opponent always plays the best move.
Here's the explanation for a game with 13 coins where each
player choses either 1, 2, or 3 coins each turn, and the player who
takes the last coin loses. Talk through the logic while you 
fill in the chart below.

If there is 1 coin left on a player's turn then they must take the 
last coin, so \textbf{1 coin is a losing position}. If there are 2 
coins left on a player's turn then the best move is to choose 1 coin 
leaving the opponent with 1 coin (a losing position). Therefore, 
\textbf{2 coins is a winning position}. If there are 3 coins left on a 
player's turn then the best move is to choose 2 coins leaving the 
opponent with 1 coin (which is losing) so 
\textbf{3 coins is a winning position}. If there are 4 coins on a 
player's turn then choosing 3 coins leaves
the opponent with 1 coin, so \textbf{4 coins is a winning position}. If
there are are 5 coins then choosing 1, 2, or 3 coins leaves the 
opponent with 4, 3, or 2 coins. These are all winning position if our
opponent plays the best. Therefore, \textbf{5 coins is a losing 
position}. If there are 6 coins then choosing 1 coin gives our
opponent a losing position, so 
\textbf{6 coins is a winning position}. If there are 7 coins then 
choosing 2 coins gives our opponent 5 coins (a losing position), so 
\textbf{7 coins is a winning position}. If there are 8 coins then 
choosing 3 coins gives our opponent 5 coins (a losing position), so 
\textbf{8 coins is a winning position}. If there are 9 coins then it
doesn't matter if we choose 1, 2, or 3 coins because the opponent
is left with a 8, 7, or 6 coins. These are all winning positions if the
opponent plays the best moves. Therefore 
\textbf{9 coins is a losing position}. If there are 10 coins then 
choosing 1 coin gives our opponent a losing position, so 
\textbf{10 coins is a winning position}. If there are 11 coins then 
choosing 2 coins gives our opponent 9 coins (a losing position), so 
\textbf{11 coins is a winning position}. If there are 12 coins then 
choosing 3 coins gives our opponent 5 coins (a losing position), so 
\textbf{12 coins is a winning position}. If there are 13 coins then it
doesn't matter if we choose 1, 2, or 3 coins because the opponent
is left with a 12, 11, or 10 coins. These are all winning positions if 
the opponent plays the best moves. Therefore 
\textbf{13 coins is a losing position}. 
 
 \begin{center}
\begin{tabular}{cc}
coins left & type of position\\
\hline
1 & losing\\
2 & winning\\
3 & winning\\
4 & winning\\
5 & losing\\
6 & winning\\
7 & winning\\
8 & winning\\
9 & losing\\
10 & winning\\
11 & winning\\
12 & winning\\
13 & losing
\end{tabular}
\end{center}

So a 13 coin game is a losing position. The second player will 
always win if they play it correctly. Building the chart helps the
class to see the pattern to the losing positions: $1, 5, 9, 13$.
They all have a remainder of 1 after they are divided by 4. So if
the game has 37 coins then $37=9(4)+1$ so 37 coins is a losing 
position for the player who moves first. If there are 1256 coins
then $1256=4(314)$. Since the remainder upon division by 4
is 0, a game with 1256 coins is a winning position for the first 
player.

Knowing the formula for losing squares gives you the winning 
strategy: make sure you leave your opponent a number of coins
which, when divided by 4, has a remainder of 1. If you have 36
coins (remainder 0) then choosing 3 coins leaves your opponent 
on a losing number position because 33 has a remainder
of 1 after division by four: $33=(8)(4)+1$. After your opponent 
is on a losing position, make sure that they stay there. If they 
choose, $k$ coins, you choose $4-k$ coins. Game over!

Thinking like a mathematician takes what seems to be a 
complicated game and makes it easy.
\end{document}