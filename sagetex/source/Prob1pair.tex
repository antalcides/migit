\documentclass[12pt]{article}%
\usepackage{latexsym,amsmath,amsfonts,amssymb}
%\usepackage{concrete}% text font
%\usepackage{euler}% math font
\usepackage{graphicx}%insert pictures
\usepackage{geometry}%setting margins
\usepackage{fancybox}
\usepackage{xcolor}
\usepackage[colorinlistoftodos,shadow]{todonotes}
\usepackage{tikz}
\pagestyle{empty}
\linespread{1.6}%double spacing created to give space for binomial coefficients
%%%%%%%%%%%%%%%%%%%%%%%%%%%%%%%%% NOTATION %%%%%%%%%%%%%%%%%%%%%%%%
\newcommand{\FPC}{Fundamental Principle of Counting }
%%%%%%%%%%%%%%%%%%%%%% END OF PREAMBLE %%%%%%%%%%%%%%%%%%%%%%%%%%%%%%
\begin{document}
\noindent Suppose $5$ cards are dealt from a standard deck of $52$ cards. Find
$P(1 \mbox{ pair})$.\\\\
\textbf{Solution:} Since order is not important and repetition of 
cards is not allowed, this is a problem that will involve combinations and
the denominator is $\binom{52}{5}$. Now use the \FPC to determine the numerator. 

\begin{center}
\begin{tabular}{lcc}
Event & Outcomes & Example\\
\hline
$E_1:$ Choose $1$ kind to be the pair & $\binom{13}{1}$ & $5$ \\
$E_2:$ Choose $2$ specific cards for the pair & $\binom{4}{2}$ & 
$5\,\heartsuit$, $5\,\spadesuit$ \\
$E_3:$ Choose $3$ other kinds & $\binom{12}{3}$ & 
$5\,\heartsuit$, $5\,\spadesuit$, $7$, $8$, J \\
$E_4:$ Choose a specific first card & $\binom{4}{1}$ & 
$5\,\heartsuit$, $5\,\spadesuit$, $7\,\heartsuit$, $8$, J \\
$E_5:$ Choose a specific second card & $\binom{4}{1}$ & 
$5\,\heartsuit$, $5\,\spadesuit$, $7\,\heartsuit$, $8\,\clubsuit$, J \\
$E_6:$ Choose a specific third card & $\binom{4}{1}$ & 
$5\,\heartsuit$, $5\,\spadesuit$, $7\,\heartsuit$, $8\,\clubsuit$, J\,$\clubsuit$ \\
\end{tabular}
\end{center}
By the \FPC there $\binom{13}{1}\binom{4}{2}\binom{12}{3}\binom{4}{1}\binom{4}{1}\binom{4}{1}$ different ways to choose $1$ pair. Therefore, $P(1 \mbox{ pair})=\frac{\binom{13}{1}\binom{4}{2}\binom{12}{3}\binom{4}{1}\binom{4}{1}\binom{4}{1}}{\binom{52}{5}}=\frac{1098240}{2598960} \approx 0.4225690276$.\\\\
\end{document}
