\documentclass[12pt]{article}
\usepackage{latexsym, amsmath,amsfonts,amssymb}
\usepackage{xcolor}
\usepackage[margin=1in]{geometry}
\usepackage{kpfonts}  %Changing the default fonts
\usepackage[T1]{fontenc}
\usepackage{tkz-euclide}
\usetkzobj{all} % all the objects
\setlength{\parskip}{1.2ex} %space between paragraphs
\setlength{\parindent}{1em} %Paragraph indentation
\usetikzlibrary{calc,trees,positioning,arrows,chains,shapes.geometric,%
  decorations.pathreplacing,decorations.pathmorphing,shapes,%
  matrix,shapes.symbols,plotmarks,decorations.markings,shadows}
\linespread{1.25} %spacing between lines
\pagestyle{empty}
\begin{document}
\begin{center}
\Large{\textbf{The Converse of the Pythagorean Theorem}}
\end{center}
\textbf{The Converse of the Pythagorean Theorem:} Let $ABC$ be a triangle with sides of length $a$, $b$, and $c$ (the longest). If $c^2=a^2+b^2$ then 
$ABC$ is a right triangle.
\begin{center}
\begin{tikzpicture}[scale=2]
\tkzInit[xmin=-1.2,xmax=2.7,ymin=-1.2,ymax=2.5]
\tkzDefPoint(0,0){A}
\tkzDefPoint(1.5,2){B}
\tkzDefPoint(2.5,0){C}
\tkzDrawSegment(A,B)
\tkzDrawSegment(B,C)
\tkzDrawSegment(A,C)
\tkzLabelPoints[above](B)
\tkzLabelPoints[left](A)
\tkzLabelPoints[below](C)
\tkzDefLine[orthogonal=through C](B,C) \tkzGetPoint{D}
\tkzLabelPoints[right](D)
\tkzDrawPolySeg(D,C)
\tkzDrawPolySeg(D,B)
\tkzMarkRightAngle(B,C,D)
\tkzLabelSegment[below](A,C){$b$}
\tkzLabelSegment[above left](A,B){$c$}
\tkzLabelSegment[below left](B,C){$a$}
\tkzLabelSegment[below](C,D){$b$}
\end{tikzpicture}
\end{center}

\noindent\textbf{Proof:} Draw a line segment $\overline{CD}$ which is 
perpendicular to $\overline{BC}$ and has length $b$ and then draw
$\overline{BD}$. Now $\overline{AC} \cong \overline{CD}$ and, by the
Reflexive Property, $\overline{BC} \cong \overline{BC}$. Applying the
Pythagorean Theorem to triangle $BCD$ yields $BD^2=BC^2+CD^2$.
Since $BC=a$ and $CD=b$ this is equivalent to $BD^2=a^2+b^2$.
By hypothesis, in triangle $ABC$ we have $c^2=a^2+b^2$, hence
$BD^2=c^2$. Since length is never negative, this means $BD=c$. 
Therefore, triangles $ACB$ and $DCB$ are congruent by $SSS$; it follows
that triangle $ABC$ is a right triangle.
\end{document}