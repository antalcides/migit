\documentclass[12pt]{article}
\usepackage[margin=1in]{geometry} % set page margins
\linespread{1.35}%spacing between lines
\begin{document}
\begin{center}
  \Large{\textbf{The Zero Property}}
  \end{center}
  \textbf{Zero Property:} If $ab=0$ then $a=0$ or $b=0$.\\\\
Suppose $ab=0$. There are $3$ cases to look at. 
\begin{description}
\item[Case 1:] Both $a=0$ and $b=0$. In this case the conclusion
is obviously true.
\item[Case 2:] If $a \neq 0$ then $\frac{1}{a}$ is defined. 
Therefore $\left(\frac{1}{a}\right)ab=\left(\frac{1}{a}\right)0$
and this simplifies to $b=0$.
\item[Case3:] If $b \neq 0$ then $\frac{1}{b}$ is defined.
Therefore $ab\left(\frac{1}{b}\right)=0\left(\frac{1}{b}\right)$
and this simplifies to $a=0$. 
\end{description}
In all $3$ cases either $a=0$ or $b=0$.
\end{document}