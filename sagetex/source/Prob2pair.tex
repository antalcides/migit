\documentclass[12pt]{article}%
\usepackage{latexsym,amsmath,amsfonts,amssymb}
%\usepackage{concrete}% text font
%\usepackage{euler}% math font
\usepackage{graphicx}%insert pictures
\usepackage{geometry}%setting margins
\usepackage{fancybox}
\usepackage{xcolor}
\usepackage[colorinlistoftodos,shadow]{todonotes}
\usepackage{tikz}
\pagestyle{empty}
\linespread{1.6}% double spacing to give room for the binomial coefficients
%%%%%%%%%%%%%%%%%%%%%%%%%%%%%%%%% NOTATION %%%%%%%%%%%%%%%%%%%%%%%%
\newcommand{\FPC}{Fundamental Principle of Counting }
%%%%%%%%%%%%%%%%%%%%%% END OF PREAMBLE %%%%%%%%%%%%%%%%%%%%%%%%%%%%%%
\begin{document}
\noindent Suppose $5$ cards are dealt from a standard deck of $52$ cards. Find
$P(2 \mbox{ pairs})$.\\\\
\textbf{Solution:} Since order is not important and repetition of 
cards is not allowed, this is a problem that will involve combinations and
the denominator is $\binom{52}{5}$. To determine the numerator start by noting
that $2$ pair does not refer to the same kind; that is, all $4$ twos would be
four of a kind and not $2$ pairs. With this distinction made we proceed by
using the \FPC to find the numerator. 

\begin{center}
\begin{tabular}{lcc}
Event & Outcomes & Example\\
\hline
$E_1:$ Choose $2$ different kinds & $\binom{13}{2}$ & $2$, K\\
$E_2:$ Choose $2$ specific cards for first pair & $\binom{4}{2}$ & 
$2\,\clubsuit$, $2\,\diamondsuit$, K\\
$E_3:$ Choose $2$ specific cards for second pair & $\binom{4}{2}$ & 
$2\,\clubsuit$, $2\,\diamondsuit$, K\,$\spadesuit$, K\,$\clubsuit$ \\
$E_4:$ Choose the final card & $\binom{44}{1}$ & 
$2\,\clubsuit$, $2\;\diamondsuit$, K\,$\spadesuit$, K\,$\clubsuit$, $5\,\heartsuit$ \\
\end{tabular}
\end{center}
By the \FPC there $\binom{13}{2}\binom{4}{2}\binom{4}{2}\binom{44}{1}$ different
ways to choose $2$ pairs. Therefore, $P(2 \mbox{ pairs})=\frac{\binom{13}{2}\binom{4}{2}\binom{4}{2}\binom{44}{1}}{\binom{52}{5}}=\frac{123552}{2598960} \approx .0475390156$.\\\\
\textbf{Note:} For $E_4$, the final card can't be the same kind as the two that
have already been chosen because that would create three of a kind. Therefore, 
from the original $52$ cards we must discard all $8$ cards of the kinds that form 
the two pairs. 
\end{document}
