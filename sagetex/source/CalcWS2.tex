
\documentclass[12pt]{article}%
\usepackage{graphicx}
\usepackage[a4paper,margin=.5in]{geometry}
\usepackage{kpfonts}  %Changing the default fonts
\usepackage[T1]{fontenc}
\usepackage{epsfig,latexsym, amsfonts,amssymb,color}
\def\mystrut(#1,#2){\vrule height #1pt depth #2pt width 0pt}
\pagestyle{empty}

\begin{document}
\noindent Name \rule{80 mm}{.2pt} \hspace{35 mm} Date \rule{25 mm}{.2pt}\\\\
Class \rule{30 mm}{.2pt} \hspace{10 mm}\textbf{More Problems with Calculators}\\

In our first calculator worksheet we saw technology doesn't always give the correct answers.
Overflow errors happened when a really small number was made equal to one. Overflow errors
happened when numbers were too big to represented by the calculator. In general, the 
calculator gave you the best answer it could from the numbers it could represent. In this 
worksheet we'll see that sometimes calculators give answers which are just plain wrong. We'll
also see that you need to know how to enter problems properly.
\begin{enumerate}
\item Calculators have lots of problems with exponentiation. Fill in the values in the table below.
 \begin{table}[htbp]
 \centering
 \begin{tabular}{|c|c|c|c|c|c|c|}
\hline
 & \mystrut(14,7)  Your calculator & computer calculator & Google & Cool Math\\
\hline
\mystrut(14,7) $0^0$ & &  & &\\
\hline
\mystrut(14,7)  $(-1)^0$  & &  & &\\
\hline
\mystrut(17,7) $(-8)^{-\frac{1}{3}}$  & &  & &\\
\hline
\mystrut(14,7)  $(-1)^{-1}$  & &  & &\\
\hline
\mystrut(17,7) $(-1)^{\frac{6}{10}}$  & &  & &\\
\hline
\mystrut(21,10) $\left(-\frac{1}{3}\right)^{-\frac{1}{3}}$  & &  & &\\
\hline
\end{tabular}
\end{table}
\item Notice that technology is, in some cases, giving us different answers for the 
same problem. Now it's your turn: What do you think are the correct answers to:\\\\
$0^0$\\\\
$(-1)^0$ \\\\
$(-8)^{-\frac{1}{3}}$\\\\
$(-1)^{-1}$\\\\
$(-1)^{\frac{6}{10}}$ \\\\
$\left(-\frac{1}{3}\right)^{-\frac{1}{3}}$ \\
\begin{center}\underline{Make sure you've shown your work.}\end{center}
\item 
If you're going to use technology then you must know the orders of operation. These can
be remembered using the mnemonic device PEMDAS (Please Excuse My Dear Aunt Sally) where:\\
\textcolor{red}{P}: Parentheses, braces, brackets (from ``inside'' to ``outside'')\\
\textcolor{red}{E}: Exponentiation\\
\textcolor{red}{M}: Multiplication (from left to right)\\
\textcolor{red}{D}: Division (from left to right)\\
\textcolor{red}{A}: Addition (from left to right)\\
\textcolor{red}{S}: Subtraction (from left to right)\\

Multiplication and division rank equally. These should be carried out before additions and
subtractions, which also rank equally. The insistence of from left to right is very important!
It tells us that $3 \div 6 \div 3$ must be evaluated as $\frac{3}{6}\div 3=\frac{3}{3(6)}=
\frac{1}{6}$ and NOT as $3 \div \frac{6}{3}=3\div 2= \frac{3}{2}$. 

If you need to calculate $\sqrt[3]{-2}$ on your calculator then of course, you'll need to 
know that it is equivalent to $(-2)^{\frac{1}{3}}$ and then be able to  enter that into
the calculator.

\item Use your calculator to determine the value of:
\begin{enumerate}
\item \verb!(-2)^1/3!  \hspace{40 mm}Answer: \rule{50 mm}{.2pt}
\item \verb!(-2)^(1/3)! \hspace{36 mm}Answer: \rule{50 mm}{.2pt}
\end{enumerate}
Which calculation is equivalent to $(-2)^{\frac{1}{3}}$?\\

\underline{Make sure you explain the reason behind your answer.}
\item Notice the orders of operation don't say Exponentiation (from left to right). That's
because this assumption is not standard. Whenever I have seen something like $3^{3^3}$ in
print it has NOT been interpreted from left to right. Rather, it is interpreted 
as $3^{(3^3)}$. Wikipedia reports that the TI-92 and TI-30XII give different results for this!
In general, people reading math are expected to follow the order 
of operations listed and use parentheses carefully
to clarify the meaning of complicated expressions. The expression \verb!2^2!! could be
interpreted as $2^{2!}$ or $2^{2}!$. When there is no standard you
should use parentheses to avoid potential problems.\\

Evaluate $2^{2!}$ \hspace{40 mm}Answer: \rule{50 mm}{.2pt}\\\\
Evaluate $2^{2}!$ \hspace{40 mm}Answer: \rule{50 mm}{.2pt}
\end{enumerate}

After completing this worksheet you should be aware that 
\begin{enumerate}
\item Calculators don't always give the exact answer. Most of the time they give an
approximation.
\item Calculators can make mistakes.
\item A good foundation in mathematics is essential to use technology properly.
\item Only the basic orders of operation listed here are standard. 
\item It is a good habit to use parentheses generously so the reader know what calculation
you intend.
\end{enumerate}

The field of mathematics that studies the methods of computing mathematical answers
accurately is called \textbf{numerical analysis.}
\end{document}
