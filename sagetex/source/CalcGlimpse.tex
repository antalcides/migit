\documentclass[12pt]{article}
\usepackage{graphicx}
\usepackage{amsmath,amsfonts,amssymb}
\usepackage{xcolor}
\usepackage[margin=.75in]{geometry}
\linespread{1.35}
\pagestyle{empty}
\begin{document}
\begin{center}
{\Large \textbf{A Glimpse of Calculus}}\\
\end{center}
\textbf{Requirement:} A solid background in precalculus and a 
graphing calculator.\\
\textbf{Background:} Calculus had so much material to cover that
I used this example in precalculus to give students a glimpse of
calculus through using a graphing calculator. You'll need to mention
how many cubic inches are in a gallon.\\

A manufacturer of paint packages the product in 1 gallon cylindrical 
cans. In an effort to save money in production (which will also 
increase profits) they want to know if there is a particular size paint 
can that will hold $1$ gallon of paint and use less metal than the 
other paint cans. What is the minimum amount of metal in the can?

We can construct a \textit{mathematical model} to solve this 
problem. The first step is to find the {\bf primary equation}; that is, 
a formula for the quantity to be minimized/maximized. Since we're 
trying to minimize the surface area, $S$, the primary equation is:
\[S=2\pi r^2+2\pi rh \]

The formula for surface area has two unknowns: the radius and the 
height. One of the variables can be eliminated by using a 
{\bf secondary equation}. Since the volume of the paint can is 
given by $V=\pi r^2h$ and we know the volume is $1$ gallon, the
equation becomes $1 \mbox{ gallon}=\pi r^2h$. But the radius
and height of the can aren't measured in units of gallons; use the 
fact that one gallon which is approximately $231$ cubic inches and 
solve for $h$ to get $h=\frac{231}{\pi r^2}\mbox{ in}^3$. 
Combine the primary equation with the secondary equation 
to get $S=2\pi r^2+2\pi r\left(\frac{231}{\pi r^2}\right)$, 
which simplifies to $S=2\pi r^2+\left(\frac{462}{r}\right)$. 
This function of $r$, whose units are inches, and this can be 
graphed in the 
Cartesian plane. Using a graphing calculator you can, with a little 
bit of work, get a graph similar to the one below.

\begin{figure}[h]
\begin{center}	
  \includegraphics[width=3.in]{PaintFunc.pdf}
  \end{center}
\caption{What is the minimum amount of material needed to make a 1 gallon paint can?} \label{fg:ptcn}
\end{figure}

We know from working with calculators that what we don't see in 
the graph might be more important than what we do 
see in the graph. There might be many other value way out that 
give us another value which is better. From our picture 
on the screen, however, by pointing the arrow on the screen to the 
lowest point on the graph we get an {\em approximate} point of 
about $(3.325, 208.42)$. By combining our math knowledge with 
technology we can quickly get an approximate answer to our 
problem: a paint can with radius of $3.325$ inches will have a 
surface area of $208.42$ square inches.
Although this appears to be a good estimate, there is an important 
question to consider:
Is it possible that there is a value of $r$ not on the graph that 
gives us an even smaller area? That isn't obvious.
Likewise, what's the exact value of $r$ that gives the absolute 
minimum value for a surface area.
Calculus will allow us to find all the possible minimums and it will 
allow us to find the exact value of the radius which produces the 
can of minimum volume. For this mathematical
model the value of $r$ that will require the least amount of 
material is $r=\sqrt[3]{\frac{231}{2\pi}} \approx 3.325$
inches. The minimum amount of material is then $S=2\pi \left( \sqrt[3]{\frac{231}{2\pi}}\, \right)^2+\left(\frac{462}{\sqrt
[3]{\frac{231}{2\pi}}}\right) \approx 208.41 \mbox{in}^2$.\\

Many important problems revolve around trying to find a minimum 
(e.g. surface area) or a maximum
(e.g. profit). As a result, an important part of calculus is to:
\begin{enumerate}
\item Find equations that describe the problem.
\item Combine the equations to get a primary equation in one 
variable.
\item Graph the equation.
\item Find the value of $x$ that corresponds to the minimum/
maximum value of the function.
\item Put the value into a number that the average person would 
understand and add the units.
\end{enumerate}
\textbf{Conclusion:} 
Without calculus, people you can find the surface area of a paint 
can with a specific radius. Calculus will give you the power to
determine which paint can, of all the infinite paint cans with 
a volume of $1$ gallon, will have the smallest surface area. 
Technology allows us to graph a function which describes the 
surface area and estimate the minimum possible surface
area but calculus gives us the power to find the exact radius that 
results in the smallest surface area. 
\end{document}
