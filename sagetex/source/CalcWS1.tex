
\documentclass[12pt]{article}%
\usepackage{graphicx,hyperref,amsmath,natbib,bm,url,microtype,todonotes}
%\usepackage[a4paper,text={16.5cm,25.2cm},centering]{geometry}
\usepackage[a4paper,margin=.75in]{geometry}
\usepackage[compact,small]{titlesec}
\setlength{\parskip}{1.2ex} %space between paragraphs
\setlength{\parindent}{1em} %Paragraph indentation
\clubpenalty = 10000
\widowpenalty = 10000
\newcommand\T{\rule{0pt}{3ex}} % \T will create extra space above (used to fix tables)
\newcommand\B{\rule[-1.5ex]{0pt}{0pt}}% \B will create extra space below (used to fix tables)
\usepackage{kpfonts}  %Changing the default fonts
\usepackage[T1]{fontenc}
\usepackage{epsfig,latexsym, amsfonts,amssymb,color}
\pagestyle{empty}

\begin{document}
\noindent Name \rule{80 mm}{.2pt} \hspace{35 mm} Date \rule{25 mm}{.2pt}\\\\
Class \rule{30 mm}{.2pt} \hspace{10 mm}\textbf{Introduction to Calculators and Computers}\\

Throughout high school and college you'll find calculators and computers are an important
tool of math, physics, biology, chemistry, and computer science. When you're finished with 
school you'll find many jobs require you to work with calculators, computers, and/or
spreadsheets. Working with technology is just a fact of modern life. But there's a ``dirty
little secret'' to technology: it doesn't always work properly. This worksheet will look
at some of the problems with technology. You will compare the calculator we use in class 
with the calculator on our school computer, the calculator in Google search, and the 
``Cool Math'' calculator at http://www.coolmath.com/graphit/

When you're working with technology in mathematics, you should be aware that although 
$+$ and $-$ represent addition and subtraction, either
$\times$ or $\ast$ is used to represent multiplication and $\div$ or $\slash$ is used to
represent division. Finally, most math technology uses either \verb!^! or $\ast \ast$ 
for exponentiation.

\begin{enumerate}
 \item Determine the symbols used for multiplication, division, and exponentiation.
 \begin{table}[htbp]
 \centering
 \begin{tabular}{|c|c|c|c|c|c|c|}
\hline
 & \T \B Your calculator & computer calculator & Google & Cool Math\\
\hline
\T \B Mult. & &  & &\\
\hline
\T \B Div. & &  & &\\
\hline
\T \B Exp. & & &  Both &\\
\hline
\end{tabular}
\end{table}

\item Most math technology has a maximum number of digits that 
will be displayed. For a handheld calculator we can easily see the 
the number of digits. For the Google search calculator the answer 
varies. If I try typing in $1.123456789101112 \ast 1$ I'll get the 
answer $1.2345679101$. Therefore, I'll record that Google 
shows us a maximum of 12 digits.  Complete the rest of the table:
 \begin{table}[htbp]
 \centering
 \begin{tabular}{|c|c|c|c|c|c|c|}
\hline
 & \T \B Your calculator & computer calculator & Google & Cool Math\\
\hline
\T \B Max digits  & & & 10 &\\
\hline
\end{tabular}
\end{table}
\item The maximum number of digits poses a problem: What happens if the answer is a little
bit bigger? Since Google gives 12 digits, try adding 
$100000000000+.1$. Although you have 
no trouble getting the correct answer $100000000000.1$, Google 
will give $100000000000$. Confirm that if you add try and make 
the number a little bit bigger than the maximum number of digits 
that your calculator has then your calculator will give you the 
exact answer either.
\item In general, if the numbers have more than the maximum number of digits displayed then
you get an approximate answer. However, the Google results are harder to make sense of.
Type the following into Google and report the answers below:
\begin{enumerate}
\item .123456789101+.000000000001  \hspace{20 mm}Answer: \rule{50 mm}{.2pt}
\item .123456789101+.0000000000009  \hspace{20 mm}Answer: \rule{50 mm}{.2pt}
\item 1234567891011 + 1 \hspace{45 mm}Answer: \rule{50 mm}{.2pt}
\item 123456789101112+1 \hspace{43 mm}Answer: \rule{50 mm}{.2pt}
\end{enumerate}
Circle the answer(s) which are correct, as opposed to approximately correct.
\item The fact that there is a maximum number of digits in any number the calculator 
displays means there will always be a smallest possible nonzero number. That number will 
be in the form of $1 \times 10^{-k}$, for some power of $k$ and any number 
smaller than it will be considered equal to $0$. Find the largest value of
$k$ for which $1 \times 10^{-k}>0$. 
 \begin{table}[htbp]
 \centering
 \begin{tabular}{|c|c|c|c|c|c|c|}
\hline
 & \T \B Your calculator & computer calculator & Google & Cool Math\\
\hline
\T \B $k$  & & &  &\\
\hline
\end{tabular}
\end{table}

An \textbf{underflow over error} is when the answer is very small (but nonzero) and 
the calculator thinks the answer is equal to $0$.
\item Likewise, every calculator will have a biggest positive number that it can represent.
Any number bigger than that will cause the calculator to issue a warning. Find the largest
value of $k$ for which $10^k$ is still a number the calculator can represent.
 \begin{table}[htbp]
 \centering
 \begin{tabular}{|c|c|c|c|c|c|c|}
\hline
 & \T \B Your calculator & computer calculator & Google & Cool Math\\
\hline
\T \B $k$  & & &  &\\
\hline
\end{tabular}
\end{table}

In a similar manner there is a smallest negative number so that $-1 \times 10^{k}$ is
still a number the calculator accepts. An \textbf{overflow error} occurs when the value 
of $k$ is so large that the calculator issues an error. Essentially the calculator 
concludes the answer is infinitely positive (or negative).
\end{enumerate}
Since there is a smallest positive number, a largest positive number, and a smallest negative
number it follows that a calculator can only represent a finite number of numbers.
After completing this worksheet you should be aware that 
\begin{enumerate}
\item Calculators don't always give the exact answer. Many times they give the
approximate answer.
\item When the calculator says a very small nonzero number is actually zero then an
underflow error has occurred.
\item If a calculator gives an error because a number is too big to calculate then an
overflow error occurred.
\item A calculator can only properly represent a finite number of numbers.
\end{enumerate}
All calculators and computers suffer from overflow and 
underflow errors, so you still need to be able to do the 
verify that your calculator is correct. The field of mathematics 
that studies how to calculate accurately with computers/calculators is called \textbf{numerical analysis}.
\end{document}
