\documentclass[]{article}
\usepackage[T1]{fontenc}
\usepackage{lmodern}
\usepackage{amssymb,amsmath}
\usepackage{ifxetex,ifluatex}
\usepackage{fixltx2e} % provides \textsubscript
% use upquote if available, for straight quotes in verbatim environments
\IfFileExists{upquote.sty}{\usepackage{upquote}}{}
\ifnum 0\ifxetex 1\fi\ifluatex 1\fi=0 % if pdftex
  \usepackage[utf8]{inputenc}
\else % if luatex or xelatex
  \ifxetex
    \usepackage{mathspec}
    \usepackage{xltxtra,xunicode}
  \else
    \usepackage{fontspec}
  \fi
  \defaultfontfeatures{Mapping=tex-text,Scale=MatchLowercase}
  \newcommand{\euro}{€}
\fi
% use microtype if available
\IfFileExists{microtype.sty}{\usepackage{microtype}}{}
\usepackage{longtable,booktabs}
\usepackage{graphicx}
% Redefine \includegraphics so that, unless explicit options are
% given, the image width will not exceed the width of the page.
% Images get their normal width if they fit onto the page, but
% are scaled down if they would overflow the margins.
\makeatletter
\def\ScaleIfNeeded{%
  \ifdim\Gin@nat@width>\linewidth
    \linewidth
  \else
    \Gin@nat@width
  \fi
}
\makeatother
\let\Oldincludegraphics\includegraphics
{%
 \catcode`\@=11\relax%
 \gdef\includegraphics{\@ifnextchar[{\Oldincludegraphics}{\Oldincludegraphics[width=\ScaleIfNeeded]}}%
}%
\ifxetex
  \usepackage[setpagesize=false, % page size defined by xetex
              unicode=false, % unicode breaks when used with xetex
              xetex]{hyperref}
\else
  \usepackage[unicode=true]{hyperref}
\fi
\hypersetup{breaklinks=true,
            bookmarks=true,
            pdfauthor={},
            pdftitle={},
            colorlinks=true,
            citecolor=blue,
            urlcolor=blue,
            linkcolor=magenta,
            pdfborder={0 0 0}}
\urlstyle{same}  % don't use monospace font for urls
\setlength{\parindent}{0pt}
\setlength{\parskip}{6pt plus 2pt minus 1pt}
\setlength{\emergencystretch}{3em}  % prevent overfull lines
\setcounter{secnumdepth}{0}

\author{}
\date{}

\begin{document}

{
\hypersetup{linkcolor=black}
\setcounter{tocdepth}{3}
\tableofcontents
}
class: center, middle

\section{Example presentation}\label{example-presentation}

by \href{http://www.fladd.de}{Florian Krause}

\begin{center}\rule{3in}{0.4pt}\end{center}

\section{Using lists}\label{using-lists}

Presentations usually use lists items:

\begin{itemize}
\item
  Bullet item 1
\item
  Bullet item 2
\end{itemize}

\begin{longtable}[c]{@{}l@{}}
\toprule\addlinespace
\begin{minipage}[t]{0.03\columnwidth}\raggedright
count: false
\end{minipage}
\\\addlinespace
\begin{minipage}[t]{0.03\columnwidth}\raggedright
* Bullet item 3 (incremental)
\end{minipage}
\\\addlinespace
\begin{minipage}[t]{0.03\columnwidth}\raggedright
1. Numbered item 1
\end{minipage}
\\\addlinespace
\begin{minipage}[t]{0.03\columnwidth}\raggedright
2. Numbered item 2
\end{minipage}
\\\addlinespace
\bottomrule
\end{longtable}

\section{Embedding images}\label{embedding-images}

\begin{itemize}
\itemsep1pt\parskip0pt\parsep0pt
\item
  Images can be included by simply pointing to the file:
\end{itemize}

.center{[}\includegraphics{image.png}{]}

\begin{center}\rule{3in}{0.4pt}\end{center}

\section{Formatting text}\label{formatting-text}

\begin{itemize}
\item
  Text can be \emph{italic} or \textbf{bold}
\item
  There can be \texttt{inline code}
\item
  And even full (syntax highlighted!) code blocks:
\end{itemize}

```python def test(a, b): c = a + b return ``a + b = \{0\}.''format(c)
````

???

\begin{itemize}
\itemsep1pt\parskip0pt\parsep0pt
\item
  This is a comment and not visible on the slides
\end{itemize}

\end{document}
