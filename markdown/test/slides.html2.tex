\documentclass[]{article}
\usepackage[T1]{fontenc}
\usepackage{lmodern}
\usepackage{amssymb,amsmath}
\usepackage{ifxetex,ifluatex}
\usepackage{fixltx2e} % provides \textsubscript
% use upquote if available, for straight quotes in verbatim environments
\IfFileExists{upquote.sty}{\usepackage{upquote}}{}
\ifnum 0\ifxetex 1\fi\ifluatex 1\fi=0 % if pdftex
  \usepackage[utf8]{inputenc}
\else % if luatex or xelatex
  \ifxetex
    \usepackage{mathspec}
    \usepackage{xltxtra,xunicode}
  \else
    \usepackage{fontspec}
  \fi
  \defaultfontfeatures{Mapping=tex-text,Scale=MatchLowercase}
  \newcommand{\euro}{€}
\fi
% use microtype if available
\IfFileExists{microtype.sty}{\usepackage{microtype}}{}
\usepackage{graphicx}
% Redefine \includegraphics so that, unless explicit options are
% given, the image width will not exceed the width of the page.
% Images get their normal width if they fit onto the page, but
% are scaled down if they would overflow the margins.
\makeatletter
\def\ScaleIfNeeded{%
  \ifdim\Gin@nat@width>\linewidth
    \linewidth
  \else
    \Gin@nat@width
  \fi
}
\makeatother
\let\Oldincludegraphics\includegraphics
{%
 \catcode`\@=11\relax%
 \gdef\includegraphics{\@ifnextchar[{\Oldincludegraphics}{\Oldincludegraphics[width=\ScaleIfNeeded]}}%
}%
\ifxetex
  \usepackage[setpagesize=false, % page size defined by xetex
              unicode=false, % unicode breaks when used with xetex
              xetex]{hyperref}
\else
  \usepackage[unicode=true]{hyperref}
\fi
\hypersetup{breaklinks=true,
            bookmarks=true,
            pdfauthor={John Doe},
            pdftitle={Habits},
            colorlinks=true,
            citecolor=blue,
            urlcolor=blue,
            linkcolor=magenta,
            pdfborder={0 0 0}}
\urlstyle{same}  % don't use monospace font for urls
\setlength{\parindent}{0pt}
\setlength{\parskip}{6pt plus 2pt minus 1pt}
\setlength{\emergencystretch}{3em}  % prevent overfull lines
\setcounter{secnumdepth}{5}

\title{Habits}
\author{John Doe}
\date{March 22, 2005}

\begin{document}
\maketitle

\section{In the morning}\label{in-the-morning}

\subsection{Getting up}\label{getting-up}

\begin{itemize}
\itemsep1pt\parskip0pt\parsep0pt
\item
  Turn off alarm
\item
  Get out of bed
\end{itemize}

\subsection{Breakfast}\label{breakfast}

\begin{itemize}
\itemsep1pt\parskip0pt\parsep0pt
\item
  Eat eggs
\item
  Drink coffee
\end{itemize}

\section{In the evening}\label{in-the-evening}

\subsection{Dinner}\label{dinner}

\begin{itemize}
\itemsep1pt\parskip0pt\parsep0pt
\item
  Eat spaghetti
\item
  Drink wine
\end{itemize}

\begin{center}\rule{3in}{0.4pt}\end{center}

\begin{figure}[htbp]
\centering
\includegraphics{image.png}
\caption{picture of spaghetti}
\end{figure}

\subsection{Going to sleep}\label{going-to-sleep}

\begin{itemize}
\itemsep1pt\parskip0pt\parsep0pt
\item
  Get in bed
\item
  Count sheep To produce the slide show, simply type
\end{itemize}

pandoc -t s5 -s habits.txt -o habits.html for S5,

pandoc -t slidy -s habits.txt -o habits.html for Slidy,

pandoc -t slideous -s habits.txt -o habits.html for Slideous,

pandoc -t dzslides -s habits.txt -o habits.html for DZSlides, or

\end{document}
