%% This document created by Scientific Word (R) Version 3.0

\documentclass[12pt]{article}
\usepackage{amsmath}
\usepackage{graphicx}
\usepackage[spanish]{babel}
\usepackage[pdftex,dvips]{color}
\usepackage{amsmath}
\usepackage[spanish]{babel}
\usepackage{fancybox}
\usepackage[ansinew]{inputenc}
\usepackage{amsfonts}
\usepackage{amssymb}
%TCIDATA{OutputFilter=latex2.dll}
%TCIDATA{CSTFile=article.cst}
%TCIDATA{Created=Friday, October 24, 2003 07:58:30}
%TCIDATA{LastRevised=Mon Oct 27 15:15:10 2003}
%TCIDATA{<META NAME="GraphicsSave" CONTENT="32">}
%TCIDATA{<META NAME="DocumentShell" CONTENT="Standard LaTeX\Blank - Standard LaTeX Article">}
\setcounter{MaxMatrixCols}{30}
\newtheorem{theorem}{Theorem}
\newtheorem{acknowledgement}[theorem]{Acknowledgement}
\newtheorem{algorithm}[theorem]{Algorithm}
\newtheorem{axiom}[theorem]{Axiom}
\newtheorem{case}[theorem]{Case}
\newtheorem{claim}[theorem]{Claim}
\newtheorem{conclusion}[theorem]{Conclusion}
\newtheorem{condition}[theorem]{Condition}
\newtheorem{conjecture}[theorem]{Conjecture}
\newtheorem{corollary}[theorem]{Corollary}
\newtheorem{criterion}[theorem]{Criterion}
\newtheorem{definition}[theorem]{Definition}
\newtheorem{example}[theorem]{Example}
\newtheorem{exercise}[theorem]{Exercise}
\newtheorem{lemma}[theorem]{Lemma}
\newtheorem{notation}[theorem]{Notation}
\newtheorem{problem}[theorem]{Problem}
\newtheorem{proposition}[theorem]{Proposition}
\newtheorem{remark}[theorem]{Remark}
\newtheorem{solution}[theorem]{Soluci\'{o}n}
\newtheorem{summary}[theorem]{Summary}
\newenvironment{proof}[1][Proof]{\noindent\textbf{#1.} }{\ \rule{0.5em}{0.5em}}

\begin{document}
\thispagestyle{plain}\ \newpage

\textbf{\qquad\qquad\qquad\qquad\qquad\qquad\qquad\qquad\qquad\qquad
\qquad\qquad\qquad}

\bigskip\thispagestyle{plain} \bigskip

\begin{center}

\textbf{{\large {M\'{E}TODOS NUM\'{E}RICOS}.\newline \newline \newline
ASIGNACI\'{O}N \#2.}}

\bigskip\bigskip

\textbf{PRESETADO POR:}

\bigskip

\textbf{\color{blue}{{ANTALCIDES OLIVO BURGOS\\    \hspace{4pt}%
IRINA RUIZ BAENA}. }}\vspace{3cm}\newline \textbf{PRESENTADO A: }
\textbf{\color{blue}{\\DR  JUAN CARLOS ORTIZ \\ UNIVERSIDAD DE
PUERTO RICO}}

\textbf{\vspace{3cm}}

\bigskip

\bigskip

\bigskip

\textbf{\ UNIVERSIDAD DEL NORTE-\-UNIVERSIDAD}

\textbf{ESPECIALIZACI\'{O}N EN F\'{I}SICA GENERAL}

\textbf{\ DEPARTAMENTO DE MATEM\'{A}TICAS Y F\'{I}SICA.\newline OCTUBRE 27 del 2.003}
\end{center}

\newpage

\section*{\color{blue}{Asignaci�n 2}}

\section{{\protect\large {M\'{E}TODOS DE INTEGRACI\'{O}N NUM\'{E}RICA}}}

\subsection{LA REGLA DEL TRAPECIO}

El m\'{e}todo de Euler aproxima la derivada en $\left[  x_{n},x_{n+1}\right]
$ por una constante, concretamente por su valor en $x_{n}$. \textquestiondown
Por qu\'{e} privilegiar al punto $x_{n}?$ \textquestiondown No ser\'{a} mejor
tomar , por ejemplo, como aproximaci\'{o}n constante de la derivada el
promedio de sus valores en los extremos del intervalo? En ese caso
\[
y(x_{n+1})=y\left(  x_{n}\right)  +h\frac{f\left(  x_{n},y\left(
x_{n}\right)  \right)  f\left(  x_{n+1},y\left(  x_{n+1}\right)  \right)  }%
{2}+R_{n}%
\]
donde $R_{n}$ es el error de truncaci\'{o}n, por lo que se aproxima de la forma.%

\[
y_{n+1}\approx y_{n}+h\frac{f\left(  x_{n},y\left(  x_{n}\right)  \right)
f\left(  x_{n+1},y\left(  x_{n+1}\right)  \right)  }{2}%
\]

Esta aproximaci\'{o}n conduce a la llamada regla del trapecio

la cual \ es convergente de grado dos, es decir%
\[
\left\Vert e_{n}\right\Vert \leq e^{\left(  b-a\right)  L/\left(  1-\frac
{hL}{2}\right)  }\left\Vert e_{0}\right\Vert +\frac{Ch^{2}}{L}\left(
e^{\left(  b-a\right)  L/\left(  1-\frac{hL}{2}\right)  }-1\right)  ,\text{ si
}0\leq n\leq N
\]%
\[
\text{donde }e_{n}=y\left(  x_{n}\right)  -y_{n},\ \frac{hL}{2}<1,\ C=\frac
{5}{12}\max_{x\in\left[  a,b\right]  }\left\Vert y^{\prime\prime\prime}\left(
x\right)  \right\Vert \
\]

a pesar de que la regla del trapecio tiene un orden de convergencia mayor que
el m\'{e}todo de Euler. eso no significa que sea un m\'{e}todo mejor , pues
hay otra diferencia importante entre ambos m\'{e}todos. el m\'{e}todo de Euler
es \textit{expl\'{\i}cito,} el valor de $y_{n+1}$ viene dado expl\'{\i
}citamente en t\'{e}rmino del valor anterior $y_{n}$ y se puede calcular
f\'{a}cilmente mediante la evaluaci\'{o}n de $f$ y una pocas operaciones
aritm\'{e}ticas. por el contrario, la regla del trapecio es un m\'{e}todo
\textit{impl\'{\i}cito}, para calcular $y_{n+1}$ hay que resolver un sistema
de ecuaciones no lineales, lo que en general es computacionalmente costoso. de
lo que se deduce, el orden no lo es todo. Aunque en un m\'{e}todo de orden
alto en principio hay que dar menos pasos, estos pueden ser muy costosos, por
lo que puede ser mejor un m\'{e}todo de menos orden en el que a pesar de dar
m\'{a}s pasos cada uno de ellos lleve menos trabajo computacional.

\subsection{Aplicaciones}

Para aplicar este m\'{e}todo resolveremos dos problemas de f\'{\i}sica
m\'{e}canica anal\'{\i}ticamente y usando Fortran lo resolveremos
computacionalmente y luego calculamos el error.

\begin{enumerate}
\item  un objeto que pesa $48\ lb$ se suelta desde el reposo en la parte
superior de un plano inclinado met\'{a}lico que tiene una inclinaci\'{o}n de
$30�$ respecto a la horizontal. la resistencia del aire es num\'{e}%
ricamente igual a un medio de la velocidad (en pies por segundo), el
coeficiente de rozamiento es $\frac{1}{4}$

\begin{enumerate}
\item \textquestiondown Cu\'{a}l es el tiempo que tarda el objeto para
adquirir una velocidad de $10.2$ $ft/s$ despu\'{e}s de haberse soltado?

\item \textquestiondown Cu\'{a}l es la distancia sobre el plano que ha
recorrido el cuerpo, para alcanzar la velocidad de $10.2$ $ft/s$?%
%TCIMACRO{\FRAME{dhFU}{2.7691in}{1.8273in}{0pt}{\Qcb{Figura1}}{}{fortran.jpg}%
%{\special{ language "Scientific Word";  type "GRAPHIC";
%maintain-aspect-ratio TRUE;  display "USEDEF";  valid_file "F";
%width 2.7691in;  height 1.8273in;  depth 0pt;  original-width 2.7268in;
%original-height 1.7902in;  cropleft "0";  croptop "1";  cropright "1";
%cropbottom "0";  filename 'fortran.jpg';file-properties "XNPEU";}}}%
%BeginExpansion
\begin{center}
\includegraphics[
natheight=1.790200in,
natwidth=2.726800in,
height=1.8273in,
width=2.7691in
]%
{fortran.jpg}%
\\
Figura1
\end{center}
%EndExpansion
\end{enumerate}
\end{enumerate}

\begin{solution}
\begin{enumerate}
\item [a.]La l\'{i}nea de movimiento es a lo largo del plano, si escogemos el
origen del sistema de referencia en la parte superior y el sentido positivo de
las $x$ hacia abajo del plano, entonces las%
%TCIMACRO{\FRAME{dhFU}{2.8643in}{2.1715in}{0pt}{\Qcb{Figura 2}}{}%
%{plano.jpg}{\special{ language "Scientific Word";  type "GRAPHIC";
%maintain-aspect-ratio TRUE;  display "USEDEF";  valid_file "F";
%width 2.8643in;  height 2.1715in;  depth 0pt;  original-width 2.8098in;
%original-height 2.1231in;  cropleft "0";  croptop "1";  cropright "1";
%cropbottom "0";  filename 'plano.jpg';file-properties "XNPEU";}}}%
%BeginExpansion
\begin{center}
\includegraphics[
natheight=2.123100in,
natwidth=2.809800in,
height=2.1715in,
width=2.8643in
]%
{plano.jpg}%
\\
Figura 2
\end{center}
%EndExpansion
fuerzas que act\'{u}na sobre el objeto A son:\newline Su peso de 48 lb que
act\'{u}a verticalmente hacia abajo.\newline La fuerza normal, N que ejerce el
plano sobre el objeto la cual act\'{u}a en la direcci\'{o}n positiva y
perpendicular al plano.\newline La resitencia del aire $f_{a}$,que tiene un
valor num\'{e}rico igual a $\frac{v}{2}$, puesto que $v>0$ esta fuerza tiene
una direcci\'{o}n negativa en $x.$\newline La fuerza de rozamiento, que tiene
un valor $\mu N$ y tiene una direcci\'{o}n negativa en $x.$\newline de la
gr\'{a}fica 2 y aplicando la segunda ley de Newton obtenemos
\begin{align*}
\text{ en }x  &  :\ \ ma\ =Wsen30\U{ba}-f_{s}-f_{r}\\
f_{r}  &  =\mu N\\
\text{en }y  &  :0=N-W\cos30\U{ba}%
\end{align*}
Reemplazando los valores y resolviendo el sistema de las tres ecuaciones
anteriores se obtiene:%
\[
\frac{3}{2}\frac{dv}{dt}=24-6\sqrt{3}-\frac{1}{2}v
\]
al separar variables se obtiene%
\[
\frac{dv}{48-12\sqrt{3}-v}=\frac{dt}{3}%
\]
como cuando $t=0,\ v=0$ se tiene%
\begin{align}
\int_{0}^{10.2}\frac{dv}{48-12\sqrt{3}-v}  &  =\int_{0}^{t}\frac{dt}%
{3}\label{ec1}\\
0.469\,66  &  =\frac{1}{3}t\\
t_{a}  &  =1.\,\allowbreak408\,99\ s
\end{align}
de la ecuaci\'{o}n \ref{ec1} observamos que la velocidad final debe cumplir la
siguiente condici\'{o}n $v<27.\,\allowbreak215$, ya que de no ser as\'{i} el
cuerpo subir\'{i}a en vez de bajar

\item[b.] Al resolver anal\'{\i}ticamente la ecuaci\'{o}n \ref{ec1} y
despejando $t$ obtenemos que
\begin{align*}
v  &  =\left(  48-12\sqrt{3}\right)  \left(  1-e^{-\frac{t}{3}}\right) \\
\frac{dx}{dt}  &  =\left(  48-12\sqrt{3}\right)  \left(  1-e^{-\frac{t}{3}%
}\right)
\end{align*}
de lo que se obtiene%
\[
\int_{0}^{x}dy=\int_{0}^{1.409}\left(  48-12\sqrt{3}\right)  \left(
1-e^{-\frac{t}{3}}\right)  dt
\]
ya que $x\left(  0\right)  =0\ ,~t_{0}=0,$ al integrar podemos determinar lo
que nos piden, as\'{\i}:%
\begin{align*}
x_{a}  &  =\left(  13.\,\allowbreak662-3.\,\allowbreak415\,6\sqrt{3}\right)
ft\\
x_{a}  &  =7.\,\allowbreak746\ ft
\end{align*}
\end{enumerate}
\end{solution}

\begin{solution}

\begin{enumerate}
\item [a.]Podemos resolver el problema \ num\'{e}ricamente utilizando la
ecuaci\'{o}n \ref{ec1} y aplicando la regla del trapecio para calcular la
integral, estableciendo en el algoritmo%
\[%
%TCIMACRO{\dint _{a}^{b}}%
%BeginExpansion
{\displaystyle\int_{a}^{b}}
%EndExpansion
\,\,t(v)\,dv\approx T_{tra}=\dfrac{\Delta v}{2}\left[  f(v_{0})+2f(v_{1}%
)+2f(v_{2})+\cdots+2f(v_{n-1})+f(v_{n})\right]
\]
donde
\[
\Delta v=(b-a)/n\text{\quad y\quad}v_{i}=a+i\,\Delta v
\]%
%TCIMACRO{\QTR{BOXHEAD}{El error m\'{a}ximo del m\'{e}todo se puede
%calcular:}}%
%BeginExpansion
El error m\'{a}ximo del m\'{e}todo se puede calcular:%
%EndExpansion
\newline Suponiendo $\left\vert f^{\prime\prime}(v)\right\vert \leq K$ para
$a\leq v\leq b$. Si $E_{T}$ es el error del m\'{e}todo del trapecio
\[
\left\vert E_{T}\right\vert \leq\frac{K(b-a)^{3}}{12n^{2}}%
\]
Para utilizar este m\'{e}todo \ determinaremos varios subintervalos de la
velocidad.\newline Por ejemplo.

\item[i.] Para $n=15$
\begin{align*}
t  &  =3\left(  \frac{0.34}{48-12\sqrt{3}}+0.68\sum_{i=1}^{14}\frac
{1}{48-12\sqrt{3}-0.68i}+\allowbreak\frac{0.34}{37.\,\allowbreak8-12\sqrt{3}%
}\right)  s\\
t  &  =1.\,\allowbreak409\,3.s.
\end{align*}

\item[ii.] Para $n=30$%
\begin{align*}
t_{n}  &  =\left(  \frac{0.51}{48-12\sqrt{3}}+0.34\sum_{i=1}^{29}\frac
{3}{48-12\sqrt{3}-0.34i}+\allowbreak\frac{0.51}{37.\,\allowbreak8-12\sqrt{3}%
}\right)  s\\
&  =\allowbreak1.\,\allowbreak409\,1s
\end{align*}
Para $n=50$%
\begin{align*}
t_{n}  &  =\left(  \frac{0.306\,}{48-12\sqrt{3}}+0.204\,\sum_{i=1}^{49}%
\frac{3}{48-12\sqrt{3}-0.204\,i}+\allowbreak\frac{0.306\,}{37.\,\allowbreak
8-12\sqrt{3}}\right)  s\\
t_{n}  &  =:1.\,\allowbreak409028s
\end{align*}

\item[iii.] Para $n=55$%
\begin{align*}
t_{n}  &  =\left(  \frac{0.278\,18}{48-12\sqrt{3}}+0.185\,45\sum_{i=1}%
^{54}\frac{3}{48-12\sqrt{3}-0.185\,45i}+\allowbreak\frac{0.278\,18}%
{37.\,\allowbreak8-12\sqrt{3}}\right)  s\\
t_{n}  &  =\allowbreak1.\,\allowbreak409\,99s.
\end{align*}
Si observamos las gr\'{a}ficas de la funci\'{o}n cuando $n=15$ y $n=55$
observamos que el error debe ser menor para $n=55$ el cual calcularemos
anal\'{\i}ticamente ya que conocemos $t\left(  v\right)  ,$ aunque tambi\'{e}n
lo calcularemos de acuerdo con la f\'{o}rmula de error del m\'{e}todo\newline
\end{enumerate}
\end{solution}%

\begin{tabular}
[c]{ll}%
%TCIMACRO{\FRAME{ihFU}{3.1038in}{1.9545in}{0in}{\Qcb{m\'{e}todo del trapecio
%n=15}}{}{trap15.jpg}{\special{ language "Scientific Word";  type "GRAPHIC";
%maintain-aspect-ratio TRUE;  display "USEDEF";  valid_file "F";
%width 3.1038in;  height 1.9545in;  depth 0in;  original-width 3.0597in;
%original-height 1.9164in;  cropleft "0";  croptop "1";  cropright "1";
%cropbottom "0";  filename 'trap15.jpg';file-properties "XNPEU";}}}%
%BeginExpansion
{\parbox[b]{3.1038in}{\begin{center}
\includegraphics[
natheight=1.916400in,
natwidth=3.059700in,
height=1.9545in,
width=3.1038in
]%
{trap15.jpg}%
\\
m\'{e}todo del trapecio n=15
\end{center}}}%
%EndExpansion
&
%TCIMACRO{\FRAME{ihFU}{3.1038in}{1.9545in}{0in}{\Qcb{M\'{e}todo del trapecio
%n=55}}{}{trap50.jpg}{\special{ language "Scientific Word";  type "GRAPHIC";
%maintain-aspect-ratio TRUE;  display "USEDEF";  valid_file "F";
%width 3.1038in;  height 1.9545in;  depth 0in;  original-width 3.0597in;
%original-height 1.9164in;  cropleft "0";  croptop "1";  cropright "1";
%cropbottom "0";  filename 'trap50.jpg';file-properties "XNPEU";}}}%
%BeginExpansion
{\parbox[b]{3.1038in}{\begin{center}
\includegraphics[
natheight=1.916400in,
natwidth=3.059700in,
height=1.9545in,
width=3.1038in
]%
{trap50.jpg}%
\\
M\'{e}todo del trapecio n=55
\end{center}}}%
%EndExpansion
\end{tabular}
\newline Para calcular el error por truncamiento debemos calcular $\left|
f^{\prime\prime}(v)\right|  $ la cual es
\begin{align*}
f^{\prime}\left(  v\right)   &  =\frac{3}{\left(  48-12\sqrt{3}-v\right)
^{2}}\\
f^{\prime\prime}\left(  v\right)   &  =\frac{6}{\left(  48-12\sqrt
{3}-v\right)  ^{3}}\\
f^{\prime\prime\prime}\left(  v\right)   &  =\allowbreak\frac{18}{\left(
48-12\sqrt{3}-v\right)  ^{4}}.
\end{align*}
Como $f^{\prime\prime\prime}\left(  v\right)  >0$ entonces $f^{\prime\prime
}\left(  v\right)  $ no tiene maximo relativo, ya que es creciente por tanto
tomamos como
\[
K=\left|  f^{\prime\prime}\left(  10.2\right)  \right|  =\frac{6}{\left(
48-12\sqrt{3}-10.2\right)  ^{3}}=1.\,\allowbreak217\,9\times10^{-3}.
\]
As\'{i} tenemos que el error es
\begin{align*}
\left|  E_{T}\right|   &  \leq\frac{K(b-a)^{3}}{12n^{2}}=\frac{\left(
1.\,\allowbreak217\,9\times10^{-3}\right)  \left(  10.2\right)  ^{3}}{12n^{2}%
}\\
\text{si }n  &  =15,\text{\ \ \ }\left|  E_{T}\right|  =\frac{1.\,\allowbreak
292\,4}{\left(  12\right)  \left(  15\right)  ^{2}}=4.\,\allowbreak
786\,7\times10^{-4}\\
\text{si }n  &  =55,\text{\ \ \ }\left|  E_{T}\right|  =\frac{1.\,\allowbreak
292\,4}{\left(  12\right)  \left(  55\right)  ^{2}}=3.\,\allowbreak
560\,3\times10^{-5}%
\end{align*}
Mientras que los errores relativos para los mismos casos
\begin{align*}
&  \varepsilon\%=100\left|  \frac{\left(  t_{a}-t_{n}\right)  }{t_{a}}\right|
\\
\text{si }n  &  =15,\ \ \varepsilon\%=100\left|  \frac{\left(  1.\,\allowbreak
40899-1.\,\allowbreak409\,3\right)  }{1.\,\allowbreak409\,0}\right|
=2.\,\allowbreak129\,2\times10^{-2}\\
\text{si }n  &  =55,\ \ \varepsilon\%=100\left|  \frac{\left(  1.\,\allowbreak
408\,99-\allowbreak1.\,\allowbreak40\,91\right)  }{1.\,\allowbreak
40\,901}\right|  =7.\,\allowbreak806\,9\times10^{-3}%
\end{align*}
Tanto los errores de truncamiento como los errores relativos son peque\~{n}os

Para el caso de la distancia recorrida no realizaremos el an\'{a}lisis,
realizaremos el estudio utilizando fortran 77
\begin{verbatim}
c Este programa utiliza la regla del trapecio para
c determinar el tiempo que tarda un cuerpo en deslizarse
c desde la parte mas alta de un plano partiendo del reposo
c hasta alcanzar una velocidad v.
c En este problema se considera la resistencia del aire y la
c friccion del plano sobre el cuerpo
program trapeciom
real v0,vf,d,suma,h,Tn,Ta,et,X,sum,Ln,La,el,e,Y
integer i,j,n,res
res =1
30 if (res.eq.1)then
write(*,*)'******************************************************'
write(6,*)'Integracion numerica'
write(6,*)'Regla del Trapecio'
write(6,*)'******************************************************'
write(6,*)'Ingrese el valor de v0 entre (0 , 27.215)'
read(*,*) v0
write(6,*)'Ingrese el valor de vf mayor que v0 y menor que 27.215'
read(*,*) vf
write(6,*)'Ingrese el tama?o de la particion N, N menor de 60'
read(*,*) n
c Empieza el algoritmo para calcular la integral usando N subintervalos
d=(vf-v0)/n
suma=F(vf)+F(v0)
X=v0
do 10 i=2,n
X=X+d
suma=suma+2*F(X)
10 continue
Tn=(d/2.)*suma
c Ahora calculamos la distancia recorrida
h=Tn/n
sum=R(Tn)+R(0)
Y=0
do 20 j=2,n
Y=Y+h
sum=sum+2*R(Y)
20 continue
Ln=(h/2.)*sum
Ta=g(vf)-g(v0)
et=abs((Tn-Ta)/Ta)*100
La=P(Tn)-P(0)
el=abs((Ln-La)/La)*100
e=s(vf)/(12*n**2)
write(6,*)'***************************************************'
write(6,*)'Resultado para N=',N,':'
WRITE(*,'(3(T3,A/),3(T3,A/T3,(3(A,1PE13.6))/))')
&' ? Valor numerico? Valor analit. ? ',
&' ? estimado de la? de la integral? Diferencia ',
&' ? integral ? ? en % ',
&'?????????????????????????????????????????????????????????????',
&'Tiempo ? ' ,Tn , ' ? ' ,Ta , ' ? ',et ,
&'?????????????????????????????????????????????????????????????',
&'Dista. ? ' ,Ln , ' ? ' ,La , ' ? ',el
write(6,*) ' Error = ',e
write(6,*)'***************************************************'
WRITE(6,*)'El programa termino!'
print*, 'Si quiere continuar digie 1 si no cero'
read*, res
goto 30
ENDIF
open(10,file='trapeciom.txt',status='old')
write(10,200)'Resultado para N=',N,':'
write(10,201)'***************************************************'
write(10,202) ' Resultado de la Integral numerica para t: ',Tn
write(10,203) ' Resultado de la Integral analitica para t: ',Ta
write(10,204) ' Diferencia = ',et,'%'
write(10,205) ' Resultado de la Integral numerica para L: ',Ln
write(10,206) ' Resultado de la Integral analitica para L: ',La
write(10,207) ' Diferencia = ',el,'%'
write(10,208) ' Error = ',e
write(10,209)'***************************************************'
200 format(a20,i4,a2)
201 format(a50)
202 format(a45,f8.6)
203 format(a45,f8.6)
204 format(a15,f8.6,a2)
205 format(a45,f8.6)
206 format(a45,f8.6)
207 format(a15,f8.6,a2)
208 format(a10,D8.3)
209 format(a50)
end
c Declaracion de funciones
function F(x)
real x
F= 3./(48-(12*(sqrt(3.)))-x)
end
function g(x)
real x
g= 3*(3.3038-log(48.-(12*(sqrt(3.)))-x))
end
function R(x)
real x
R= (48.-12.*(sqrt(3.)))*(1.-exp(-(x/3.)))
END
function P(x)
real x
P= (48.-12.*(sqrt(3.)))*(x+3.*exp(-(x/3.))-3.)
END
function s(x)
real x
s=6./(48.-(12.*(sqrt(3.)))-x)**3
end
\end{verbatim}

Las salidas de este programa para los casos estudiados son:
\begin{verbatim}
   Resultado para N=  15 :
**************************************************
   Resultado de la Integral numerica para t: 1.409236
  Resultado de la Integral analitica para t: 1.408993
  Diferencia = 0.017268 %
   Resultado de la Integral numerica para L: 7.746278
  Resultado de la Integral analitica para L: 7.748781
  Diferencia = 0.032295 %
  Error = .451E-06
**************************************************

   Resultado para N=  30 :
**************************************************
   Resultado de la Integral numerica para t: 1.409054
  Resultado de la Integral analitica para t: 1.408993
  Diferencia = 0.004332 %
   Resultado de la Integral numerica para L: 7.746294
  Resultado de la Integral analitica para L: 7.746920
  Diferencia = 0.008076 %
  Error = .113E-06
**************************************************
   Resultado para N=  50 :
**************************************************
   Resultado de la Integral numerica para t: 1.409015
  Resultado de la Integral analitica para t: 1.408993
  Diferencia = 0.001591 %
   Resultado de la Integral numerica para L: 7.746302
  Resultado de la Integral analitica para L: 7.746526
  Diferencia = 0.002893 %
  Error = .406E-07
**************************************************

   Resultado para N=  55 :
**************************************************
   Resultado de la Integral numerica para t: 1.409011
  Resultado de la Integral analitica para t: 1.408993
  Diferencia = 0.001294 %
   Resultado de la Integral numerica para L: 7.746296
  Resultado de la Integral analitica para L: 7.746485
  Diferencia = 0.002444 %
  Error = .336E-07
**************************************************

\end{verbatim}

Vemos que los errores son menores ahora, pero eso era de esperar ya que cuando
utilizamos el m\'{e}todos anal\'{i}tico cometimos errores de redondeo que son
mucho menores cuando se usa el computador.

\subsubsection{METODO DE BISECCION}

\bigskip

Si f es una funci\'{o}n continua en el intervalo $\left[  a.b\right]  $ y si
$f\left(  a\right)  f\left(  b\right)  <0$, entonces $f$ debe tener un cero en
$\left(  a,b\right)  .$ Esta es una consecuencia del teorema del valor
intermedio para funciones continuas.

El m\'{e}todo de bisecci\'{o}n explota esta idea de la siguiente manera , si
$f\left(  a\right)  f\left(  b\right)  <0,$ se calcula $c=1/2(a+b)$ y se
averigua si $f\left(  a\right)  f\left(  c\right)  <0,$ si lo es, entonces $f$
tiene un cero en $\left[  a,c\right]  .$ A continuaci\'{o}n se renombra $c$
como $b$ y se comienza una vez mas con elintervalo $\left[  a,b\right]  ,$
cuya longitud es igual a la mitad de la longitud del intervalo original.

Si $f\left(  a\right)  f\left(  c\right)  >0,$ entoonces $f\left(  a\right)
f\left(  b\right)  <0,$y en este caso se renombra $c$ como $a$ , en ambos
casos se ha generado un nuevo intervalo que contiene un cero de $f$ y el
proceso puede repetirse.Claro est\'{a} que si $f\left(  a\right)  f\left(
c\right)  =0,$ entonces $f\left(  c\right)  =0$ y con ello se ha encontrado un
cero . Sin embargo por los errores de redondeo es poco factible que $f\left(
c\right)  =0,$ as\'{\i} el criterio para concluir no deber\'{a} depender de
que $\ f\left(  c\right)  $ sea cero . Se debe permitir una tolerancia
razonable tal como $f\left(  c\right)  <10^{-5}.$ Si hay varios ceros en el
intervalo dado, el m\'{e}todo de la bisecci\'{o}n encuentra uno cada vez. Este
m\'{e}todo tambi\'{e}n se conoce como el m\'{e}todo de la bipartici\'{o}n.

\bigskip

A la hora de programar es conveniente contar con varios criterios que detengan
el programa, uno es m\'{a}ximo n\'{u}mero de pasos que se permitiran, esto
reduce las posibilidades de que el programa se quede en un ciclo infinito. Por
otra parte la ejecuci\'{o}n del programa se puede detener ya sea cuando el
error sea lo suficientemente peque\~{n}o o cuando lo sea el valor de $f\left(
c\right)  .$

\bigskip

ANALISIS DE ERRORES

\bigskip Si $\left[  a_{0},b_{0}\right]  ,\left[  a_{1},b_{1}\right]  $,
...,$\left[  a_{n},b_{n}\right]  $ son los intervalos que resultan del
proceso, se pueden hacer las siguientes observaciones:%

\begin{align*}
a_{0}  &  \leqslant a_{1}\leqslant a_{2}\leqslant...\leqslant a_{n}\text{
}\left(  1\right) \\
b_{0}  &  \geqslant b_{1}\geqslant b_{2}\geqslant...\geqslant b_{1}\text{
}\left(  2\right) \\
b_{n+1}-a_{n+1}  &  =\frac{1}{2}\left(  b_{n}-a_{n}\right)  \text{
\ \ }\left(  n\geqslant0\right)  \text{ }\left(  3\right)
\end{align*}

La sucesi\'{o}n $\left[  a_{n}\right]  $ converge debido a que es creciente y
est\'{a} acotada superiormente , la sucesi\'{o}n $\left[  b_{n}\right]  $
tambi\'{e}n converge por razones analogas .Si se utiliza $\left(  3\right)  $
repetidamente se llega a que :%

\[
b_{n}-a_{n}=2^{-n}\left(  b_{0}-a_{0}\right)
\]

As\'{\i}%

\[
\lim_{n\rightarrow\infty}b_{n}-\lim_{n\rightarrow\infty}a_{n}=\lim
_{n\rightarrow\infty}2^{n}\left(  b_{0}-a_{0}\right)  =0
\]

Si se escribe%

\[
r=\lim_{n\rightarrow\infty}b_{n}=\lim_{n\rightarrow\infty}a_{n}%
\]

entonces tomando l\'{\i}mite en la desigualdad $0\geqslant f\left(
a_{n}\right)  f\left(  b_{n}\right)  ,$ entonces se obtiene $\ 0\geqslant
\left[  f\left(  r\right)  \right]  ^{2},$ y por tanto $f\left(  r\right)  =0.$

\bigskip

Sup\'{o}ngase que en cierta etapa del proceso \ se ha definido el intervalo
$\left[  a_{n},b_{n}\right]  .$ Si se detiene el proceso en este momento, la
raiz se encontrar\'{a} en este intervalo, en esta etapa la mejor
estimaci\'{o}n de la raiz es el punto medio del intervalo.

El error se acota de la siguiente forma:%

\[
\left|  r-c_{n}\right|  \leq\frac{1}{2}\left|  b_{n}-a_{n}\right|
\leq2^{-\left(  n+1\right)  }\left(  b_{0}-a_{0}\right)
\]

Resumiendo lo anterior \ se tiene el siguiente

TEOREMA:

Si $\left[  a_{0},b_{0}\right]  ,\left[  a_{1},b_{1}\right]  $, ...,$\left[
a_{n},b_{n}\right]  ,$..., denotan los intervalos en el m\'{e}todo de la
bisecci\'{o}n, entonces los l\'{\i}mites $\lim_{n\rightarrow\infty}b_{n}$ y
$\lim_{n\rightarrow\infty}a_{n},$ existen, son iguales y representan un cero
de $f.$ Si $r=\lim_{n\rightarrow\infty}c_{n}$ y $c_{n}=\frac{1}{2}\left(
a_{n}+b_{n}\right)  $, entonces:%

%TCIMACRO{\TEXTsymbol{\backslash}}%
%BeginExpansion
$\backslash$%
%EndExpansion
begin\{equation*\}%

%TCIMACRO{\TEXTsymbol{\backslash}}%
%BeginExpansion
$\backslash$%
%EndExpansion
left%
%TCIMACRO{\TEXTsymbol{\vert}}%
%BeginExpansion
$\vert$%
%EndExpansion
r-c\_\{n\}%
%TCIMACRO{\TEXTsymbol{\backslash}}%
%BeginExpansion
$\backslash$%
%EndExpansion
right%
%TCIMACRO{\TEXTsymbol{\vert}}%
%BeginExpansion
$\vert$%
%EndExpansion%
%TCIMACRO{\TEXTsymbol{\backslash}}%
%BeginExpansion
$\backslash$%
%EndExpansion
leq 2\symbol{94}\{-%
%TCIMACRO{\TEXTsymbol{\backslash}}%
%BeginExpansion
$\backslash$%
%EndExpansion
left( n+1%
%TCIMACRO{\TEXTsymbol{\backslash}}%
%BeginExpansion
$\backslash$%
%EndExpansion
right) \}%
%TCIMACRO{\TEXTsymbol{\backslash}}%
%BeginExpansion
$\backslash$%
%EndExpansion
left( b\_\{0\}-a\_\{0\}%
%TCIMACRO{\TEXTsymbol{\backslash}}%
%BeginExpansion
$\backslash$%
%EndExpansion
right)%

%TCIMACRO{\TEXTsymbol{\backslash}}%
%BeginExpansion
$\backslash$%
%EndExpansion
end\{equation*\}

En el siguiente ejemplo muestra un programa que resuelve una ecuaci\'{o}n
cuadr\'{a}tica introducida por un usuario.
\begin{verbatim}
c **Este programa halla las raices de una ecuaci?n cuadr?tica por el
c m?todo de la secante**
program biseccion
c **Declaraci?n de variables**
real G,H,I,D,R,u,v,w,a,b,c,e,dis
integer n,j,resp,k
D=0.00001
R=0.00001
n=300
resp=1
do while(resp.EQ.1)
print*, 'Este programa resuelve una ecuacion cuadratica:'
print*, 'ax**2 + bx +c '
Print*, 'Digite el coeficiente a'
read*, G
Print*, 'Digite el coeficiente b'
read*, H
Print*, 'Digite c'
read*, I
dis=(H**2)-(4*G*I)
if(dis.GT.0)then
Print*, 'Digite el l?mite inferior del intervalo'
read*, a
Print*, 'Digite el l?mite superior del intervalo'
read*, b
write(*,*)'La funcion es : ',G,'x**2 ',H,'x ', I
u= (G*(a**2))+(H*a)+I
v= (G*(b**2))+(H*b)+I
if(u.EQ.0)then
print*,'La raiz es',a
else
if(v.EQ.0)then
print*,'La raiz es',b
else
k=1
do while((v*u.GE.0).and.(k.LT.30))
k=k+1
a=a+0.1
u= (G*(a**2))+(H*a)+I
v= (G*(b**2))+(H*b)+I
enddo
e=b-a
j=1
if((u*v).LT.0)then
do while(j.LT.n)
j=j+1
e=e/2
c=a+e
w=(G*(c**2))+(H*c)+I
if(abs(e).GE.D)then
if(abs(w).GE.R)then
if((u*w).LT.0)then
b=c
v=w
else
a=c
u=w
endif
else
j=n
endif
else
j=n
endif
enddo
write(*,*)'*********************************************'
print*, 'El numero de iteraciones es',j
print*, 'La raiz es',c
print*, 'El error es',e
write(*,*)'*********************************************'
else
print*, 'No existen raices en el intervalo dado,'
print*,'o el la longitud del intervalo es muy grande'
endif
endif
endif
else
print*,'Esta ecuaci?n no tiene raices reales'
endif
open(10,file='biseccion.txt',status='unknown')
write(10,201)'***************************************************'
write(10,200)'La funcion es : ',G,'x**2 ',H,'x ', I
write(10,202) ' El numero de iteraciones es : ',j
write(10,203) ' La raiz es: ',c
write(10,204) ' El error es ',e
write(10,205)'***************************************************'
201 format(a50)
200 format(a15,f8.6,a5,f8.3,a2,2f8.3)
202 format(a25,i4)
203 format(a15,f8.6)
204 format(a15,f8.6,a2)
205 format(a50)
print*,'Si desea continuar digite 1, sino digite 0'
read*,resp
enddo
end
\end{verbatim}

Utlizaremos el programa para calcular la ra\'{i}z que se encuentra entre
$[0,3]$ de la ecuaci\'{o}n $x^{2}-5x+3=0$ . en la gr\'{a}fica se presenta como
trabaja m\'{a}s o menos el programa%
%TCIMACRO{\FRAME{dhFU}{2.4924in}{1.8109in}{0pt}{\Qcb{Figura5 . M\'{e}todo de
%bisecci\'{o}n}}{}{bisec1.jpg}{\special{ language "Scientific Word";
%type "GRAPHIC";  maintain-aspect-ratio TRUE;  display "USEDEF";
%valid_file "F";  width 2.4924in;  height 1.8109in;  depth 0pt;
%original-width 8.0315in;  original-height 5.8124in;  cropleft "0";
%croptop "1";  cropright "1";  cropbottom "0";
%filename 'bisec1.jpg';file-properties "XNPEU";}}}%
%BeginExpansion
\begin{center}
\includegraphics[
natheight=5.812400in,
natwidth=8.031500in,
height=1.8109in,
width=2.4924in
]%
{bisec1.jpg}%
\\
Figura5 . M\'{e}todo de bisecci\'{o}n
\end{center}
%EndExpansion
y la salida es la siguiente
\begin{verbatim}
**************************************************
La funcion es :1.000000x**2   -5.000x    3.000
 El numero de iteraciones 300
   La raiz es: 0.697226
   El error es 0.000006
**************************************************
\end{verbatim}
\end{document}
