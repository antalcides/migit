%% This document created by Scientific Word (R) Version 3.5

%TCIDATA{LaTeXparent=0,0,Est12.tex}
%TCIDATA{ChildDefaults=%
%chapter:5,page:150
%}


\section{Distribuciones}

\subsection{Algunas distribuciones discretas importantes}

\subsubsection{Distribuci\'{o}n binomial}

Se dice que una variable aleatoria $X$ tiene una distribuci\'{o}n binomial de
pr\'{a}metros $n,p$ si $X=\sum_{i=1}^{n}X_{i}$ donde cada $X_{i}%
\leadsto\mathbb{B}\left(  p,pq\right)  $ y se denota $B\left(  np,npq\right)  $%

\begin{definition}
Supongase que se realiza un experimento de Bernoulli n veces , donde en todas
ellas, la probabilidad de \'{e}xito es la misma (p), y queremos calcular el
n\'{u}mero de \'{e}xitos, X obtenidos en el total de n pruebas, entonces su
funci\'{o}n de probabilidad es \[ f\left(    x\right)    =\left\{
\begin{array}
[c]{ccc} \binom{n}{x}p^xq^n-x & si & x=0,1,2,\cdots,n\\ 0 &  & \text{en otro
caso}
\end{array}
\right.   \] Por tanto, su funci\'{o}n de distribuci\'{o}n es \[ F\left(
x\right)    =\left\{
\begin{array}
[c]{ccc} 0 & si & x<0\\ \sum_k=0^[x]\binom{n}{k}p^kq^n-k & si & n\geq x\geq0\\
1 & si & x\geq n
\end{array}
\right.   \] La esperanza y la varianza se calcularon en los ejemplos 4.23 y
4.27 \[ E\left(    X\right)    =np,V\left(    X\right)    =npq \]
\end{definition} 

\subsubsection{Distribucci\'{o}n de Poisson}

Sea $X$ una variable aleatoria con una distribuci\'{o}n discreta y
sup\'{o}ngase que el valor de $X$ debe ser un entero no negativo. Se dice que
$X$ tiene una distribuci\'{o}n de Poisson con media $\lambda$ si la f.p de $X$
es
\[
f\left(  x\right)  =\left\{
\begin{array}
[c]{ccc}%
\frac{e^{-\lambda}\lambda^{x}}{x!} & si & x=1,2,3,\cdots\\
0 &  & \text{ en otro caso}%
\end{array}
\right.
\]
Se denota $X\leadsto P\left(  \lambda,\lambda\right)  $

\subsection{Algunas distribuciones continuas}

\subsubsection{Distribuci\'{o}n exponencial}

La distribuci\'{o}n exponencial es el equivalente continuo por asi decirlo de
la distribuci\'{o}n geom\'{e}trica discreta, Esta describe un proceso en el que.

Nos interesa saber el tiempo hasta que ocurre determinado evento, sabiendo que
el tiempo que pueda ocurrir desde cualquier instante dado $t$, hasta que ello
acurra en un instante $t_{f}$, no depende del tiempo trascurrido anteriormente
en el que no pas\'{o} nada

Si $X$ es una v.a se dice que tiene una distribuci\'{o}n exponencial si
\[
f\left(  x\right)  =\left\{
\begin{array}
[c]{ccc}%
\lambda e^{-\lambda x} & si & x>0\\
0 &  & \text{si no}%
\end{array}
\right.
\]
Su experanza y su varianza son
\begin{align*}
E\left(  X\right)   &  =\frac{1}{\lambda}\\
V\left(  X\right)   &  =\frac{1}{\lambda^{2}}%
\end{align*}
Distribuci\'{o}n normal o Gaussiana

Se dice que una v.a tiene una distribuci\'{o}n normal de param\'{e}tros $\mu$
y $\sigma^{2}$ lo que se denota $X\leadsto N\left(  \mu,\sigma^{2}\right)  $
si su funci\'{o}n de densidad es
\[
F\left(  x\right)  =P\left(  x\geq X\right)  =\int_{-\infty}^{x}f\left(
x\right)  dx
\]
La media y la varianza son
\begin{align*}
E\left(  X\right)   &  =\mu\\
V\left(  X\right)   &  =\sigma^{2}%
\end{align*}

\begin{itemize}
\item Estas variables tienen una propiedad importante que se llama
reproductividad, es decir la suma de variables aleatorias \ e independientes
normales tambi\'{e}n es normal

\item Una propiedad de esta distribuci\'{o}n es que es sim\'{e}trica con
respecto a la media

\item Si $X\leadsto N\left(  0,1\right)  $ se dice que $X$ tiene una
distribuci\'{o}n normal est\'{a}ndar y su f.d se denota $\varphi\left(
X\right)  $ y su funci\`{o}n de distribuci\'{o}n $\Phi\left(  X\right)  $

\item Los puntos de inflexi\'{o}n est\'{a}n en $x=\mu\pm\sigma$

\item $\lim_{x\longrightarrow+\infty}f\left(  x\right)  =0\;\;$y$\;\;\lim
_{x\longrightarrow-\infty}f\left(  x\right)  =0$
\end{itemize}