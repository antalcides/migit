%% This document created by Scientific Word (R) Version 3.0

\documentclass{article}
\usepackage{graphicx}
\usepackage{amsmath}
\usepackage{amsfonts}
\usepackage{amssymb}
%TCIDATA{OutputFilter=latex2.dll}
%TCIDATA{CSTFile=LaTeX article (bright).cst}
%TCIDATA{Created=Fri Nov 14 13:09:39 2003}
%TCIDATA{LastRevised=Fri Nov 14 15:10:13 2003}
%TCIDATA{<META NAME="GraphicsSave" CONTENT="32">}
%TCIDATA{<META NAME="DocumentShell" CONTENT="General\Blank Document">}
\newtheorem{theorem}{Theorem}
\newtheorem{acknowledgement}[theorem]{Acknowledgement}
\newtheorem{algorithm}[theorem]{Algorithm}
\newtheorem{axiom}[theorem]{Axiom}
\newtheorem{case}[theorem]{Case}
\newtheorem{claim}[theorem]{Claim}
\newtheorem{conclusion}[theorem]{Conclusion}
\newtheorem{condition}[theorem]{Condition}
\newtheorem{conjecture}[theorem]{Conjecture}
\newtheorem{corollary}[theorem]{Corollary}
\newtheorem{criterion}[theorem]{Criterion}
\newtheorem{definition}[theorem]{Definition}
\newtheorem{example}[theorem]{Example}
\newtheorem{exercise}[theorem]{Exercise}
\newtheorem{lemma}[theorem]{Lemma}
\newtheorem{notation}[theorem]{Notation}
\newtheorem{problem}[theorem]{Problem}
\newtheorem{proposition}[theorem]{Proposition}
\newtheorem{remark}[theorem]{Remark}
\newtheorem{solution}[theorem]{Solution}
\newtheorem{summary}[theorem]{Summary}
\newenvironment{proof}[1][Proof]{\textbf{#1.} }{\ \rule{0.5em}{0.5em}}



\begin{document}

\begin{center}
ASIGNACION N$%
%TCIMACRO{\UNICODE{0xb0}}%
%BeginExpansion
{{}^\circ}%
%EndExpansion
4$ DE FISICA MATEMATICA \ Y

COMPUTACIONAL.
\end{center}

\begin{enumerate}
\item  Halle la aproximaci\'{o}n en diferencias hacia delante y hacia
atr\'{a}s de orden $\left(  \Delta x\right)  $para $\frac{\partial^{6}%
f}{\left(  \partial x\right)  ^{6}}$.(sugerencia: utilice las ecuaciones 19,20
y 21 de las notas)
\end{enumerate}

\bigskip SOLUCION

\begin{enumerate}
\item  Hacia adelante
\end{enumerate}%

\begin{align*}
\frac{\partial^{6}f}{\left(  \partial x\right)  ^{6}}  & =\frac{\left(
\Delta^{6}xf_{i}\right)  }{\left(  \Delta x\right)  ^{6}}\\
& =\frac{\Delta x^{5}\left(  \Delta xf_{i}\right)  }{\left(  \Delta x\right)
^{6}}\\
& =\frac{\Delta x^{5}\left(  f_{i+1}-f_{i}\right)  }{\left(  \Delta x\right)
^{6}}\\
& =\frac{\Delta x^{4}\left(  \Delta x\left(  f_{i+1}-f_{i}\right)  \right)
}{\left(  \Delta x\right)  ^{6}}\\
& =\frac{\Delta x^{4}\left(  f_{i+2}-2f_{i+1}+f_{i}\right)  }{\left(  \Delta
x\right)  ^{6}}\\
& =\frac{\Delta x^{3}\left(  \Delta x\left(  f_{i+2}-2f_{i+1}+f_{i}\right)
\right)  }{\left(  \Delta x\right)  ^{6}}\\
& =\frac{\Delta x^{3}\left(  f_{i+3}-3f_{i+2}+3f_{i+1}-f_{i}\right)  }{\left(
\Delta x\right)  ^{6}}\\
& =\frac{\Delta x^{2}\left(  \Delta x\left(  f_{i+3}-3f_{i+2}+3f_{i+1}%
-f_{i}\right)  \right)  }{\left(  \Delta x\right)  ^{6}}\\
& =\frac{\Delta x^{2}\left(  f_{i+4}-4f_{i+3}+6f_{i+2}-4f_{i+1}+f_{i}\right)
}{\left(  \Delta x\right)  ^{6}}\\
& =\frac{\Delta x\left(  \Delta x\left(  f_{i+4}-4f_{i+3}+6f_{i+2}%
-4f_{i+1}+f_{i}\right)  \right)  }{\left(  \Delta x\right)  ^{6}}\\
& =\frac{\Delta x\left(  f_{i+5}-5f_{i+4}+10f_{i+3}-10f_{i+2}+5f_{i+1}%
-f_{i}\right)  }{\left(  \Delta x\right)  ^{6}}\\
\frac{\partial^{6}f}{\partial x^{6}}  & =\frac{\left(  f_{i+6-}6f_{i+5}%
+15f_{i+4}-20f_{i+3}+15f_{i+2}-6f_{i+1}+f_{i}\right)  }{\left(  \Delta
x\right)  ^{6}}%
\end{align*}

Hacia atr\'{a}s:%

\begin{align*}
\frac{\partial^{6}f}{\left(  \partial x\right)  ^{6}}  & =\frac{\left(
\nabla^{6}xf_{i}\right)  }{\left(  \Delta x\right)  ^{6}}\\
& =\frac{\nabla^{5}x\left(  \Delta xf_{i}\right)  }{\left(  \Delta x\right)
^{6}}\\
& =\frac{\nabla^{5}x\left(  f_{i}-f_{i-1}\right)  }{\left(  \Delta x\right)
^{6}}\\
& =\frac{\nabla^{4}x\left(  \nabla x\left(  f_{i}-f_{i-1}\right)  \right)
}{\left(  \Delta x\right)  ^{6}}\\
& =\frac{\Delta x^{4}\left(  f_{i}-2f_{i-1}+f_{i-2}\right)  }{\left(  \Delta
x\right)  ^{6}}\\
& =\frac{\nabla^{3}x\left(  \nabla x\left(  f_{i}-2f_{i-1}+f_{i-2}\right)
\right)  }{\left(  \Delta x\right)  ^{6}}\\
& =\frac{\nabla^{3}x\left(  f_{i}-3f_{i-1}+3f_{i-2}-f_{i-3}\right)  }{\left(
\Delta x\right)  ^{6}}\\
& =\frac{\nabla^{2}x\left(  \nabla x\left(  f_{i}-3f_{i-1}+3f_{i-2}%
-f_{i-3}\right)  \right)  }{\left(  \Delta x\right)  ^{6}}\\
& =\frac{\nabla^{2}x\left(  f_{i}-4f_{i-1}+6f_{i-2}-4f_{i-3}+f_{i-4}\right)
}{\left(  \Delta x\right)  ^{6}}\\
& =\frac{\nabla x\left(  \nabla x\left(  f_{i}-4f_{i-1}+6f_{i-2}%
-4f_{i-3}+f_{i-4}\right)  \right)  }{\left(  \Delta x\right)  ^{6}}\\
& =\frac{\nabla x\left(  f_{i}-5f_{i-1}+10f_{i-2}-10f_{i-3}+5f_{i-4}%
-f_{i-5}\right)  }{\left(  \Delta x\right)  ^{6}}\\
\frac{\partial^{6}f}{\partial x^{6}}  & =\frac{\left(  f_{i-}6f_{i-1}%
+15f_{i-2}-20f_{i-3}+15f_{i-4}-6f_{i-5}+f_{i-6}\right)  }{\left(  \Delta
x\right)  ^{6}}%
\end{align*}

\begin{enumerate}
\item  Dada la funci\'{o}n escriba un programa donde calcule las primeras dos
derivadas en x = 1.5 en diferencias centradas de orden para los siguientes
'step sizes' 0.0005, 0.001, 0.01, 0.1, 0.2. Determine el error num\'{e}rico
para cada c\'{o}mputo

\item  Dada la funci\'{o}n escriba un programa para calcular en x = 0.5 y x =
1.5 por diferencias hacia delante y hacia atr\'{a}s de orden y diferencias
centradas de orden . Use 'step sizes' de 0.00001, 0.0001, 0.001, 0.1 y 0.5.
Haga una gr\'{a}fica del error para cada una de los c\'{o}mputos.
\end{enumerate}
\end{document}