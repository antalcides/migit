% example LaTeX file using Prosper to make PDF presentation
%
\documentclass[pdf,geoffs,slideColor,colorBG,total,accumulate]{prosper}
%'geoffs' is my hack of the style to make it plain white background
% this is the file PPRgeoffs.sty
%
% load all the packages you require...and maybe some you dont
%
\usepackage{pstricks,pst-node,pst-text,pst-3d}
% pst* are for drawing within LaTeX
\usepackage{moreverb,epsfig,color,subfigure}
% how to define new colors
% \newrgbcolor{the name}{r g b}
\newrgbcolor{hotpink}{0.9 0 0.5}
%
% put a logo in the corner
\Logo(\textwidth,-0.5){\includegraphics[width=0.08\textwidth]
{/home/gnm/TeX/Crests/unswcrest-colour}}
% puts the logo unswcrest-colour.eps in the bottom corner
% you need to change the location of the file to where you have yours.....
% fiddle the numbers to change the size and location on the page
%
% font stuff....
% define a new font called goodfont
\def\goodfont{\usefont{T1}{pcr}{b}{n}\fontsize{36pt}{40pt}\selectfont\green}
\renewcommand{\familydefault}{\rmdefault}
\renewcommand{\rmdefault}{cmr}
%
\usepackage[ps2pdf]{hyperref}  % this has to be the last package loaded
%\usepackage[hdvipdfm]{hyperref}  % this has to be the last package loaded


\parindent 0pt
\parskip 5pt

%% the talks title goes here, this also used as a bottom runner
% unless you specify \slideCaption{......}
% \maketitle makes a title page using the following information
% but i didnt like it so made my own as the first slide
% hence \maketitle is commented out
%
\title{{\blue The title}}
\slideCaption{\blue An Introduction to Prosper}   % the bottom runner
% subtitle if wanted
\subtitle{{\red MATLAB section}}
\author{\href{http://www.ma.adfa.edu.au/~gnm}{\goodfont \hotpink Geoff Mercer}}
% uncomment the above line to make it hyperlinked to my home page
\institution{%
  \goodfont School of Mathematics and Statistics,\\UNSW at ADFA}

%% PDFtransition
%% Transition from one page to another.
%% Possible values:
%%       (Split, Blinds, Box, Wipe, Dissolve, Glitter, R)
%\def\Split{} \def\Blinds{} \def\Box{} \def\Wipe{} \def\Dissolve{}
%\def\Glitter{} \def\R{} \def\Replace{}
% used like    \begin{slide}[Wipe]{the title of the slide in here}
% dont overdo it, replace is usually the best....


%-------------------------------------------------------------------------------

\begin{document}
%\maketitle
% uncommenting maketitle means propser will make the title page,
% i wanted my own title page below.....
% it will use the title as a runner at the bottom if you want it
% or \slideCaption{......} if that is defined, to turn off just use
%  \slideCaption{}


%----------------------------------------------------------SLIDE -
\overlays{1}{% not strictly necessary when only one overlay
% 1=number of different overlays on this page
\begin{slide}[Split]{}
\begin{center}
\vspace*{-1cm}
\Large {Prosper: making PDF presentations}\\[1cm]
{\hotpink Geoff Mercer }\\[1cm]
{\dgreen School of Mathematics and Statistics\\[4mm]
UNSW at ADFA}% 
% dgreen is a colour defind in PPRgeoffs.sty
\end{center}
\end{slide}
}%overlays

%----------------------------------------------------------SLIDE -
\overlays{8}{%
\begin{slide}[Replace]{What is Prosper} % uses Replace transition
\begin{itemstep}
\item A LaTeX style file for
producing PDF presentations
\item For example
\begin{itemize}
\item item stepping
\item all the usual PDF transitions if you want them\\
      Split, Blinds, Box, Wipe, Dissolve, Glitter, Replace
\item different backgrounds, that are customizable
\item easy to make overheads (postscript) of the same presentation
      as a backup copy
\end{itemize}
\item It is based on the seminar class
\item At a basic level very easy to use but can do lots of fancy things
      if you get into it
\end{itemstep}
\end{slide}
}%overlays

%----------------------------------------------------------SLIDE -
\overlays{6}{%
\begin{slide}[Replace]{How to use Prosper}
\begin{itemstep}
  \item Once the Prosper style is installed on your machine
        just use the documentclass prosper (instead of seminar or whatever else you might have used)  
  \item how to run it: \ if your LaTeX file is filename.tex
  {\hotpink 
  \begin{itemize}
    \item latex filename ;  
    \item dvips filename; 
    \item ps2pdf filename.ps
  \end{itemize}
  }
\item You can do all the usual LaTeX things \ldots
\end{itemstep}
\end{slide}
}%overlays

%----------------------------------------------------------SLIDE -
\overlays{8}{%
% 6=number of different overlays on this page 
\begin{slide}[Box]{Usual LaTeX stuff}

\fromSlide*{2}{Using PStricks} 
\fromSlide*{2}{\rnode{N1}}
\onlySlide*{1}{\rnode{N2}}%

\fromSlide*{3}{watch for the line.... coming in here....(\rnode{N2}{X}) }
\fromSlide*{4}{\nccurve[linecolor=purple,linewidth=4pt, 
angleA=0,angleB=120]{->}{N1}{N2}}

\bigskip
\onlySlide*{5}{this text is only on this overlay...}
\fromSlide*{6}{\dgreen from now onwards it is  this text....}

\blue
\vspace*{0.5cm}
\fromSlide*{7}{\rput[tl](6,-2){\rnode{A}{\psframebox{Node A}}}
\rput[tr](2,0){\ovalnode{B}{Node B}}
\ncdiag[angleA=70,angleB=90,arm=.5,linearc=.2,linewidth=3pt]{A}{B} }

\vspace*{2.5cm}
\fromSlide*{8}{ The usual fonts \\
{\bf bold} {\large large} {\bf \large bold
  large} {\Large \purple Large purple}}

\end{slide}
}%overlays

%----------------------------------------------------------SLIDE -
\overlays{6}{%
% 6=number of different overlays on this page 
\begin{slide}[Glitter]{More Prosper} % uses Glitter type transition
\hypertarget{Moreprosper}{}%
%this puts a link so you can click on a link later to return here
% the position is called Moreprosper

This page has 'Glitter' Transition\\[4mm]

\fromSlide{1}{Clickable links to web pages}
\url{http://www.ma.adfa.edu.au/~gnm}\\%
% need the browser set up in acroread for this to work and an 
% internet connection if the page is not local

\fromSlide{2}{Or links to other pages in the presentation\\
For example the 
\hyperlink{lastpage}{last} page}
%a link to the position lastpage

\bigskip 
\fromSlide{3}{The abox command that I wrote will do the current text highlighting}
\onlySlide*{4}{\abox{4}{5}{coloured maths turns red to blue {\red $ x=y$}}}
        \fromSlide*{5}{\abox{4}{5}{coloured maths turns red to blue {\blue $ x=y$}}}
% \abox is a command i wrote it is in PPrGeoffs.sty

\bigskip 
\abox{6}{7}{\fbox{this is boxed text}}
\end{slide}
}%overlays

%\end{document}
%----------------------------------------------------------SLIDE -
\overlays{10}{%
\begin{slide}[Replace]{Colour Changes}


\begin{itemstep}
  \begin{minipage}[c]{0.4\textwidth}
      \FromSlide{1}\item dot points%
      \FromSlide{2}\item  using%
  \end{minipage}
  \begin{minipage}[c]{0.4\textwidth}
      \FromSlide{3}\item minipages %
      \FromSlide{4}\item works too\ldots%
  \end{minipage}
\end{itemstep}


% example of changing color in test or equations
% use the same equation or text and vary the colours
% just have multiple copies of the same text and turn the colours on and off
% and show at different onlySlide
\bigskip
\fromSlide*{5}{Maths that changes colour: \dgreen governing equation }
\onlySlide*{6}{{\dgreen $${\frac{\partial F}{\partial t}}={\frac{\partial^2 F}{\partial x^2} + \int g(x) dx}$$}}
\onlySlide*{7}{$${\red \frac{\partial F}{\partial t}}=\frac{\partial^2 F}{\partial x^2} + \int g(x) dx$$}
\onlySlide*{8}{$${\frac{\partial F}{\partial t}}={\red\frac{\partial^2 F}{\partial x^2}} + \int g(x) dx$$}
\onlySlide*{9}{$${\frac{\partial F}{\partial t}}=\frac{\partial^2 F}{\partial x^2} + {\red\int g(x) dx}$$}
\fromSlide*{10}{$${\frac{\partial F}{\partial t}}={\frac{\partial^2 F}{\partial x^2} + \int g(x) dx}$$}
\onlySlide*{7}{\red rate of change of $F$ w.r.t time}
\onlySlide*{8}{\hspace*{3.0cm}\red diffusion of $F$}
\onlySlide*{9}{\hspace*{5cm}\red integral of $g(x)$}

\fromSlide*{10}{\blue Now take the \dgreen Laplace \hotpink Transform}
\end{slide}
}%overlays


%----------------------------------------------------------SLIDE -
\overlays{4}{%
\begin{slide}[Replace]{graphics ?}
This page has 'Replace' transition and shows how to 
include a graphic or two\\
% puts the 3 figures prosperplot1.eps, prosperplot2.eps, prosperplot3.eps
% in the same place one after the other, usual way of building up a graph
% these plots where done in MATLAB but i made sure i thickened the 
% line widths and fonts
\vspace*{0cm}
\onlySlide*{2}{%
\begin{center}
\includegraphics[width=0.6\textwidth]{prosperplot1.eps}
\end{center} }
\onlySlide*{3}{%
\begin{center}
\includegraphics[width=0.6\textwidth]{prosperplot2.eps}
\end{center} }
\onlySlide*{4}{%
\begin{center}
\includegraphics[width=0.6\textwidth]{prosperplot3.eps}
\end{center} }
\end{slide}
}%overlays


%---------------------------------------------------------------------- SLIDE -
\overlays{4}{%
\begin{slide}[Blinds]{Last slide}
  This is the \hypertarget{lastpage}{last} slide. Do you want to go to the
  \hyperlink{Moreprosper}{\underline{More Prosper page}}?
  \href{http://www.ma.adfa.edu.au/~gnm}{\underline{Geoff's home page}} 
  or a combustion \href{http://www.ma.adfa.edu.au/~gnm/Animations/2d-small.mpg}
  {\underline{animation}}
  and a local \href{run:video.sh}{running video.sh}\\
  you will need to configure your browser to work.....

\FromSlide{2}
\begin{itemstep}
\item somethings that may make it work better depending on  your set up
\FromSlide{3}   
\begin{itemize}
   \item use the flag -Ppdf with dvips  eg dvips -Ppdf filename
   \item put p +psfonts.cmz \newline
         and p +psfonts.amz \newline
         in your .dvipsrc file in your home directory
   \end{itemize}
\end{itemstep}
\fromSlide*{4}{\hotpink If you want any help just ask me}
\end{slide}
}
%---------------------------------------------------------------------- SLIDE -


\end{document}
