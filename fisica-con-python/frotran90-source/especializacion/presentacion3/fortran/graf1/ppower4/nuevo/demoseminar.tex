%% demoseminar.tex
%%
%% Copyright (c) 1999 David Wilson
%% Utopia Precision Typesetting
%%
%% demo of Tex/Acrobat capabilities
%% using seminar.cls with u-build and u-background
%%
%% * make sure minisketch.pro is where dvipsone (or dvips) can find it
%% * you'll probably need to change "nu" to some locally-available font
%%   in the definition of \Fancy below
%% * in the \NewSound command below you will need to change
%%   the pathname to provide your PostScript to PDF converter with
%%   enough information to locate the whistle1.dat file
%%
%% if you are using Ghostscript
%% * you may need to put \makeatletter\let\U@SoundInputFilter\empty\makeatother
%%   in the document header to get the sound to work
%% * depending on the version of Ghostscript, you may need to change the
%%   \LinkBorderWidth units (instead of .5pt you could try 2sp)
%% * the flyingbuttress background may cause problems if you don't
%%   specify a screen-appropriate resolution
%%
%% Depending on your dvi previewer's colour support you may need to
%%   comment out the \color and \pagecolor commands just after
%%   the \begin{document} below
%%
\documentclass{seminar}
%% comment out the following line if you are not using Y&Y TeX
\usepackage[LY1]{fontenc}
\usepackage{semhelv}
\usepackage{graphicx}
\usepackage[dvipsone]{u-background} %% remove the [dvipsone] option if you are not using Y&Y TeX
\usepackage{u-build}
%%
%% Document Info for PDF
%%
\PDFDocInfo{title=Slideshow demo (LaTeX version),
            subject=Demo of TeX/Acrobat capabilities,
            author={DC Wilson, Utopia Precision Typesetting},
            keywords=seminar u-build u-background}
%%
%% setting up for the demo*.eps graphics used below
%%
\AtBeginDocument{\special{header=minisketch.pro}
  \special{! /TextColour {254 255 div 254 255 div 254 255 div } def}}
%%
%% set up the slides (seminar.cls)
%%
\slideframe{none}
\newpagestyle{u}{\NoDim{\tiny\color{Uwhite}\MyHeadXtra\hfil0.\thepage}}{}
\let\MyHeadXtra\empty
\slidepagestyle{u}
\slidesmag{7}
%%
%% from plain TeX, because I don't like LaTeX's equation alignments
%%
\def\eqalign#1{\null\,\vcenter{\openup1\jot\mathsurround=0pt%
  \everycr={}\tabskip=0pt\halign
  {\strut\hfil$\displaystyle{##}$&$\displaystyle{{}##}$\hfil\crcr#1\crcr}}\,}
\def\hb{\hfil\break}
%%
%% A few non-default values
%%
\parindent=0pt
\font\tenar=msam10 at8pt
\BblobSymbol{\tenar\char"07}
\BbblobSymbol{\tenar\char"46}
\NavButtonBorderWidth{0pt}\NavButtonSep{-20pt}\NavButtonTextColor{Uwhite}
\NavButtonSpace{0pt plus1fil}\PlaceNavButtons{bl}
\LinkBorderWidth{0pt}
%%
\begin{document}
\NewSound{whistle1}{c:/tex/utopiatype/products/bundle/sounds/whistle1.dat}
\begin{slide}
\color{Uwhite}\pagecolor{black} % so we can see the text in the dvi viewer
 
\Heading{\advance\baselineskip-10pt\TeX\ and Acrobat:\\\hfill
 better than PowerPoint?}

% to put explanatory note at the bottom of the page
% but before the first build
\BackgroundOverlay{\rightline{\color{Uwhite}\tiny(use PgDn or Enter to move forward)}\medskip}

\Build Microsoft's PowerPoint is the most commonly used
application for computer-assisted presentations

\Build However, PowerPoint works best with simple text-and-graphics
presentations\Build, not with technical --- and especially
mathematical --- material

\newslide\DimmingOn\PageTransition{Dissolve}
\BackgroundOverlay{} % turn off the first-page explanatory note
Using \TeX\ to format the `slides' and Adobe Acrobat to display them
(in full-screen mode) provides a viable --- and much more
mathematics-friendly --- alternative

\Blob[transition=Dissolve] builds are easily effected\hb
{\scriptsize(with earlier stuff dimmed if you want)}

\Bblob[transition=Dissolve] with the usual blobs at various levels

\Bbblob[transition=Dissolve] Acrobat Reader provides transition effects
{\scriptsize(if you like that kind of thing)}

\newslide\PageTransition{BlindsVert}
\Background{horizontal}{Ublue}
\blob different backgrounds are possible

\Blob Mathematics can be run inline: \Build you can
discuss implicit differentiation of $\sqrt{x^3- y}$
or the graph of $\cos(3\pi z_\alpha)$ without
jumping through typographical hoops

\vfil
\Build[sound=whistle1]
\Heading{Main Theorem}\Destination{MagnumOpus}
If you wish you can announce your major points with a flourish
{\scriptsize (on Windows and Macs anyway)}

\DimmingOff
\newslide\PageTransition{BoxOut}
\Background{horizontal}{Ugreen}

\blob In PowerPoint, mathematics has to be included as
an embedded object or as a graphic\Build, leading to

\blob difficulty getting consistent sizing

\Bblob difficulty with alignment

\Bblob difficulty with `building' the display of the mathematics

\Bblob difficulty ensuring all needed fonts are installed on the displaying machine

\newslide\PageTransition{Replace}
\Background{diagonal}{Uviolet}

\blob Builds can be inserted almost anywhere: \Build
to work through a mathematical calculation---

\Bblob {\scriptsize We have $\let\vcenter\vtop
\eqalign{SE&=\sqrt{{s_A^2\over n_A}+{s_B^2\over n_B}}\cr\Build
&=\sqrt{{(6.8)^2\over75}+{(7.5)^2\over100}}\cr\Build
&=1.086\cr\Build
\llap{compare {} }{s_A\over\sqrt{n_A}}&=0.785\cr
{s_B\over\sqrt{n_B}}&=0.750\cr}$

}
\newslide
\Background{Diagonalcurve}{Ublue}
\blob between two graphics side by side:
$$\includegraphics[width=2cm]{demo3.eps}
  \Build\qquad\qquad
  \includegraphics[width=2cm]{demo4.eps}$$
\Build followed by more text

\vfill
{\tiny(getting PowerPoint to build an arbitrary sequence of what it considers
to be different objects---text, graphics, equations, etc---requires significant effort)

}

\newslide
\Background{Vertical}{Ublue}
\blob\hbox{with a bit of effort, in the middle of a graphic:}
$$\includegraphics[width=2.3cm]{demo1.eps}\Build
  \llap{\includegraphics[width=2.3cm]{demo2.eps}}
$$
{\scriptsize for more dynamic illustrations}

\newslide
\Background{horizontal}{Ublue}
\NavButton{\Goto{MagnumOpus}}{\reflectbox{\includegraphics[width=10pt,bb=6 44 198 162]{arrow.eps}}
Back to Main Theorem}
\NavButton{\GotoEnd}{Skip to End \includegraphics[width=10pt,bb=6 44 198 162]{arrow.eps}}

\blob You can incorporate navigation buttons \Build\hb
{\scriptsize(to relive past glories, or gracefully shorten your presentation)}

\Blob\hbox{You can use hyperlinks to external documents}

\Bblob Another pdf document: 
\JumpToPDF{acrguide.pdf}{acrguide.pdf}
{\scriptsize (assuming it's accessible)}

\Bblob Launch another \Launch{inference.xls}{application}\hb
{\scriptsize (assuming you own a file called
inference.xls)}

\Bblob A \Weblink{http://www.utopiatype.com.au/}{webpage}

\newslide
\def\MyHeadXtra{\LinkBorderWidth{.5pt}\LinkTextColour{Uwhite}\GeneralDocInfo{About}}
\NavButton{\GoBack}{Click {\color{Uyellow}here} to go back.}
\NavButton{\ClosePresentation}{Click {\color{Uyellow}here} to close,}
\NavButton{\QuitAcrobat}{or {\color{Uyellow}here} to exit.}
\Background{flyingbuttress}[rgb]{0,.0588,.75294}[rgb]{.184,.376,.941}[rgb]{0,0,.188235}
And eventually we reach\dots\Build
\vfill
\font\Fancy=nu at50pt
\HeadingTextColour{Ugold}
\Heading[center]{\Fancy The End}
\vfill
\end{slide}
\end{document}



