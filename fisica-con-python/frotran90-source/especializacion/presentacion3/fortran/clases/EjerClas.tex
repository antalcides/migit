
\documentclass{article}
%%%%%%%%%%%%%%%%%%%%%%%%%%%%%%%%%%%%%%%%%%%%%%%%%%%%%%%%%%%%%%%%%%%%%%%%%%%%%%%%%%%%%%%%%%%%%%%%%%%%%%%%%%%%%%%%%%%%%%%%%%%%
%TCIDATA{OutputFilter=LATEX.DLL}
%TCIDATA{Version=4.00.0.2312}
%TCIDATA{Created=Wednesday, August 06, 2003 15:28:49}
%TCIDATA{LastRevised=Wednesday, August 06, 2003 16:16:30}
%TCIDATA{<META NAME="GraphicsSave" CONTENT="32">}
%TCIDATA{<META NAME="DocumentShell" CONTENT="Standard LaTeX\Blank - Standard LaTeX Article">}
%TCIDATA{CSTFile=40 LaTeX article.cst}

\newtheorem{theorem}{Theorem}
\newtheorem{acknowledgement}[theorem]{Acknowledgement}
\newtheorem{algorithm}[theorem]{Algorithm}
\newtheorem{axiom}[theorem]{Axiom}
\newtheorem{case}[theorem]{Case}
\newtheorem{claim}[theorem]{Claim}
\newtheorem{conclusion}[theorem]{Conclusion}
\newtheorem{condition}[theorem]{Condition}
\newtheorem{conjecture}[theorem]{Conjecture}
\newtheorem{corollary}[theorem]{Corollary}
\newtheorem{criterion}[theorem]{Criterion}
\newtheorem{definition}[theorem]{Definition}
\newtheorem{example}[theorem]{Example}
\newtheorem{exercise}[theorem]{Exercise}
\newtheorem{lemma}[theorem]{Lemma}
\newtheorem{notation}[theorem]{Notation}
\newtheorem{problem}[theorem]{Problem}
\newtheorem{proposition}[theorem]{Proposition}
\newtheorem{remark}[theorem]{Remark}
\newtheorem{solution}[theorem]{Solution}
\newtheorem{summary}[theorem]{Summary}
\newenvironment{proof}[1][Proof]{\noindent\textbf{#1.} }{\ \rule{0.5em}{0.5em}}
\input{tcilatex}

\begin{document}


\fbox{\emph{6}} Se pueden llenar globos de $2.5g$ de masa con helio, para
formar esferas perfectas de radio $15cm.$\textquestiondown Cu\'{a}ntos de
esos globos debe sugetar un ni\~{n}o de 25kg de masa para despegar de la
superficie de la tierra? Suponga que $\rho _{He}=0.18kg/m^{3}$ y que $\rho
_{aire}=1.29kg/^{3}.$

\bigskip 

\fbox{\emph{7}} Un cubo de hielo cuyo arista mide $20mm$flota en un baso de
agua casi tan fr\'{\i}a como el hielo con una de sus caras paralela a la
superficie del agua. a) \textquestiondown A que distancia por debajo de la
superficie del agua se encuentra la cara inferior del bloque? b) Alcohol et%
\'{\i}lico hecho hielo se vierte cuidadosamente sobre la superficie del agua
para formar una capa de $5mm$ de espesor sobre el agua. Cuando el cubo de
hielo alcanza el equilibrio hidrost\'{a}tico otra vez, \textquestiondown cu%
\'{a}l ser\'{a} la distancia desde la parte superior del agua hasta la cara
inferior del bloque? c) Se vierte alcohol et\'{\i}lico adicional sobre la
superficie del agua hasta que la superficie superior del alcohol coincida
con la superficie superior del cubo de hielo (en equilibrio hidrot\'{a}%
tico). \textquestiondown Cu\'{a}l es el espesor que se requiere del alcohol
et\'{\i}lico? 

\bigskip 

\fbox{\emph{8.}} Un vaso de vidrio tiene $2\times 10^{2}kg$ de masa y
contiene $1\times 10^{-4}m^{3}$ de agua cuando est\'{a} totalmente lleno. Si
se pone a flotar, se le pueden poner $91g$ de granulado de plomo, hasta que
se comienza a hundir, \textquestiondown Cu\'{a}l es densidad del vidrio?

\bigskip 

\fbox{\emph{9.}}\textbf{\ }En un recipiente de forma cil\'{\i}ndrica de \'{a}%
rea seccional $S$ que contiene agua, flota un pedazo de hielo con una bolita
de plomo en su interior. El volumen total del pedazo de hielo junto con la
bolita de plomo es igual a $V$ y sobre el nivel del agua en el recipiente
sobresale $1/20$ de dicho volumen. Que altura desciende el nivel del agua en
el recipiente, una vez que el hielo se haya derretido? Suponga conocidas las
densidades del hielo y del plomo.

\bigskip 

\fbox{\emph{10.}}\textbf{\ }Un bloque de madera flota parcialmente sumergido
en agua. Cuando se coloca sobre \'{e}l una masa de $200g$, su cara inferior
se sumerge $2cm$ m\'{a}s. Halle la arista del cubo de madera. Suponga que $%
\rho _{H_{2}O}=1\times 10^{3}kg/m^{3}.$

\bigskip 

\fbox{\emph{11.}} Un objeto de masa $0.18kg$ pero de densidad desconocida $%
\rho _{o}$ se \textquotedblleft pesa\textquotedblright\ totalmente sumergido
en agua (densidad del agua $\rho _{H_{2}O}=1000kgm^{-3}$), y el peso as\'{\i}
obtenido corresponde a una masa equilibrante de $0.15kg$; al
\textquotedblleft pesarlo\textquotedblright\ de nuevo totalmente sumergido
en un l\'{\i}quido de pensidad desconocida $\rho _{l}$, resulta que se
necesita una masa equilibrante de $0.144kg.$ Determine la densidad del l%
\'{\i}quido $\rho _{l}$ y la densidad del objeto $\rho _{o}.$ \textbf{Nota: }%
\emph{Suponga que el proceso de }\textquotedblleft \emph{pesar}%
\textquotedblright \emph{\ el objeto de densidad desconocida se lleva a cabo
mediate el dispositivo ilustrado en la figura 3.}

\bigskip 

\bigskip 

\bigskip 

\textbf{1.} Considere un gran tanque que contiene agua y que esta abierto en
su extremo superior. El nivel del agua en el tanque se mantiene constante a
una altura $h_{0}$ medida desde la base del mismo ya que en \'{e}l entra
agua proveniente de un grifo situado en la parte superior. Se perforan dos
peque\~{n}os orificios de igual \'{a}rea seccional al mismo lado del tanque
pero a alturas $h_{1}$ y $h_{2}$ medidas tambi\'{e}n a partir de la base del
tanque. Si se quiere que el chorro de agua que sale por los dos orificios
referidos, caiga a la misma distancia $D$ medida a partir de la base del
tanque y la raz\'{o}n entre las alturas $h_{1}$ y $h_{2}$ es de $1/3,$
determine, en t\'{e}rminos de $h_{0}$: \textbf{a) }las alturas $h_{1}$ y $%
h_{2}$ y \textbf{b) }la con que debe entrar el agua en el tanque para que se
cumpla con lo deseado.

\bigskip

\textbf{2. }Dos depositos abiertos muy grandes, $A$ y $F$ (ver figura ),
contienen el mismo l\'{\i}quido. Un tubo horizontal $BCD,$ que tiene un
estrechamiento en $C,$ descarga l\'{\i}quido del fondo del deposito $A,$ y
un tubo vertical $E,$ abierto en el estrechamiento $C,$ se introduce en el l%
\'{\i}quido del deposito $F.$ Suponga que el fluido es ideal y desprecie las
variaciones de la presi\'{o}n atmosf\'{e}rica con la altura. Si la secci\'{o}%
n transversal en $C$ es la mitad que en $D,$ y si $D$ se encuentra a una
distancia $h_{1}$ por debajo del nivel del l\'{\i}quido en $A,$ demuestre
que la altura $h_{2}$ que alcanzar\'{a} el l\'{\i}quido en el tubo $E$ esta
dada por. $h_{2}=3h_{1}.$

\FRAME{dtbphF}{3.5475in}{2.2364in}{0pt}{}{}{grantanque.wmf}{\special%
{language "Scientific Word";type "GRAPHIC";maintain-aspect-ratio
TRUE;display "USEDEF";valid_file "F";width 3.5475in;height 2.2364in;depth
0pt;original-width 7.6112in;original-height 4.7781in;cropleft "0";croptop
"1";cropright "1";cropbottom "0";filename 'grantanque.WMF';file-properties
"XNPEU";}}

\textbf{5. }Agua de mar (de densidad $\rho =1083\,kgm^{-3}$) alcanza en un
deposito una altura de $1.2m.$ El deposito contiene aire comprimido en su
parte superior a la presi\'{o}n manom\'{e}trica de $7.2\times 10^{5}Pa.$ El
tubo horizontal de desag\"{u}e tiene secciones transversales m\'{a}xima y m%
\'{\i}nima de $1.8\times 10^{-3}m^{2}$ y $9\times 10^{-4}m^{2}$
respectivamente y est\'{a} abierto en su extremo derecho como se muestra en
la figura 1. Suponiendo que el \'{a}rea seccional del deposito es mucho
mayor que el \'{a}rea seccional m\'{a}xima del tubo de desag\"{u}e, \textbf{%
a) }Halle el flujo de agua que circula por la secci\'{o}n transversal m\'{\i}%
nima del tubo de desag\"{u}e. \textbf{b) }Hasta que altura $h$ llega el agua
en el tubo abierto? \textbf{Nota: }\emph{La presi\'{o}n atmosf\'{e}rica es }$%
P_{0}=1.013\times 10^{5}Pa.$

\end{document}
