\documentclass[letter]{article}
\usepackage[pdftex]{graphicx}
\usepackage[ansinew]{inputenc}
\usepackage{amsmath}
\usepackage{amsthm}
\usepackage{amsbsy}
\usepackage{amssymb}
\usepackage[spanish]{babel}
\usepackage{float}
\usepackage{fixseminar}
\usepackage[display]{texpower}
%\usepackage[contnav]{pdfslide}
\usepackage[ams]{pdfslide}
\newtheorem{obs}{Observaci{\'o}n}
\newtheorem{cor}{Corolario}
\newtheorem{theo}{Teorema}
\newtheorem{Not}{Nota}
\newtheorem{Def}{Definici{\'o}n}
\newtheorem{lem}{Lema}
\newtheorem{Prop}{Proposici{\'o}n}
\pagestyle{title}
\begin{document}
\orgname{Existencia, Perturbaci{\'o}n y Anulaci{\'o}n de Soluciones
Peri{\'o}dicas en un Sistema de Ecuaciones Diferenciales en el
Espacio}
\title{Departament de Matem{\`a}tiques}
\author{\scalebox{1}[1.0]{\emph{\textbf{ Jorge ~L.~Rodr{\'\i}guez ~C.} } }}
\address{Memoria presentada para optar al t{\'\i}tulo de Doctor\\
{\realfootnotesize {por la UAB }}} \overlay{azur.jpg} \maketitle
\pagedissolve{Wipe /D 2 /Dm /V /M /O} %\overlay{Overlay1.jpg}
\overlay{blanco1.png}
%\sffamily \Large
\color{section0} \headskip=20pt
\section{Introducci{\'o}n}
%\pagedissolve{Wipe /D 2 /Di /V /M /O} \overlay{blanco1.png} \large
%\color{section2}
%\color{section1}
\color{black} {\bf\Large{El Problema}}:\quad {\bf\Large{Origen}}
\\
El problema objeto de nuestra investigaci{\'o}n est{\'a} expuesto en el
libro de J. K. Hale[48] y tiene su origen en una Tesis de Maestr{\'\i}a
presentada en la Facultad de Ingenier{\'\i}a Mec{\'a}nica en la Universidad
de Purdue, cuyo autor es Boyer, R. C., y titulada Sinusoidal
Signal Stabilization (Febrero de 1960).
\\
Se trata de una extinci{\'o}n de oscilaciones de una ecuaci{\'o}n
diferencial  de tercer orden, por medio de la introducci{\'o}n de una
perturbaci{\'o}n peri{\'o}dica de amplitud y frecuencias suficientemente
grandes.
\\
Tal como expone Hale, Boyer considera la ecuaci{\'o}n diferencial de
tercer orden
\begin{equation}\label{Ia5}
\dddot{x}+2\ddot{x}+\dot{x}+Kf(x)=0
\end{equation}
donde $f(x)$ es dada en la Fig \ref{figI1}.
\\
\begin{figure}[h]
\centering
\includegraphics[height=7cm]{fig1.png}
\caption{Gr{\'a}fico de $f(x)$} \label{figI1}
\end{figure}
El sistema \ref{Ia5} para algunos valores de $K$ tiene una
oscilaci{\'o}n auto-excitada la cual es asint{\'o}ticamente estable, y el
problema es tratar de anular esta oscilaci{\'o}n reemplazando $f(x)$
por $f(x+B\sin\omega t) $ y escogiendo $B$ y $\omega$ grandes.
\begin{center}
\begin{tabular}{l|l|l}\hline
\multicolumn{3}{c}{Tabla de valores}\\\hline $K$&$B$&Amplitud de
Oscilaci{\'o}n
\\\hline
2&0&13.2\\
&5&12.2\\
&7&10.0\\
&7.5&8.6\\
&8&0.00\\
&10&0.00\\
4&0&26.4\\
& 5& 26.3\\
&10&25.5\\
&15& 23.0\\
&15.5& 22.0\\
& 16 & 21.0\\
& 17 & 0.00\\
& 20& 0.00
\end{tabular}
\end{center}
Los resultados de Boyer est{\'a}n dados en la tabla con $b=10$, $a=5$
en Figura \ref{figI1}, y $\omega$ escogido al menos veinte veces
la frecuencia de la oscilaci{\'o}n auto excitada de (\ref{Ia5}).
\\
\\
Se debe entender que superpuesta a las oscilaciones que da la
tabla \-existen oscilaciones de amplitud muy peque{\~n}a, de
frecuencia $\omega$. No hemos tenido la oportunidad de consultar
directamente el trabajo de Boyer, por lo cual nos hemos apoyado en
el tratamiento de {\'e}l que Hale en su libro [48], y en el que se
dice "Para explicar este fen{\'o}meno Boyer utiliz{\'o} el m{\'e}todo de las
funciones descriptoras, para analizar el sistema promediado
\begin{equation}\label{Simo81}
\dddot{x}+2\ddot{x}+\dot{x}+Kf_0(x,B)=0
\end{equation}
donde
\begin{equation}\label{Simo7}
f_0(x,B)=\displaystyle{\frac{1}{2\pi}}\int_0^{2\pi}
f(x+B\sin\tau)\quad d\tau\quad"
\end{equation}
El m{\'e}todo de las funciones descriptoras, seg{\'u}n el mismo Hale, es
un procedimiento gr{\'a}fico que no esta justificado rigurosamente.
\begin{figure}[h]
\centering
\includegraphics[height=6cm]{fig2.png}
\caption{Funci{\'o}n Promedio} \label{figura2}
\end{figure}
\\
\\
{\bf\Large{Objetivos}}
\\
Para dar una explicaci{\'o}n al fen{\'o}meno de extinci{\'o}n de oscilaciones
que presenta el ejemplo de Boyer, investigamos en este trabajo una
familia de ecuaciones diferenciales de tercer orden, o
equivalentemente de sistemas din{\'a}micos en dimensi{\'o}n 3, que lo
incluyen. No se ha podido probar que existe una {\'o}rbita peri{\'o}dica
atractora, por lo que nos hemos tenido que conformar con una
{\'o}rbita peri{\'o}dica que pueda estar inmersa en un atractor
oscilatorio m{\'a}s complicado, pero cuyas oscilaciones tienen una
amplitud mayor que cierto valor calculable. Se ha podido
establecer que con una perturbaci{\'o}n peri{\'o}dica de suficientemente
amplitud y frecuencia, todas las soluciones tienden a un entorno
de $\mathbf{0}$ tan peque{\~n}o como se quiera, con lo cual las
oscilaciones, que las hay, tienen una amplitud menor que una
$\epsilon$ predeterminada, y en este sentido podemos decir que
hemos extinguido las oscilaciones originales.
\\
\emph{Dada la ecuaci{\'o}n diferencial
$$\dddot{x}+a\ddot{x}+b\dot{x}+f(x)=0$$
o el Sistema Diferencial equivalente
\\
\begin{equation}\label{obt1}
\begin{array}{lcl}
\dot{x}&=& y\\&&\\ \dot{y}&=&z\\&&\\
\dot{z}&=&-az-by-f(x)
\end{array}
\end{equation}
con $a>0$, $b>0$, $f$ Lipschitz, no nula,  no decreciente, $f'$
continua en un entorno de {\bf 0}, $f'(0)=c>0$, existe $C>0$ tal
que $|f(x)|<C$ para toda $x$, $\frac{|f(x)|}{|x|}<C$ para todo
$x\in (0,1)$.\\
\begin{figure}[h]
\centering
\includegraphics[height=6cm]{fig3.png}
\caption{Funci{\'o}n Lipschitz} \label{figura3}
\end{figure}
Queremos ver que
\\
\begin{enumerate}
\item[a)]Si $ab>c>0$ y $a>0$ y $f'(x)<ab$ para toda $x$, entonces
la soluci{\'o}n trivial es globalmente asint{\'o}ticamente estable
\item[b)]Si $ab<c$ y $a^2> 4b$, entonces existe un atractor
oscilatorio de gran amplitud que contiene (o consiste de) una
{\'o}rbita peri{\'o}dica. En el caso que $a^2\leq 4b$ tenemos que todas
las {\'o}rbitas excepto las dos que tienden al origen oscilan, pero no
podemos asegurar que existe una {\'o}rbita peri{\'o}dica.
\end{enumerate}}
Igualmente se busca  estudiar, el  efecto que causa para el caso
b) el introducir una funci{\'o}n  sinusoidal gran amplitud y  de alta
frecuencia, de manera que $f(x)$ queda sustituida por
$f(x+B\sin\omega t)$.
\\
Nuestra meta es demostrar la siguiente afirmaci{\'o}n:
\\
\emph{En el caso b), si $f(-x)=-f(x)$ y si  B y la frecuencia
$\omega$ son suficientemente grandes, las soluciones del sistema
perturbado tienden a una vecindad del origen O tan peque{\~n}a como
querramos. En este caso el sistema perturbado ya no mantiene las
oscilaciones de amplitud re\-lativamente grande que presenta el
sistema aut{\'o}nomo. En este sentido podemos decir que se han
extinguido estas {\'u}ltimas oscilaciones. }
\\
\\
{\bf\Large{Antecedentes}}
\\
Se conocen numerosos trabajos acerca de la existencia de
soluciones peri{\'o}dicas para la ecuaci{\'o}n de tercer grado no
aut{\'o}noma, por ejemplo J. O. Ezeilo [35], [36], Rolf Reissig [81],
[82], [83] entre otros; mientras en el caso aut{\'o}nomo encontramos
poca investigaci{\'o}n sobre este tema: Rauch, L. L. [77], J. O.
Ezeilo [28].
\\
\\
En  la segunda parte de este trabajo de investigaci{\'o}n, a pesar de
las ideas interesantes propuestas por Jack Hale, no  fue posible
utilizar el Teorema 16.2 de [48] debido a la imposibilidad de
probar la existencia de ciclo l{\'\i}mite asint{\'o}ticamente estable, sino
que nos tenemos que conformar con un comportamiento oscilatorio de
gran amplitud, con un atractor que contiene al menos una {\'o}rbita
peri{\'o}dica.
\\
\\
{\bf\Large{Resumen}}
\\
\\
{\bf Cap{\'\i}tulo 1}.  Se establece el tipo del punto de equilibrio en
t{\'e}rminos de los par{\'a}metros $a$, $b$ y $c$, y se reduce  el sistema
a una forma normal mediante un cambio lineal de coordenadas.
\\
\\
{\bf Cap{\'\i}tulo 2}. Se prueba que si $f'(x)<ab$ para toda $x$,
entonces las soluciones de \ref{obt1} tienden a 0.
\\
\\
{\bf Cap{\'\i}tulo 3}. Teniendo en cuenta las condiciones $ab<c$,
construimos una Esfera Atractora definida por  $
V(x,y,z)+\lambda(z+ay+bx)-\kappa=0$ y que es cortada en un solo
punto por cada rayo que emana de 0, ampliando el resultado de
Ezeilo[33] al caso $ab<c$.La dificultad radica en hallar una
funci{\'o}n $V(x,y,z)$ ya que no se cumple el criterio de Routh
Hurwitz. Para superar esta dificultad se parametriza el sistema
con $\mu\in[0,1]$ y as{\'\i} se obtiene un sistema que depende de este
par{\'a}metro que bajo ciertas condiciones, satisface las condiciones
de Routh Hurwitz. De esta manera se puede entrar a construir una
funci{\'o}n de Lyapunov para el sistema lineal resultante y por
analog{\'\i}a se construye una funci{\'o}n $V$ definida positiva con
$\dot{V}\leq 0$ en cierta regi{\'o}n.  En esta forma  se termina la
construcci{\'o}n de la Esfera Atractora, que adem{\'a}s cumple con la
condici{\'o}n de que cualquier semirrecta que parte del origen la
corta en un solo punto.
\\
\\
{\bf Cap{\'\i}tulo 4}. En el caso que $a^2>4b$, a la regi{\'o}n limitada
por la esfera atractora \-construida en el cap{\'\i}tulo 3  se le quita
una regi{\'o}n c{\'o}nica-cil{\'\i}ndrica para formar as{\'\i} una regi{\'o}n
tridimensional cerrada topol{\'o}gicamente equivalente al toro s{\'o}lido.
El campo vectorial tiene direcci{\'o}n en todos los puntos de la
frontera hacia el interior de la regi{\'o}n con res\-tricciones
convenientes en los par{\'a}metros. Las {\'o}rbitas del sistema contenidas
en el toro giran alrededor del agujero del mismo definiendo una
aplicaci{\'o}n continua de cada secci{\'o}n meridiana en ella misma.
Aplicando el teorema del punto fijo de Brouwer se establece la
existencia de una {\'o}rbita cerrada alrededor del agujero del toro.
Esto corresponde a una soluci{\'o}n peri{\'o}dica. Las dem{\'a}s trayectorias
tienden a ella o bien a un atractor que la contiene. Cuando
$a^2\leq 4b$ no podemos establecer la regi{\'o}n toroidal ni, por
tanto, la existencia de una soluci{\'o}n peri{\'o}dica alrededor del
cilindro $\mathbf{C}$. Sin embargo todas las {\'o}rbitas, excepto las
dos que tienden a 0, tienen un movimiento oscilatorio de "gran"\-
amplitud. Ponemos "gran"\- para distinguirlas de las oscilaciones
que aparecen al perturbar el sistema, y que podemos hacer tan
peque{\~n}a como querramos. De hecho la amplitud de estas oscilaciones
tendr{\'a} una cota inferior dada por los par{\'a}metros de sistema, tal
como calcularemos m{\'a}s adelante.
\\
\\
{\bf Cap{\'\i}tulo 5}. Teniendo en cuenta que: $a>0$,\quad $b>0$, \quad
$f$ Lipschitz, no decreciente, con derivada continua en  0 y
$f'(0)=c>0$,\quad $|f(x)|<C$ para toda $x$, se prueba la acotaci{\'o}n
de las soluciones de la ecuaci{\'o}n diferencial perturbada
$$\dddot{x}+a\ddot{x}+b\dot{x}+f(x+B\sin(\omega t))=0$$
{\bf Cap{\'\i}tulo 6}.  Considerando el sistema del cap{\'\i}tulo 4, en que
el sistema  tiene un comportamiento oscilatorio, se demuestra en
el caso que $f(-x)=-f(x)$, que al reemplazar la funci{\'o}n $f(x)$ por
$f(x+B\sin\omega t)$, y para va\-lores de $B$ y $\omega$
suficientemente grande el sistema no tiene movimiento oscilatorio
de gran amplitud. De hecho todas las soluciones tienden a una
vecindad del origen tan peque{\~n}a como se quiera.
\\
\\
Primero se elige $B$ suficientemente grande para que el sistema
promediado tenga el origen globalmente asint{\'o}ticamente estable, y
despu{\'e}s se toma $\omega$ suficientemente grande para que el
sistema perturbado este suficientemente pr{\'o}ximo al sistema
promediado.
\\
\\
En cada caso hemos considerado las condiciones de $f$ m{\'a}s
generales que hacen cierto el Teorema correspondiente, lo cual ha
producido cierta heterogeneidad en los enunciados. En el Teorema
final estamos en la intersecci{\'o}n de todas las hip{\'o}tesis.
\section{Contenido}
\sffamily\Large
\begin{itemize}
\item \emph{Punto de Equilibrio.}\pause \item \emph{El Origen como
Atractor Global.} \pause \item \emph{Esfera Atractora.}\pause
\item\emph{Existencia de Oscilaciones.}\pause \item\emph{Acotaci{\'o}n
de las soluciones de los sistemas
\newline perturbados.}\pause \item\emph{Eliminaci{\'o}n de las
grandes Oscilaciones.}\pause
\end{itemize}
\headskip=20pt
\section{1 Punto de Equilibrio}
\pagedissolve{Wipe /D 2 /Di /V /M /O} \overlay{blanco1.png} \large
%\color{section2}
%\color{section1}
\color{black} Sea  la ecuaci{\'o}n diferencial
\begin{equation}\label{se1}
\frac{d^3x}{dt^3}+a\frac{d^2x}{dt^2}+b\frac{dx}{dt}+f(x)=0
\end{equation}
Consideraremos en lugar de la ecuaci{\'o}n (\ref{se1}), el sistema de
ecuaciones equivalente
\begin{equation}\label{se2}
\left\{\begin{array}{lcl}
\dot{x}&=&y\\
\dot{y}&=&z\\
\dot{z}&=&-az-by-f(x)
\end{array}\right.
\end{equation}
donde la funci{\'o}n $f$ es Lipschitz, $f'$ existe y es continua en un
entorno de 0 y $f'(0)=c>0$, $xf(x)>0$ si $x\neq0$, $a>0$, $b>0$.\\
Es claro que $(0,0,0)$ es el {\'u}nico punto de equilibrio del
sistema.
\newpage
La forma normal del sistema nos queda:
\begin{equation}\label{se7}
\begin{array}{lcl}
\dot{x_1}&=&\lambda_1
x_1-\displaystyle{\frac{1}{[(\alpha-\lambda_1)^2+\beta^2]}}[f(x_1+x_3)-c(x_1+x_3)]\\&&
\\
\dot{x_2}&=&\alpha x_2-\beta
x_3-\displaystyle{\frac{\alpha-\lambda_1}{\beta[(\alpha-\lambda_1)^2+\beta^2]}}[f(x_1+x_3)-c(x_1+x_3)]
\\ &&
\\
\dot{x_3}&=&\beta x_2+\alpha
x_3+\displaystyle{\frac{1}{[(\alpha-\lambda_1)^2+\beta^2]}}[f(x_1+x_3)-c(x_1+x_3)]
\end{array}
\end{equation}
{\bf\large{Diagramas de cambio de los valores propios al crecer
$c$}}
\begin{figure}[h]
\centering
\includegraphics[height=4cm]{fig11.png}
\end{figure}
\begin{figure}[h]
\centering
\includegraphics[height=5cm]{fig12.png}
\label{figura2}
\end{figure}
\begin{figure}[h]
\centering
\includegraphics[height=5cm]{fig13.png}
\label{figura3}
\end{figure}
%\section{}
%\pagedissolve{Wipe /D 3 /Di /V /M /O} \overlay{blanco1.png} \large
% En coordenadas cil{\'\i}ndricas el sistema nos queda:
%\begin{equation}\label{se8}
%\begin{array}{lcl}
%\dot{x_1}&=&\lambda_1
%x_1-\displaystyle{\frac{1}{[(\alpha-\lambda_1)^2+\beta^2]}}S(x_1,r,\theta)\\&&
%\\
%\dot{r}&=&\alpha
%r+\displaystyle{\frac{1}{[(\alpha-\lambda_1)^2+\beta^2]}}\Big[\sin\theta-\displaystyle{
%\frac{\alpha-\lambda_1}{\beta}}\cos\theta\Big]S(x_1,r,\theta)
%\\ &&
%\\
%\dot{\theta}&=&\beta
%+\displaystyle{\frac{1}{r[(\alpha-\lambda_1)^2+\beta^2]}}\Big[\cos\theta+\displaystyle{
%\frac{\alpha-\lambda_1}{\beta}}\sin\theta\Big]S(x_1,r,\theta)
%\end{array}
%\end{equation}
%donde $S(x_1,r,\theta)=[f(x_1+r\sin\theta)-c(x_1+r\sin\theta)]$
\headskip=20pt
\section{2 El Origen como Atractor Global}
\overlay{blanco1.png} \color{black} \pagedissolve{Wipe /D 2 /Di /H
/M /O} Estudiamos  la estabilidad  asint{\'o}tica global de la
ecuaci{\'o}n no lineal de tercer orden (\ref{se1}), donde $f(x)$ es
una funci{\'o}n que cumple las siguientes condiciones:
\\
\begin{equation}\label{Condf1}
\begin{array}{lll}
a>0& b>0& f\hspace{0.3 cm}\textrm{derivable.}\\
ab-f'(x)>0&&\\
xf(x)>0& \textrm{para}\quad\quad x\neq0&\\
\int_0^{\pm \infty}f(x)\, dx&=F(\pm\infty)=+\infty
\end{array}
\end{equation}  $a>0$, $ab>f'(x)>0$ para todo $x$, para ello se
construye una funci{\'o}n de Lyapunov para el sistema lineal y por
analog{\'\i}a se obtiene una funci{\'o}n de Lyapunov para el sistema no
lineal
%\begin{theo}\label{SAT1}La soluci{\'o}n trivial del sistema \ref{se2} es
globalmente
%asint{\'o}ticamente estable si las condiciones anteriores se cumplen.
%\end{theo}
%En analog{\'\i}a con el sistema lineal, obtenemos la funci{\'o}n de
%Lyapunov
\begin{equation}\label{SAe11}
V(x,y,z)=a F(x)+f(x)y+\Phi(y)+a^2G(y)+ayz+H(z)
\end{equation}
donde
$$F(x)=\int_0^xf(x)\,
dx\hspace{0.8cm}G(y)=\displaystyle{\frac{1}{2}}y^2\hspace{0.8cm}
\Phi(y)=bG(y)\hspace{0.8cm}H(z)=\displaystyle{\frac{1}{2}}z^2$$
\begin{displaymath}
\begin{array}{lcl}
\dot{V}&=&a
F'(x)\displaystyle{\frac{dx}{dt}}+f'(x)y\displaystyle{\frac{dx}{dt}}+f(x)
\displaystyle{\frac{dy}{dt}}+\Phi'(y)\displaystyle{\frac{dy}{dt}}+
a^2G'(y)\displaystyle{\frac{dy}{dt}}+\\&&\\
&&+a\displaystyle{\frac{dy}{dt}}z+ay\displaystyle{\frac{dz}{dt}}+
z\displaystyle{\frac{dz}{dt}}\\&&\\
&=&af(x)y+f'(x)y^2+f(x)z+byz+a^2yz+az^2-\\&&\\
&&-a^2yz-aby^2-af(x)y-az^2-byz-f(x)z\\&&\\
\dot{V}&=&-(ab-f'(x))y^2
\end{array}
\end{displaymath}
\\
Considerando el dominio acotado:\\
$$D_l=\{(x,y,z)|\hspace{0.7cm}0\leq V<l,\hspace{0.6cm}|x|<N\}$$
Se prueba que $\dot{V}\leq 0$ en la frontera de $D_l$, entonces el
campo vectorial del sistema (\ref{se2}) va de afuera hacia adentro
\headskip=20pt
\section{3 Esfera Atractora}
\pagedissolve{Wipe /D 2 /Dm /V /M /O} \overlay{blanco1.png}
%\sffamily \Large
\color{black} Consideramos el sistema
\begin{equation}\label{Be2}
\begin{array}{lcl}
\displaystyle{\frac{dx}{dt}}&=&y,\\&&\\
\displaystyle{\frac{dy}{dt}}&=&z,\\&&\\
\displaystyle{\frac{dz}{dt}}&=&-az-by-f(x),
\end{array}
\end{equation}
$f$ del tipo considerado en el Teorema \ref{tsimo1}, con la
posibilidad de que el origen sea inestable, es decir, con
$f'(0)=c>ab$, que corresponde a una silla-foco con una variedad
asint{\'o}tica de dimensi{\'o}n 2 inestable. Mostraremos que en este caso
tambi{\'e}n existe una superficie esf{\'e}rica positivamente invariante.
\begin{theo}\label{tsimo1}Suponga que $a>0$,\hspace{0.5cm}
$b>0$,\hspace{0.5cm}$f$ es Lipschitz, no decreciente, con derivada
continua en 0 y \, $f'(0)=c>0$,\, existe un $C>0$ tal que para
toda $x$\quad $|f(x)|<C$ y $\displaystyle{\frac{|f(x)|}{|x|}}<C$
para $0<|x|<1$.
\\
\\
Entonces, existe una superficie esf{\'e}rica  ${\bf{S}}$ en el espacio
que cumple la propiedad $\mathbf{P_1}$ y tal que toda trayectoria
del sistema aut{\'o}nomo la cruza hacia el interior, para $t$
suficientemente grande.
\end{theo}
Buscaremos superficies de la forma
\begin{equation}\label{Be3}
\Gamma(\lambda,\kappa)\equiv V(x,y,z)+\lambda(z+ay+bx)-\kappa=0
\end{equation}
donde $\lambda$, y, $\kappa$ son constantes finitas.
\\
\\
Consideremos la familia ($E^*_{\mu}$) de ecuaciones diferenciales
\\
$$(E^*_{\mu})\hspace{2cm}\dddot{x}+\alpha(\mu)\ddot{x}+\beta(\mu)\dot{x}+\gamma(\mu)x+\mu^2f(x)=0,\hspace{0.6cm}\mu\in[0,1]$$
\\
donde
\begin{equation}\label{Be4}
\alpha(\mu)=\mu a+(1-\mu)a_1,\quad\beta(\mu)=\mu
b+(1-\mu)b_1,\quad\gamma(\mu)=(1-\mu)^2c_1.
\end{equation}
\\
Los n{\'u}meros $a_1$, $b_1$, $c_1$ son constantes positivas que
satisfacen:
\begin{equation}\label{Be5}
0<c_1<a_1 b_1
\end{equation}
Ahora consideraremos el Sistema Lineal, para el cual construiremos
una funci{\'o}n de Lyapunov
\begin{equation}\label{Be7}
\begin{array}{lcl}
\displaystyle{\frac{dx}{dt}}&=&y\\&&\\
\displaystyle{\frac{dy}{dt}}&=&z\\&&\\\displaystyle{\frac{dz}{dt}}&=&-\alpha
z-\beta y-\gamma  x
\end{array}
\end{equation}
\\
donde $\alpha$, $\beta$, y $\gamma $ son  constantes que
satisfacen las condiciones
\begin{equation}\label{Be10}
\alpha>0\hspace{0.7cm}\alpha\beta>\gamma>0.
\end{equation}
\\
De la ecuaci{\'o}n (\ref{Be10}) existe una constante $k$ tal que
\begin{equation}\label{Be11}
\displaystyle{\frac{1}{\alpha}}<k<\displaystyle{\frac{\beta}{\gamma}}
\end{equation}
\\
Sea $W$ una funci{\'o}n definida positiva dada por
\\
$$W=k_1(A x+B y)^2+k_2(D y+E z)^2+k_3 y^2 $$
\\
Busquemos las constantes positivas $k_1$, $ k_2$, $ k_3$, $A$,
$B$, $D$, y, $E$ de manera que
\begin{equation}\label{Be13}
\begin{array}{lcl}
\dot{W}&=&-U=-(\beta-\gamma k)y^2-(\alpha k-1)z^2
\end{array}
\end{equation}
De este modo obtenemos la funci{\'o}n
\begin{equation}\label{Be15}
\begin{array}{lcl}
2V(x,y,z;\mu)&=&2\mu^2\int_0^xf(\xi)\,d\xi+\gamma x^2+(\alpha +k
\beta )y^2+k z^2+2yz+\\&+&2\gamma( k yx+exz)
\end{array}
\end{equation}
\begin{lem}\label{BL1}Sean las funciones $\alpha$, $\beta$, $\gamma$
definidas por (\ref{Be4}), donde los n{\'u}meros $a_1$, $b_1$, $c_1$
satisfacen (\ref{Be5}) y sea $k$ una constante independiente de
$\mu$ tal que
\begin{equation}\label{Be16}
\displaystyle{\frac{1}{\alpha}}<k<\displaystyle{\frac{\beta}{\gamma}}\hspace{1cm}\textrm{para
todo }\mu\in[0,1]
\end{equation}
(por (\ref{Be5}) tal n{\'u}mero siempre puede ser escogido); Entonces
existen n{\'u}meros positivos $\epsilon$, $\lambda_1$ tal que para
$e=\epsilon$, $l=\lambda_1$ la forma cuadr{\'a}tica
\begin{displaymath}
\begin{array}{lcl}
\Phi(x,y,z;\mu)&=&\gamma x^2+k(\beta y^2+z^2)+\alpha
y^2+2yz+2\gamma(exz+kxy)-\\&&\\&&-l(\gamma x^2+y^2+z^2),
\mu\in[0,1]
\end{array}
\end{displaymath}
\\
es semidefinida positiva en $[0,1]$.
\end{lem}
\begin{lem}$V(x,y,z)=V(x,y,z;1)$ tiende a infinito con $(x,y,z)$
\end{lem}
\begin{proof}[{\bf Demostraci{\'o}n}.]
Por el Lema \ref{BL1} tenemos la desigualdad
\\
$$2V-2 F(x)-\lambda_1(y^2+z^2)=\Phi(x,y,z;1)\geq 0 $$
lo cual nos produce
$$2 F(x)+\lambda_1(y^2+z^2)\leq 2V(x,y,z)$$
\\
Por lo tanto
\begin{equation}\label{Be17}
0\leq 2F(x)+\lambda_1(y^2+z^2)\leq 2V(x,y,z),
\end{equation}
Dado que $F(x)>0$, para todo $x\neq0$ y $F(x)\rightarrow\infty$
cuando $|x|\rightarrow\infty$, $V(x,y,z)$ es definido
positivamente y
$$\lim_{(x,y,z)\rightarrow\infty}V(x,y,z)=\infty .$$
\end{proof}
{\bf\Large{Propiedades de la Superficie }}
\begin{lem}\label{BL2} Existe $D_0(\lambda)\geq 0$ tal que si $\kappa\geq
D_0(\lambda)$, entonces la superficie
${\bf{S}}={\bf{S}}(\lambda,\kappa)$ definida por (\ref{Be3}) tiene
la propiedad $\mathbf{P_1}$.
\end{lem}
Combinando los resultados tenemos que
\begin{equation}\label{Be18}
\left.\begin{array}{lcl} \Gamma&\leq
&\displaystyle{\frac{1}{2}}D_1(x^2+y^2+z^2)+\lambda
(z+ay+bx)-\kappa
\\
\Gamma&\geq &F(x)+
\displaystyle{\frac{\lambda_1}{2}}(y^2+z^2)+\lambda
(z+ay+bx)-\kappa
\end{array}\right\}
\end{equation}
Entonces,
\begin{equation}\label{Be20}
\triangle_1\subset \triangle_2,\hspace{1.5cm}\mathbf{S}\subset
(\triangle_2-\triangle_1)
\end{equation}
Y considerando
$$\Phi(r)\equiv V(lr,mr,nr)+\lambda r(n+am+bl)-\kappa=0$$
probamos que la recta que parte desde el origen interseca a ${\bf
S}$ en un solo punto.
\\
{\bf\Large{Las Superficies ${\bf S}^{+}$, $ {\bf S}^{-}$}}
\\
Sea $\pi$ el plano
$$z+ay+bx=0.$$
Como $\pi$ pasa a trav{\'e}s del origen es claro que $\pi$ interseca
la superficie ${\bf S}(0,\kappa)$, esto es, la superficie
$$V(x,y,z)=\kappa$$ en un punto real, y, como ${\bf S}(0,\kappa)$ tiene
la propiedad $\mathbf{P_1}$, estos puntos evidentemente est{\'a}n
sobre una curva de Jordan $J_0$. Dado cualquier punto
$(\xi,\eta,\zeta)$ el cual satisface
$$V(\xi,\eta,\zeta)-\kappa=0,\hspace{1.5cm}\zeta+a\eta+b\xi=0,$$
necesariamente $\Gamma(\xi,\eta,\zeta,\lambda,\kappa)=0$ para
todos los valores de $\lambda$, es claro que toda superficie ${\bf
S}(\lambda,\kappa)$ pasa a trav{\'e}s de $J_0$. En lo que sigue aqu{\'\i},
dada cualquier superficie $S(\lambda,\kappa)$ acostumbraremos a
denotar ${\bf S}^+(\lambda,\kappa)$ al conjunto de todos los
puntos de ${\bf S}(\lambda,\kappa)$ los cuales est{\'a}n en o sobre el
plano $\pi$, y ${\bf S}^-(\lambda,\kappa)$ denota el conjunto de
todos los puntos de ${\bf S}(\lambda,\kappa)$ que est{\'a}n debajo o
en $\pi$.
\\
\begin{lem}\label{BL4}Si $\lambda_1$, $\lambda_2$ son dos valores de
$\lambda$, entonces la superficie $${\bf
S}^+(\lambda_1,\kappa)\bigcup {\bf S}^{-}(\lambda_2,\kappa)$$
tiene la propiedad $\mathbf{P_1}$.
\end{lem}
El resultado es una consecuencia del Lema \ref{BL2} y del hecho
que ambas ${\bf S}(\lambda_1,\kappa)$ y ${\bf
S}(\lambda_2,\kappa)$ pasan a trav{\'e}s de
$J_0$.\\
Igualmente mediante c{\'a}lculos elementales obte\-nemos que
\begin{equation}\label{Be26}
\dot{V}\leq -D_4(y^2+z^2)+D_5(|y|+|z|)
\end{equation}
\begin{lem}\label{BL5}Existe $D_6=D_6(\lambda_0)$ tal que si $\kappa\geq
D_6$ entonces
$$\dot{\Gamma}\equiv\displaystyle{\frac{d}{dt}}\Gamma(x,y,z,\lambda_0)\leq0$$
en cualquier punto $T(x,y,z)$ de  $E_3$ en  cualquier trayectoria
$\tau$ del sistema de ecuaciones que se encuentra en ${\bf
S}^+(\lambda_0,\kappa)$.
\end{lem}
%\begin{proof}[{\bf Demostraci{\'o}n}.]
$$\dot{\Gamma}=\dot{V}+\lambda_0(-f(x))$$
Luego
\begin{equation}\label{esfera}
\dot{\Gamma}\leq -D_4(y^2+z^2)+D_5(|y|+|z|)+\lambda_0(-f(x))
\end{equation}
\begin{figure}[h]
\centering
\includegraphics[height=6 cm]{sigma2.png}
\caption{ $\sigma^+$} \label{sfig30}
\end{figure}
\\
%Sea $\sigma^+$ el conjunto de todos los puntos de la esfera
%$$D_1(x^2+y^2+z^2)+2\lambda_0(z+ay+bx)-2\kappa=0$$
%y escojamos una constante $D_7$ tal que cada una de las
%condiciones se satisfacen para $\kappa\geq D_7$:
%\\
%(I) $\sigma^+$ interseca al plano
%$x=\displaystyle{\frac{8}{\delta}} $
%\\
%\\
%(II) $\max(|y|,|z|)>8\displaystyle{\frac{D_5}{D_4}}$, en todos los
%puntos $(x,y,z)$ de $\sigma^+$ los cuales permanecen debajo del
%plano $x=D_8=\displaystyle{\frac{8}{\delta}}$
%\\
%\\
%(III) No existe ning{\'u}n de intersecci{\'o}n entre $\sigma^+$ y el
%paraboloide
%  $$D_4(y^2+z^2)+2\lambda_0\delta x=0$$
%Dado que todo punto de $\mathbf{S}(\lambda_0,\kappa)$ est{\'a}n fuera
%de la esfera $\sigma$, se sigue que las condiciones (I), (II) y
%(III) tambi{\'e}n se cumplen para ${\bf S}^+(\lambda_0,\kappa)$. Si
%$\kappa\geq D_7$ cada punto $(x,y,z)$ de ${\bf
%S}^+(\lambda_0,\kappa)$ queda en una de las siguientes regiones:
%\begin{equation}\label{r1}
%  x\geq D_8
%\end{equation}
%\begin{equation}\label{r2}
%  0\leq x\leq D_8\quad \max(|y|,|z|)>8\displaystyle{\frac{D_5}{D_4}}
%\end{equation}
%\begin{equation}\label{r3}
%\max(|y|,|z|)>8\displaystyle{\frac{D_5}{D_4}}\quad
%D_4(y^2+z^2)+2\lambda_0\delta x>0
%\end{equation}
%probamos que
%\begin{center}
%$\dot{\Gamma}<0 $
%\end{center}
%\end{proof}
\begin{theo}\label{BT1}
Existe una superficie ${\bf S}$ en el espacio que cumple la
propiedad $\mathbf{P_1}$ y tal que toda trayectoria del sistema de
ecuaciones  cruza solo hacia el interior
\end{theo}
Considerando $D_{7}=\max\Big(D_0(\lambda_0),D_6,D_5\Big)$ y  la
superficie ${\bf S}$ definida por
\begin{equation}\label{Be35}
{\bf S}={\bf S}^+\Big(\lambda_0,D_{7}\Big)\bigcup {\bf
S}^-\Big(-\lambda_0,D_7\Big)
\end{equation}
donde $D_0$, $D_6$, $D_{5}$ son las constantes que aparecen en
Lemas anteriores. Por lo tanto, ${\bf S}$ tiene la propiedad
$\mathbf{P_1}$ y adem{\'a}s  las trayectorias del sistema de
ecuaciones cruzan a ${\bf S}$ hacia el interior.
\section{4 Existencia de Oscilaciones}
\pagedissolve{Wipe /D 2 /Di /H /M /O} \overlay{blanco1.png}
\color{black}Investigaremos acerca de la existencia de
oscilaciones en la ecuaci{\'o}n diferencial (\ref{se1}), con
$a>0$\quad $b>0$\quad $f$ Lipschitz, no decreciente\quad
$f(0)=0$,\quad $f$ con derivada continua en 0 y $f'(0)=c>ab$,
 existe $C$ tal que para todo $x$ \quad $|f(x)|<C$ y
$\displaystyle{\frac{|f(x)|}{|x|}}<C$ para $0<|x|<1$.
\\
\\
La regi{\'o}n c{\'o}nica $\mathbf{K}$ que quitaremos a la
esfera del capitulo 3, tiene por fronteras el cono definido en el octante\\
$\{ x>0, y<0, z>0\}$ por las
superficies \\
$\{x=0|\quad z>2\frac{by}{a} \}$, $\{y=0\}$, $\{z=0|\quad
y>f(x)\}$, \\
$\{y>-\displaystyle{\frac{f(x)}{b}}\}$, y,
$\{by+f(x)+\displaystyle{\frac{az}{2}}=0\}$ y el cono an{\'a}logo para
el octante sim{\'e}trico respecto a {\bf 0}.
\begin{lem}\label{lcono2}
La regi{\'o}n c{\'o}nica $\mathbf{K}$ es negativamente invariante si
$\displaystyle{\frac{a^2}{4}}-b>0$.
\end{lem}
\begin{proof}[{\bf Demostraci{\'o}n}.]
En el octante $x>0$,  $y<0$, y $z>0$ el campo $(y,z,-f(x)-by-az)$
para $y>-\displaystyle{\frac{f(x)}{b}}$.
\\
\\
En el plano $x=0$ fluye hacia $x$ negativa.\\
\\
En el plano $y=0$ fluye hacia $y$ positiva
\\
\\
En el plano $z=0$ fluye hacia $z$ negativa.
\\
\\
Sea $F(x,y,z)=-by-f(x)-\displaystyle{\frac{az}{2}}=0$
\\
El vector normal en la parte superior de la superficie viene dado
por $X_N=\nabla F(x,y,z)=(-f'(x),-b,-\displaystyle{\frac{a}{2}})$,
luego el producto interno\\
\begin{displaymath}
\begin{array}{lcl}
\Big(X_N,\dot{X}\Big)&=&-f'(x)y-bz+\displaystyle{\frac{a}{2}}(az+by+f(x))\\
                     &=&-f'(x)y+\big(\displaystyle{\frac{a^2}{4}}-b\big)z>0
\end{array}
\end{displaymath}
Por lo tanto  las {\'o}rbitas cruzan al cono de adentro hacia afuera.
\\
\end{proof}
\begin{figure}[h]
\centering
\includegraphics[height=8 cm]{Cono.png}
\caption{Superficie C{\'o}nica} \label{Sc1}
\end{figure}
\begin{lem}\label{lcilindro}
Sea la ecuaci{\'o}n del cilindro
\\
\begin{displaymath}
\begin{array}{lcl}
0<r^2&=&x_2^2+x_3^2
\end{array}
\end{displaymath}
\\
entonces  las soluciones fluyen de adentro hacia fuera del
cilindro en aquellos  puntos en que $|f(x)-cx|<Ar$, donde $A$ es
una constante que depende de $a$, $b$ y $c$.
\end{lem}
\begin{proof}[{\bf Demostraci{\'o}n}.]
Del sistema (\ref{se2}) expresado en coordenadas cil{\'\i}ndricas
tenemos que si $c>ab$,
$$r'=\alpha
r+\displaystyle{\frac{f(x)-cx}{(\alpha-\lambda_1)^2+\beta^2}}\Big(\sin\theta-
\displaystyle{\frac{\alpha-\lambda_1}{\beta}}\cos\theta\Big)$$
para que $r'$ sea mayor  que cero basta con que
$$|f(x)-cx|<\alpha\beta^2\sqrt{1+\gamma^2}r=Ar$$
\end{proof}
Combinando los Lemas \ref{lcilindro}, y, \ref{lcono2}  obtenemos.
\begin{lem}\label{linvariante}
El toro $\mathbf{T}$ es positivamente invariante si $a^2>4b$ y si
en los puntos del cilindro $\mathbf{C}$ fuera de $\mathbf{K}$ se
cumple
$$|f(x)-cx|<Ar$$
para $x<r+B(r)$.
\\
\\
Aqu{\'\i} $A=\alpha\beta^2\sqrt{1+\gamma^2}$,
$\gamma=\displaystyle{\frac{\alpha-\lambda_1}{\beta}}$ y $B(r)$
nos da el m{\'a}ximo valor de $x$ para los puntos de $\mathbf{C}$
fuera de $\mathbf{K}$.
\\
Notamos que $A<c-ab=2\alpha(1+\gamma^2)\beta^2$, como debe ser.
\\
En particular las condiciones del lema se cumplen si $r$ es
suficientemente peque{\~n}o, ya que
$$\displaystyle{\frac{|f(x)-cx|}{r}}<\displaystyle{\frac{|f(x)-cx|}{|x|}}\rightarrow
f'(0)-c=0$$ cuando $r\rightarrow0$.
\end{lem}
{\bf\large{Comportamiento Oscilatorio y Soluci{\'o}n Peri{\'o}dica}}
\\
Debido a la propiedad de $\mathbf{S}$ de ser cortada en  solo un
punto por las semirrectas que parten del origen ( Capitulo 3), la
secci{\'o}n meridiana $\Upsilon=\{(x,y,z)\, |\quad x>0\quad z<0 \quad
y=0\}\bigcap\mathbf{T}$ es ce\-rrada y simplemente conexa, es
decir homeomorfa a un disco.
\begin{lem}\label{Transcont}
Si $a^2>4b$ el flujo del sistema (\ref{se2}), forma una aplicaci{\'o}n
continua de la $\Upsilon$-secci{\'o}n-transversal cerrada y
simplemente-conexa del Toro invariante en s{\'\i} mismo.
\end{lem}
\begin{proof}[{\bf Demostraci{\'o}n}.]
Es bastante claro que si empezamos en $\Upsilon$ pasamos al
octante $x>0$, $y<0$, $z<0$. Ah{\'\i} te\-nemos que  $x'<0$, $y'<0$,
$z'>0$. Por lo tanto llegar{\'a} un momento en que o bien\\
i) $x=0$ o bien ii) $z=0$.
\\
\\
En el primer caso, i), pasamos al octante $x<0$, $y<0$, $z<0$,
donde $x'<0$, $y'<0$, $z'>o$ (esto {\'u}ltimo porque $-f(x)-by-az>0$).
Llegar{\'a} un momento en que cruzaremos $z=0$, con lo que se har{\'a}
$y'>0$, y posteriormente cruzaremos $f(x)+by+az=0$, quedando
entonces con $x'<0$, $y'>0$ y $z'<0$, con lo que eventualmente
llegamos a $-\Upsilon$, e decir, $x<0$, $y=0$, $z>0$. Por simetr{\'\i}a
si se sigue la {\'o}rbita se llega a $\Upsilon$. En el segundo caso,
ii), pasamos a tener o bien $z>0$ con $y<0$, $x>0$, de donde
llegamos a $f(x)+by+ax=0$ y de ah{\'\i} a $x=0$, o bien a $z=0$ en
donde nos encontramos en las mismas condiciones que en el caso i).
De este modo hemos demostrado que existe una aplicaci{\'o}n continua
de $\Upsilon$ en si mismo, y por lo tanto un comportamiento
oscilatorio con al menos una {\'o}rbita peri{\'o}dica.
\end{proof}
As{\'\i}, tenemos finalmente,
\begin{theo}\label{STL1} Las {\'o}rbitas en $\mathbf{T}$ tienen un
comportamiento oscilatorio, girando alrededor del agujero de
$\mathbf{T}$, y tienen un  atractor $\mathbf{A}$ que contiene al
menos una soluci{\'o}n peri{\'o}dica
\end{theo}
\begin{proof}[{\bf Demostraci{\'o}n}.]
Del Lema (\ref{Transcont}), el flujo del sistema define una
aplicaci{\'o}n continua, de la secci{\'o}n meridiana $\Upsilon$ cerrada y
simplemente-conexa en  si mismo. El Teorema del punto fijo de
Brouwer  establece en este caso,  que la aplicaci{\'o}n tiene un punto
fijo, que corresponde a una {\'o}rbita cerrada alrededor del agujero
del  toro.
\end{proof}
\headskip 20pt
\section{5 Acotaci{\'o}n de las soluciones de los sistemas perturbados}
\pagedissolve{Wipe /D 2 /Di /V /M /O} \overlay{blanco1.png} En
esta parte probamos la acotaci{\'o}n de las soluciones de la ecuaci{\'o}n
Perturbada
\begin{equation}{\label{EPA1}}
\dddot{x}+a\ddot{x}+b\dot{x}+f(x+B\sin(\omega t))=0
\end{equation}
donde  $f$  Lipschitz, no decreciente,  con derivada continua en
0, \, $f'(0)=c>0$, y, existe $C$ tal que para toda $x$ \quad
$|f(x)|<C$.
\begin{theo}\label{epert} Dada la ecuaci{\'o}n Perturbada (\ref{EPA1}), $f$
definida como arriba, entonces existe una constante $D>0$  que
depende solo de $a$, $b$, $c$ y $C$, tal que toda soluci{\'o}n
$x(\omega,t)$, de la ecuaci{\'o}n Perturbada (\ref{EPA1}) satisface
$$|x(\omega,t)|\leq D\hspace{1cm}|\dot{x}(\omega,t)|\leq
D\hspace{1cm}|\ddot{x}(\omega,t)|\leq D,$$ dado que  $a>0$, $b>0$,
y,  $c>0$, para $t$ suficientemente grande.
\end{theo}
\newpage
{\bf\Large{Proceso de la prueba }}
\begin{figure}[h]
\centering
\includegraphics[height=3.6 cm]{franja1.png}
\caption{} \label{F1}
\end{figure}
\\
1) Si el gr{\'a}fico de  $x(\omega,t)$, $\dot{x}(\omega,t)$ permanecen
en la franja [d] para $t_0\leq t\leq t_1$, Entonces existe
$D_1(d)$ tal que
\begin{equation}\label{Ae11}
|\ddot{x}(\omega,t_1)|\leq |\ddot{x}(\omega,t_0)|+D_1(d).
\end{equation}
2) Si se cumple que el gr{\'a}fico de  $x(\omega,t)$ permanece en la
franja, para  $t\geq t_0$, entonces existe $T_0\geq t_0$ tal que
$|\ddot{x}(\omega,t)|\leq D(d)$ para  $t\geq T_0$.
\\
3) Si se cumple que el gr{\'a}fico de $\dot{x}(\omega,t)$ permanece en
la franja [$d$], entonces existe $D(d)$ tal que $|x(\omega,t)|\leq
D(d), \hspace{0.7 cm}|\ddot{x}(t)|\leq D(d)$ para  $t$
suficientemente grande.
\\
\begin{figure}[h]
\centering
\includegraphics[height=3.8 cm]{franja2.png}
\caption{ } \label{F1}
\end{figure}
4) Pero en el caso que se cumpla que el gr{\'a}fico de $x(\omega,t)$
entra y sale repetidamente de la franja cuando
$t\rightarrow\infty$, se tiene que $x(\omega,t)$ es acotada.
\\
Por lo tanto el Teorema ser{\'a} probado si mostramos que
\\
5) Para cualquier soluci{\'o}n $x(\omega,t)$, tal que
$\dot{x}(\omega,t)$ entra y sale repetidamente de la franja
[$\displaystyle{\frac{2C}{b}}$] cuando $t\rightarrow\infty$,
entonces existe $D(\displaystyle{\frac{2C}{b}})$ tal que
$\dot{x}(\omega,t)$ permanece en la la franja
\Big[$D\big(\displaystyle{\frac{2C}{b}}\big)$\Big]. \headskip20pt
\section{6 Eliminaci{\'o}n de grandes Oscilaciones}
\pagedissolve{Wipe /D 2 /Dm /V /M /O} \overlay{blanco1.png}
\color{black} En este capitulo estudiamos como la introducci{\'o}n de
una funci{\'o}n sinusoidal de suficiente amplitud y frecuencia en el
sistema (\ref{se2}), elimina las oscilaciones grandes, y deja
soluciones peri{\'o}dicas de alta frecuencia y amplitud tan peque{\~n}a
como se quiera. De hecho conseguiremos que todas las soluciones
del sistema perturbado tiendan a un entorno del origen tan peque{\~n}o
como se quiera, a base de aumentar la amplitud y la frecuencia de
la perturbaci{\'o}n.
\\
\\
Investigaremos el efecto que causa al introducir una perturbaci{\'o}n
sinusoidal de gran amplitud y frecuencia, es decir se estudia el
comportamiento de las soluciones de la ecuaci{\'o}n diferencial no
lineal de tercer grado:
\\
\begin{equation}
\displaystyle{\frac{d^3
x}{dt^3}+a\frac{d^2x}{dt^2}+b\frac{dx}{dt}+f(x+B\sin\omega t)}=0
\end{equation}
\\
o el Sistema  Perturbado equivalente
\\
\begin{equation}\label{sao8}
\begin{array}{lcl}
\dot{x}&=& y\\&&\\ \dot{y}&=& z\\&&\\\dot{z}&=&
-az-by-f(x+B\sin\omega t)
\end{array}
\end{equation}
\\
\\
para valores de $B$ y $\omega $ suficientemente grandes; donde $f$
est{\'a}  definida como en el capi\-tulo 4, es decir $f$ Lipschitz, no
decreciente\quad $f(0)=0$,\quad $f$ con derivada continua en 0 y
$f'(0)=c>ab$,\quad existe $C$ tal que para toda $x$\quad
$|f(x)|<C$ y $\displaystyle{\frac{|f(x)|}{|x|}}<C$ para $0<|x|<1$
y $f(-x)=-f(x)$
\newpage
La funci{\'o}n perturbada $f(x+B\sin\omega t)$ se puede visualizar en
general por medio de la figura \ref{afig61}
\begin{figure}[h]
\centering
\includegraphics[height=7 cm]{cap61.png}
\caption{Funci{\'o}n Perturbada}\label{afig61}
\end{figure}
\\
En el caso particular en que
\begin{equation}\label{FAP1}
f(x)=\left\{\begin{array}{lcl} c sign(x)&\textrm{ si } & |x|>1\\&&
\\
cx &\textrm{ si }&|x|\leq 1
\end{array}\right.
\end{equation}
\begin{figure}[h]
\centering
\includegraphics[height=7 cm ]{cap62.png}
\caption{Funci{\'o}n Perturbada} \label{afig62}
\end{figure}
\begin{equation}\label{eao3}
\begin{array}{lcll}
f(x+B\sin\omega t)&=&c(x+B\sin\omega  t)&0\leq\omega
t\leq\phi_1\\&&&\\ f(x+B\sin\omega t)&=&c&\phi_1\leq\omega  t\leq
\pi-\phi_1\\&&&\\ f(x+B\sin\omega t)&=&c(x+B\sin\omega
t)&\pi-\phi_1\leq\omega  t\leq \pi+\phi_2\\&&&\\ f(x+B\sin\omega
t)&=&-c&\pi+\phi_2\leq\omega  t\leq
2\pi-\phi_2\\&&&\\f(x+B\sin\omega t)&=&c(x+B\sin\omega
t)&2\pi-\phi_2\leq \omega  t\leq 2\pi
\end{array}
\end{equation}
Donde $\phi_1$ y $-\phi_2$ son los {\'a}ngulos m{\'a}s pr{\'o}ximos a 0 en que
$x+B\sin(\omega t)$ vale $c$ y $-c$ respectivamente.
\begin{equation}\label{eao4}
\phi_1(x)=\left\{\begin{array}{cll}
&\arcsin\Big(\displaystyle{\frac{1-x}{B}}\Big)&|1-x|<B\\&&\\&\quad
\displaystyle{\frac{\pi}{2}}&1-x\geq
B\\&&\\&-\displaystyle{\frac{\pi}{2}}&1-x\leq-B
\end{array}\right.
\end{equation}
\begin{equation}\label{eao5}
\phi_2(x)=\left\{\begin{array}{cll} &\arcsin\Big(\displaystyle{
\frac{1+x}{B}}\Big)&|1+x|<B\\&&\\&\quad\displaystyle{\frac{\pi}{2}}&1+x\geq
B\\&&\\&-\displaystyle{\frac{\pi}{2}}&1+x\leq- B
\end{array}\right.
\end{equation}
{\bf\Large{Funci{\'o}n Promedio}}
\begin{equation}\label{eao6}
f_0(x,B)=\frac{1}{2\pi}\int_0^{2\pi}f(x+B\sin\omega t)d(\omega  t)
\end{equation}
\begin{Prop}Para toda funci{\'o}n $f$ Lipschitz, no decreciente, acotada,
con derivada continua en 0 y $f(-x)=-f(x)$, se tiene que:
\begin{enumerate}
\item[a)]$f_0(x,B)$ es derivable en todos los valores de $x$.
\item[b)] $f_0(0,B)=0$ \item[c)]$f_0(x,B)$ es no decreciente
\item[d)] $f'_0(x,B)\rightarrow0$ y en particular
$f'_0(0,B)\rightarrow0$ cuando $B\rightarrow\infty$. \item[e)]
$|f_0(x,B)|<C$ \item[f)] $f''_0(0,B)$  es positiva en un intervalo
$(0,x_1)$ si $B$ es suficientemente grande.
\end{enumerate}
\end{Prop}
%Realizando las sustituciones propias de las ecuaciones
%(\ref{eao3}), (\ref{eao4}) y (\ref{eao5}) en la ecuaci{\'o}n
%(\ref{eao6}) e integrando se tiene
En el caso particular de la ecuaci{\'o}n (\ref{eao6}) se tiene
\begin{equation}\label{eao7}
f_0(x,B)=\left\{\begin{array}{lr} c\, sign(x)& \textrm{ si
}|x|\geq (B+1)\\&\\\rho(x) & \textrm{ si  } |x|\leq (B+1)
\\&\\
\end{array}\right.
\end{equation}
donde
$\rho(x)=\displaystyle{\frac{1}{\pi}}c[x(\phi_1+\phi_2)+B(\cos\phi_2-\cos\phi_1)+
(\phi_2-\phi_1)]$
\\
As{\'\i} que
\begin{equation}\label{eao9}
f'_0(0,B)=\displaystyle{\frac{2c}{\pi}}\arcsin(\displaystyle{\frac{1}{B}})
\end{equation}
\newpage
La funci{\'o}n promedio  $f_0(x,B)$ es dada por la Fig.\ref{afig63}
\begin{figure}[h]
\centering
\includegraphics[height=8 cm]{cap63.png}
\caption{Funci{\'o}n Promedio} \label{afig63}
\end{figure}
\newpage
{\bf\Large{Aproximaci{\'o}n}}
\\
Sea $h(t, x)=-f(x+B\sin\omega t)+f_0(x,B)$, es una funci{\'o}n
peri{\'o}dica de pe\-riodo $2\displaystyle{\frac{\pi}{\omega}}$,
acotada y promedio cero.
\\
Sea
\begin{equation}\label{ea6.1}
H(t,x,\omega,B)=\int_{-\infty}^{ \omega
t}\exp\Big[-\displaystyle{\frac{1}{\omega}}(\omega
t-\tau)\Big]h(\tau,x)\, d\tau
\end{equation}
\begin{lem}\label{la1}
Existe una funci{\'o}n continua $\eta(\omega)$ tal que
$\eta(\omega)\rightarrow0$ cuando $\omega\rightarrow\infty$\\
entonces
\begin{equation}\label{ea6.2}
|H(t,x,\omega,B)|\leq \omega\eta(\omega)
\end{equation}
\\
Finalmente,
\\
\begin{equation}\label{ea6.3}
\Big|\displaystyle{\frac{1}{\omega}\frac{\partial H}{\partial t}
}-h(t,x)\Big|\leq \eta(\omega)
\end{equation}
\end{lem}
\newpage
{\bf\Large{Equivalencia del Sistema Perturbado}}
\begin{theo}\label{P621} El Sistema Perturbado es equivalente al Sistema
\\
\begin{equation}\label{PP6621}
\begin{array}{lcl}
\dot{x}&=&y\\&&\\
\dot{y}&=&z+\displaystyle{\frac{1}{\omega}}H\Big(
t,x,\omega, B\Big)\\&&\\
\dot{z}&=&-az-by-f_0(x,B)-\displaystyle{\frac{a-1}{\omega}}H(t,x,\omega,B)-
\displaystyle{\frac{1}{\omega}\frac{\partial H}{\partial x}}y
\end{array}
\end{equation}
\end{theo}
\begin{theo}\label{N61}El sistema Perturbado tiende al promediado cuando
$\omega \rightarrow\infty $
\end{theo}
{\bf\Large{Existencia y Anulaci{\'o}n de Oscilaciones}}
\\
\\
Sean $a>0$\quad $b>0$\quad $f$ Lipschitz, no decreciente,\quad
$f(0)=0$,\quad $f$ con derivada continua en 0 y $f(-x)=-f(x)$
\begin{obs}Bajo las hip{\'o}tesis en (\ref{sao8}) tenemos
\begin{enumerate}\item[a)] Si
$ab<f'_0(0,B)<c$, y $a^2>4b$, entonces el sistema promediado tiene
un atractor oscilatorio alejado del origen.
\\
\item[b)]Si  $f'_0(x,B)<ab$ para toda $x$, entonces  el sistema
promediado tiene al origen como atractor global. Tal ser{\'a} si $B$
es suficientemente grande.
\end{enumerate}
\end{obs}
Utilizando el teorema inverso de Lyapunov que nos asegura la
existencia de una funci{\'o}n de Lyapunov en el caso de estabilidad
asint{\'o}tica, y tomando en cuenta que el sistema perturbado tiende
al promediado cuando $\omega\rightarrow\infty$, obtenemos el
resultado final
\begin{theo}
Existe un $B_0$ tal que si $B>B_0$, entonces el origen del sistema
promediado es globalmente asint{\'o}ticamente estable. En tal caso,
dada $\epsilon>0$, existe   $\omega_0$ tal que si
$\omega>\omega_0$, las soluciones del sistema perturbado
permanecen en un entorno de radio $\epsilon$ de $\mathbf{O}$ para
valores de $t$ mayores que un valor dependiente de la soluci{\'o}n.
\end{theo}
\end{document}

