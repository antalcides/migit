




Se considera el operador el\'{\i}ptico sim\'{e}trico
\begin{equation}
Lu=\sum_{i,j=1}^{n}\frac{\partial}{\partial x_{i}}\left(  a_{ij}\left(
x\right)  \frac{\partial}{\partial x_{j}}u\right)  ,\tag{12}%
\end{equation}
donde $a_{ij}\left(  x\right)  \in C^{\infty}\left(  \overline{\Omega}\right)
$ y $\partial\Omega$ es suave.

Del teorema $10$ se tiene que $\lambda_{1}$, valor propio principal de $-L$,
es positivo y existe $w_{1}>0$ tal que
\[
\left\{
\begin{array}
[c]{c}%
Lw_{1}+\lambda_{1}w_{1}=0\text{ \ en }\Omega,\\
w_{1}=0\text{ \ sobre }\partial\Omega.
\end{array}
\right.
\]
Adem\'{a}s se sabe que
\[
\left(  PMF\right)  \left\{
\begin{array}
[c]{c}%
Lu+cu\leq0\text{ \ en }\Omega\text{ y }u\geq0\text{ sobre }\partial\Omega,\\
\text{implica}\\
u\equiv0\text{ \ en }\Omega\text{ \ \'{o} \ }u>0\text{ \ en }\Omega\text{,}%
\end{array}
\right.
\]
es verdadera si $c\leq0$ y falsa si $c\equiv\lambda_{1}.$

\begin{theorem}
Dado $L$ en $\left(  12\right)  $, los coeficientes de $L$, $\Omega$ y $u$
satisfaciendo las mismas hip\'{o}tesis del teorema $12.$ Entonces $\left(
PMF\right)  $ es verdadero si $c(x)\lvertneqq\lambda_{1}$ en $\Omega.$
\end{theorem}

%\begin{proof}
%Es consecuencia del teorema $12$ tomando $h:=w_{1}.$
%\end{proof}
