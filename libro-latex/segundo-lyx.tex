% Vista preliminar del código fuente

%% LyX 2.0.4 created this file.  For more info, see http://www.lyx.org/.
%% Do not edit unless you really know what you are doing.
\documentclass[spanish,svgnames,x11names,x11names,HTML]{book}
\usepackage[utf8x]{inputenc}
\pagestyle{headings}
\setcounter{secnumdepth}{3}
\setcounter{tocdepth}{3}
\usepackage{float}
\usepackage{textcomp}
\usepackage{url}
\usepackage{amssymb}
\usepackage{graphicx}

\makeatletter

%%%%%%%%%%%%%%%%%%%%%%%%%%%%%% LyX specific LaTeX commands.
\providecommand{\LyX}{L\kern-.1667em\lower.25em\hbox{Y}\kern-.125emX\@}
%% A simple dot to overcome graphicx limitations
\newcommand{\lyxdot}{.}


%%%%%%%%%%%%%%%%%%%%%%%%%%%%%% Textclass specific LaTeX commands.
\newenvironment{lyxlist}[1]
{\begin{list}{}
{\settowidth{\labelwidth}{#1}
 \setlength{\leftmargin}{\labelwidth}
 \addtolength{\leftmargin}{\labelsep}
 \renewcommand{\makelabel}[1]{##1\hfil}}}
{\end{list}}

%%%%%%%%%%%%%%%%%%%%%%%%%%%%%% User specified LaTeX commands.
\input{preamble2}
\input{composicion-codigo}
\usepackage{subfloat}
 \usepackage{morewrites}
\usepackage{fancyvrb}
\usepackage[unicode=true,pdfusetitle,
 bookmarks=true,bookmarksnumbered=true,bookmarksopen=true,bookmarksopenlevel=1,
 breaklinks=false,pdfborder={0 0 0},backref=false,colorlinks=false]
 {hyperref}
\hypersetup{
 pdfpagelayout=OneColumn, pdfnewwindow=true, pdfstartview=XYZ, plainpages=false}
 \newsavebox{\LstBox}
 %\graphicspath{ {./img/} }
 \makeatletter
\def\input@path{{./}{./build/}}
\makeatother
 \graphicspath{{./}{./img/}{./build/}}
 \renewcommand{\baselinestretch}{1.5}
\makeatother

\usepackage{babel}
\addto\shorthandsspanish{\spanishdeactivate{~<>}}

\begin{document}
\pagecolor{white} \mainmatter \pagestyle{headings}
\chapter{Edici\'on B\'asica}
\chaptertoc 
\begin{objetivos}
 \begin{lista} 
\item Comprender el esquema básico de funcionamiento de \TeX{}\,
y \LaTeX{}. 

\item Conocer las diferentes salidas que produce \LaTeX{}. 

\item Conocer las diferentes herramientas que interactúan con \LaTeX{}. 

\item Aprender a instalar \LaTeX{}\, en diferentes sistemas. \end{lista}
\end{objetivos} 


\section{Introducción}

A diferencia de un procesador de textos como Writer, con \LaTeX{}
tenemos un control más adecuado sobre cualquier aspecto tipográfico
del documento. 

\LaTeX{} formatea las páginas de acuerdo a la clase de documento especificado
por el comando \lstinline+\documentclass{}+, por ejemplo, \lstinline+\documentclass{report}+
formatea el documento de tal forma que el producto es un documento
con formato de artículo. 

Un documento \LaTeX{} puede tener texto ordinario junto con texto
en modo matemático. Los comandos vienen precedidos por el símbolo
“\lstinline+\+ ” (barra invertida).

Hay comandos que funcionan en modo texto y hay comandos que solo funcionan
en modo matemático, pero para escribir en modo matemático hay varios
entornos el más común es el entorno delimitado por dos signos de dólar
(\lstinline+$...$+) .


\section{Ordenes en \protect\TeX{}\, y \protect\LaTeX{}:}
\begin{description}
\item [{$\blacklozenge$}] Comienzan por una barra invertida: (( $\backslash$))
\item [{$\blacklozenge$}] Distinguen mayúsculas y minúsculas
\item [{$\blacklozenge$}] $\ $Dos tipos: 


1. sólo con letras (pueden ser varias)

\end{description}
2. con carácter especial (uno sólo)
\begin{description}
\item [{$\blacklozenge$}] \TeX{}\, ignora los espacios en blanco justo
después de un mandato: para tenerlos en cuenta, escribir \lstinline+\,+
\item [{$\blacklozenge$}] Parámetros: {[}opcionales{]} y \{obligatorios\}
, es decir
\end{description}

\subsection{Ejemplos de comandos}
\begin{description}
\item [{$\spadesuit$}] Comentarios: a partir de signo \%, son ignorados 
\end{description}
Veamos algunas ordenes: \lstinline+\TeX \LaTeX+ Como podemos observar
los dos logos aparecen juntos \\
 \TeX{}\LaTeX{}

%  \\es una orden de tipo 2
para que se separen debemos colocar un comando de indique el espacio,
por ejemplo un espacio normal\\
 \lstinline+\TeX{}\,\LaTeX{}\\[2ex] \today\\[4ex]+

\TeX{}\,\LaTeX{}\\[2ex] \today\\[4ex]

\lstinline+\textbf{texto resaltado}+

\textbf{texto resaltado}


\subsection{Caracteres especiales}

Los caracteres con un significado especial, si se desean transcribir
hay que indicarlo de alguna manera: \begin{verbatim} $\$\&\%\#\_\{\}\verb+~^{\+}\end{verbatim}
\section{Mi primer documento}
\subsection{Estructura de un fichero de entrada}
Cuando\LaTeXe{}procesa un fichero de entrada, espera de\'el que siga una determinada \wi{estructura}. Todo fichero de entrada debe comenzar con la orden 
\begin{code}\lstinline+\documentclass{...}+\end{code} 
Esto indica qu\'e tipo de documento es el que sepretende crear.
Tras esto, se pueden incluir \'ordenes que influir\'an sobre el estilo del documento entero, o puede cargar \wi{paquete}s que a~nadir\'an nuevas propiedades al sistema de \LaTeX. Para cargar uno de estos paquetes se usar\'a la instrucci\'on
 \begin{code}\lstinline+\usepackage{...}+\end{code} 
 Cuando todo el trabajo deconfiguraci\'on est'e realizado \footnote{El \'area entre\texttt{\bs documentclass} y \texttt{\bs begin\ensuremath{\mathtt{\{}}document\ensuremath{\mathtt{\}}}} se llama \emph{\wi{preámbulo}}.} entonces comienza el cuerpo del texto con la instrucci\'on 
 \begin{code}\lstinline|\begin{document}|
 \end{code} 
 A partir de entonces se introducir\'a el texto mezclado con algunas instrucciones\'utiles de \LaTeX. Al finalizar el documento debe ponerse la orden 
 \begin{code}\lstinline|\end{documeent}|\end{code}
  LaTeX{} ingorar\'a cualquier cosa que se ponga tras esta instrucci\'on. La figura~\ref{mini} muestra el contenido m\'inimo de un fichero de \LaTeXe. En la figura~\ref{document} se expone un \wi{fichero de entrada} algo m\'as complejo.
  %\begin{example} \end{example}
   Ejemplo para un art\'iculo cient\'ifico en español \begin{problema}
   \documentclass[a4paper,11pt]{article}
   \usepackage{latexsym}
   \usepackage[activeacute,spanish]{babel}
   \author{H.~Partl}
   \title{Minimizando}
   \frenchspacing
   \begin{document}
   \maketitle
   \tableofcontents
   \section{Inicio}
   Bien\ ldots{} y aqu\'i comienza mi art\'iculo tan estupendo.
   \section{Fin}
   \ldots{} y aqu\'i acaba.\end{document}
   \end{problema}
   \section{El formato del documento}
   \subsection{Clases de documentos} Cuando procesa un fichero de entrada,lo primero que necesita saber \LaTeX{} es el tipo de documento que el autor quiere crear. Esto se indica con la instrucci\'on \ci{documentclass}.\begin{command} 
   \ci{documentclass}\ \lstinline+{opciones}{clase}+\end{command}\noindent En este caso, la \emph{clase} indica el tipo de documento que crear\'a. En la tabla~\ref{documentclasses} se muestran las clases de documento que se explican en esta introducci\'on. La distribuci\'on de \LaTeXe{} proporciona m\'as clases para otros documentos, como cartas y transparencias. El par\'ametro de \emph{\wi{opciones}} personaliza el comportamiento de la clase de documento elegida. Las opciones se deben separar con comas. En latabla~\ref{options} se indican las opciones m\'as comunes de las clases de documento est\'andares. 
   \begin{table}[!bp]
   \caption{Clasesdedocumentos}
   \label{documentclasses}
   \begin{description}
   \item[\normalfont\texttt{article}] para art\'iculos de revistas especializadas, ponencias, trabajos de pr\'acticas de formaci\'on, trabajos de seminarios, informes peque'nos, solicitudes, dict\'amenes, descripciones de programas, invitaciones y muchos otros. 
   \index{articulo@art\'iculo} 
   \index{clase\texttt{article}@clasearticle}
    \item[\normalfont\texttt{report}] para informes mayores que constan dem\'as de un cap\'itulo, proyectos fin de carrera, tesis doctorales, libros peque'nos, disertaciones, guiones y similares.
   \index{informe} \index{clase\texttt{report}@clasereport} \item[\normalfont\texttt{book}] para libros de verdad
   \index{libro}
   \index{clase\texttt{book}@clasebook}
   \item[\normalfont\texttt{slide}] para transparencias. Esta clase emplea tipos grandes\textsf{sans serif}.\index{transparencias}\index{clase\texttt{slide}@claseslide}
   \end{description}
   \end{table}
   \begin{table}[!bp]
     \caption{Opciones de clases de documento}
     \label{options}
     \begin{flushleft}
     \begin{description}
     \item[\normalfont\texttt{10pt},\texttt{11pt},\texttt{12pt}]
     \quad Establecen el tama~no(cuerpo) para los tipos. Si no se especifica ninguna opci\'on, se toma\texttt{10pt}.\index{tama~no de lostipos!deldocumento}
     \item[\normalfont\texttt{a4paper},\texttt{letterpaper},\ldots]\quad Define el tama~no del papel. Si no se indica nada, se toma \texttt{letterpaper}. Aparte de 
     \end{description}
     \end{flushleft}
     \end{table}
     Por ejemplo: un fichero de entrada para un documento de \LaTeX{} podr\'ia comenzar con 
     \begin{code}\ci{documentclass}\verb|[11pt,twoside,a4paper]{article}|\end{code}
     Esto le indica a \LaTeX{} que componga el documento como un \emph{art\'iculo} utilizando tipos del cuerpo 11, y que produzca un formato para impresi\'on a \emph{doble cara} en \emph{papel DIN-A4}.
     \pagebreak[2]
     \subsection{Paquetes}
     \index{paquete}
     Mientras escribe su documento,probablemente se encontrar\'a en situaciones  donde el \LaTeX{}  b\'asico no basta para solucionar  su
     problema.
     
     Si  desea incluir \wi{gr\'aficos}\wi{texto en color} o el c\'odigo fuente de un fichero, necesita mejorar las capacidades de \LaTeX.
     Tales mejoras se realizan con ayuda de los llamados \emph{paquetes.} Los paquetes se activan con la orden 
     \begin{command} \ci{usepackage}\verb|[|\emph{opciones}\verb|]{|\emph{paquete}\verb|}|\end{command}\noindent donde \emph{paquete} es el nombre del paquete y \emph{opciones} es una lista palabras clave que activan funciones especiales del paquete, a las que \LaTeX{} les a~nade las opciones que previamente se hayan indicado en la orden \verb|\documentclass|.
     Algunos paquetes vienen con la distribuci\'on b\'asica de \LaTeXe{}(v'easelatabla~\ref{packages}).
     Otros se proporcionan por separado.
     En la\guia{} puede encontrar m\'as informaci\'on sobre los paquetes disponibles en su instalaci\'on local.
     La fuente principal de informaci\'on sobre\LaTeX{} es\companion.
     Contiene descripciones de cientos depaquetes, as\'i como informaci\'on sobre c\'omo escribir sus propias extensiones a \LaTeXe.
     \begin{table}[!hbp] \caption{Algunos paquetes distribuidos con LaTeX}
     \label{packages}
     \begin{description}
     \item[\normalfont\pai{doc}] Permite la documentaci\'on de paquetes y otros ficheros de \LaTeX.
     \end{description}
     \end{table}
     \clearpage
     \subsection{Estilo de p\'agina} Con \LaTeX{} existen tres combinaciones predefinidas de \wi{cabeceras} y \wi{pies de p\'agina}, a las que se llamane stilos dep\'agina.
     \index{estilo de pagina@estilo dep\'agina} El par\'ametro \emph{estilo} de la instrucci\'on 
     \index{estilo de pagina@estilo de p\'agina!plain@\texttt{plain}}
     \index{plain@\texttt{plain}}\index{estilo de pagina@estilo de p\'agina!headings@\texttt{headings}}
     \index{headings@\texttt{headings}}
     \index{estilo de pagina@estilo dep\'agina!empty@\texttt{empty}}
     \index{empty@\texttt{empty}}
     \begin{command}
     \ci{pagestyle}\verb|{|\emph{estilo}\verb|}|
     \end{command}
     \noindent define cu\'al emplearse.
     La tabla~\ref{pagestyle} muestra los estilos dep\'agina predefinidos.\begin{table}[!hbp]
     \caption{Estilos dep\'agina predefinidos en \LaTeX}
     \label{pagestyle}
     \begin{description}
     \item[\normalfont\texttt{plain}]imprime los n\'umeros dep\'agina en el centro del pie de las p\'aginas.
     Este es el estilo dep\'agina que se toma si no se indica ning\'un otro.
     \item[\normalfont\texttt{headings}] en la cabecera de cada p\'agina imprime el cap\'itulo que se est\'a procesando y el n\'umero dep\'agina, mientras que el pie est\'a vac\'io.
     (Este estilo es similar al empleado en este documento).
     \item[\normalfont\texttt{empty}]deja tanto la cabecera como el pie de las p\'aginas vac\'ios.
     \end{description}
     \end{table}
     Es posible cambiar el estilo de p\'agina de la p\'agina actual con la instrucci\'on 
     \begin{command}
     \ci{thispagestyle}\verb|{|\emph{estilo}\verb|}|\end{command}
     En \companion{} hay una descripci\'on de c\'omo crear sus propias cabeceras y pies dep\'agina.
     \section{Proyectosgrandes} Cuando trabaje con documentos grandes, podr\'ia, si lo desea, dividir el fichero de entrada en varias partes.
     \LaTeX{} tiene dos instrucciones que le ayudan a realizar esto.
     \begin{command}\ci{include}\verb|{|\emph{fichero}\verb|}|\end{command}
     \noindent se puede utilizar en el cuerpo del documento para introducir el contenido de otro fichero.
     En este caso,\LaTeX{} comenzar\'a una p\'agina nueva antes de procesar el texto del \emph{fichero}.
     La segunda instrucci\'on s\'olo puede ser empleada en el pre\'ambulo.
     Permite indicarle a \LaTeX{} que s\'olo tome la entrada de algunos ficheros de los indicados con\verb|\include|.
     \begin{command}\ci{includeonly}\verb|{|\emph{fichero}\verb|,|\emph{fichero}\verb|,|\ldots\verb|}|\end{command}
     Una vez que esta instrucci\'on se ejecute en el pre\'ambulo del documento, s\'olo se procesar\'an las instrucciones \ci{include} con los ficheros indicados en el rgumento de la orden \ci{includeonly}.
     Observe que no hay espacios entre los nombres de los ficheros y las comas.
     \section{resumen} Todo documento en\LaTeX est\'{a} compuesto de dos partes
     \begin{description}
     \item[$\Diamond$] El preambulo :\\
En esta parte de colocan las ordenes global es para el documento, además
de los paquetes de \LaTeX{}que se usarán

\item$Diamond$ El Body, este está dividido a su vez entres parte
el Front matter, main matter y el back matter 
Para empezar explicaremos como se diseña un artículo
\begin{description}
\item [{$blacksquare$}] Se escribe el código
\end{description}
\begin{verbatim}
 \documentclass[a4paper]{article}
  \usepackage[spanish,activeacute]{babel} 


\author{Pon tu nombre aquí}


\title{Mi Primer Documento}

\maketitle
\begin{document} Hola . Este es mi primer documento . \end{document}
\end{verbatim}
\begin{description}
\item [{$blacksquare$}] Se realiza el proceso de compilación

\begin{description}
\item [{$clubsuit$}] Compilar: 
\end{description}
\item [{>latex}] archivo.tex

\begin{description}
\item [{$clubsuit$}] Pre-visualizar: 
\end{description}
\item [{>xdvi}] archivo.dvi 
\end{description}
$\clubsuit$\textbf{Generar Post-Script:}
\begin{description}
\item [{>dvips}] archivo.dvi -o archivo.ps

\begin{description}
\item [{$clubsuit$}] Imprimir: 
\end{description}
\item [{>lpr}] -Plaser1sala4 archivo.ps 
\end{description}
\begin{figure}[H]
\includegraphics[scale=0.3]{diagrama} 
\end{figure}
\end{description}
\end{document}
