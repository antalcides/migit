\chapter{tercer}
\chaptertoc 
\begin{objetivos}
\begin{lista}
\item Comprender el esquema básico de funcionamiento de \TeX\, y \LaTeX .
\item Conocer las diferentes salidas que produce \LaTeX.
\item Conocer las diferentes herramientas que intecactuan con \LaTeX.
\item Aprender a instalar \LaTeX\, en diferentes sistemas.
\end{lista}
\end{objetivos}
\section{Reglas generales}

\begin{description}
\item[$\clubsuit$] Usar espacios para separar palabras.

\item Un espacio vale igual que mil.

\begin{description}
\item[$\clubsuit$] Los fines de linea sencillos no valen.

\item[$\clubsuit$] Usar líneas vac\'{\i}as para separar p\'{a}rrafos.
\end{description}
\end{description}

Una linea vac\'{\i}a vale igual que mil.

\begin{description}
\item
\begin{description}
\item[$\clubsuit$] El espaciado y las sangr\'{\i}as son trabajo de
\end{description}

\item LATEX, y lo sabe hacer muy bien.

\begin{description}
\item[$\clubsuit$] No forzar espacios ni cortes de l\'{\i}nea.
\end{description}
\end{description}

\subsection{Ejemplo 1}
\begin{verbatim}
\begin{document}
\maketitle Este es el primer párrafo ,
y esta sigue
siendo par te del primer p árrafo
 Este ya es e l segundo párrafo .
%y esto es un comentario
Aquí puedes e s c r i b i r más .
\ end{ document }

\end{verbatim}

\subsection{Ejemplo 2}
\begin{verbatim}

\begin{document}
\maketitle Este es un ejemplo con un
p á r ra fo m\'{a}s grande
que , por c i e r t o , tambi \'{e}n
es mucho m\'{a}s i n te re sa
n te . Recuerda que un p\'{a} r ra fo
debe expresar una idea
completa y coherente . Justo como este
 p \'{a} r ra fo que nos ha
servido como un ejemplo genial . Observa
que los p \'{a} r ra fo s
en \ LaTeX { } forman l a unidad
e s t r u c t u r a l m\'{a}s
peque~na dent ro de los documentos .
 Recuerda que es tu responsabilidad
  e l contenido de estos
párrafos , y de \ LaTeX { } e l
que se vean boni tos . \ end{ document }
\end{verbatim}

\subsection{Acentos}

La opci\'{o}n activeacute de babel permite

usar acentos cortos: \'{a}, \'{a}, \'{a}, \~{n}, etc.

Los acentos cortos no funcionan en el

pre\'{a}mbulo, all\'{\i} hay que usar acentos largos:

\'{a}
%TCIMACRO{\TEXTsymbol{\backslash}}%
%BeginExpansion
$\backslash$%
%EndExpansion
\'a \'{o}
%TCIMACRO{\TEXTsymbol{\backslash}}%
%BeginExpansion
$\backslash$%
%EndExpansion
\'o

\'{e}
%TCIMACRO{\TEXTsymbol{\backslash}}%
%BeginExpansion
$\backslash$%
%EndExpansion
\'e \'{u}
%TCIMACRO{\TEXTsymbol{\backslash}}%
%BeginExpansion
$\backslash$%
%EndExpansion
\'u

\'{\i}
%TCIMACRO{\TEXTsymbol{\backslash}}%
%BeginExpansion
$\backslash$%
%EndExpansion
'\{%
%TCIMACRO{\TEXTsymbol{\backslash}}%
%BeginExpansion
$\backslash$%
%EndExpansion
i\} \~{n}
%TCIMACRO{\TEXTsymbol{\backslash}}%
%BeginExpansion
$\backslash$%
%EndExpansion%
%TCIMACRO{\U{2dc}}%
%BeginExpansion
\protect\rule{0.1in}{0.1in}
%EndExpansion
n

\textquestiondown Por qu\'{e} no usar directamente los caracteres

acentuados en mi c\'{o}digo de \LaTeX?

Tambi\'{e}n podemos usar el paquete inputenc con la opci\'{o}n latin1

\subsection{F\'{o}rmulas en l\'{\i}neas}

\begin{description}
\item[$\blacksquare$] Los signos \$ \$ son para indicar el contenido
\end{description}

matem\'{a}tico.

\begin{description}
\item[$\blacksquare$] Todo el contenido matem\'{a}tico (y s\'{o}lo el
\end{description}

contenido matem\'{a}tico) debe de ser marcado.

\begin{description}
\item[$\blacksquare$] No usar el contenido matem\'{a}tico para poner
\end{description}

it\'{a}licas.

\begin{description}
\item[$\blacksquare$] Y no usar comandos de formato para marcar
\end{description}

contenido matem\'{a}tico.

\begin{description}
\item[$\blacksquare$] Pensar en el contenido, \textexclamdown no en el formato!.
\end{description}

\subsection{Mas ejemplos}

\begin{example}
Haciendo salvedad de ((efectos es-peciales)),
para escribir un texto
normal en \TeX \,basta con teclear
exactamente el texto que se de-
sea. El cajista (\TeX) se ocupa de
formar y ajustar las l\'{\i}neas. Para
separar las palabras se emplean
espacios en blanco o ((retornos de
carro)) (nueva l\'{\i}nea). El n\'{u}mero
de espacios en blanco no impor-
ta: uno es igual que                100.
\end{example}
\subsection{Tipos de documentos}
\begin{description}
\item[$\blacksquare$] Clase del documento
\item (\lstinline+\documentclass[...]{clase}+):

\item[$\bigstar$]  article: art\'{\i}culos, trabajos, : : :

\item[$\bigstar$]  letter: cartas

\item[$\bigstar$]  report, book: documentos m\'{a}s largos, con cap\'{\i}tulos

\item[$\bigstar$]  slides: presentaciones (transparencias)
\item[$\blacksquare$] Par\'{a}metros opcionales (\lstinline+\documentclass[opciones]{...}+):
\end{description}

10pt, 11pt, 12pt: tama\~{n}os o tipos

letterpaper, a4paper:  tama\~{n}o o papel

twocolumn: dos columnas

